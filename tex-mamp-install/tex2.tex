\documentclass[uplatex, dvipdfmx]{jsarticle}

\include{begin}

\section{MAMPの設定}

\subsection{PHPの設定}

MAMPがせっかくPHP8に対応したが、PHP8だとWordPressを動かした場合、
不具合が発生するので、PHP7を動かすことにする。

C:\yen MAMP\yen bin\yen php にある以下のフォルダの名前を変更する。

\vspace{3mm}
\begin{tabular}{|lll|} \hline
php8.0.1 & $\rightarrow$ & \_php8.0.1 \\
php8.1.0 & $\rightarrow$ & \_php8.1.0 \\ \hline
\end{tabular}
\vspace{3mm}

PHPの設定ファイルは以下である。(php v7.4.16 の場合)

"C:\yen MAMP\yen conf\yen php7.4.16\yen php.ini"

一応バックアップはとっておく。

\fbox{pnp.ini\_org} という名前でオリジナルファイルは残しておく。

\begin{lstlisting}[caption=php.ini]
307 memory_limit = 512M

375 display_errors = on

494 post_max_size = 512M

518 default_charset = "UTF-8"

607 upload_max_filesize = 512M

705 date.timezone = Asia/Tokyo
\end{lstlisting}

307, 494, 607 を 512M としたのは、今後 WordPress でバックアップファイルから
復元するときのことを考慮したのである。

375 で on とすることにより、エラーがブラウザに出力されるようになる。

518 で UTF-8 と指定しているが、これは指定しなくても UTF-8 になるようである。

705 でタイムゾーンを指定している。これは必須である。

要するに、絶対に指定しなければならないのは、705行めのタイムゾーンだけである。


\subsection{MySQLの設定}

設定ファイルは C:\yen MAMP\yen conf\yen mysql\yen my.ini である。

オリジナルは my.ini\_org としてバックアップしておく。

\begin{lstlisting}[caption=my.ini]
...
40 character-set-server=utf8 
41 collation-server=utf8_general_ci
...
\end{lstlisting}

とあるが、これを以下のようにしておく。

\begin{lstlisting}[caption=my.ini]
...
40 character-set-server=utf8mb4 
41 collation-server=utf8mb4_general_ci
...
\end{lstlisting}

UTF-8の文字コードは本来4バイトなのだが、
MySQLで使われてきたUTF-8は3バイトであった。
そのことにより使用できる文字が制限されてきた。

MySQL5.7.9 以降では UTF-8mb4(4バイト)が使えるようになった。





\include{end}

%% 修正時刻: Sat 2022/03/19 18:01:172
