\documentclass[uplatex, dvipdfmx]{jsarticle}

\include{begin}

\section{MySQLにログインする}

\subsection{root(管理者)でログインする}

データベースを利用するためには、まず、そのデータベースを管理している人
(管理者)から、アカウント(ユーザー名とパスワード)を発行してもらわなくては
ならない。

そして、通常は1つのデータベースが与えられ、そのデータベースの中に
複数のテーブル(表)を作成していくことになる。

しかし、ここでは 管理者のままで MySQLデータベースを操作していくことにする。
そして、操作に慣れたら、一般ユーザーを作成し、一般ユーザーとして
データベースを操作していくことにする。

\vspace{6mm}
\textgt{管理者(root) でのログイン}

\begin{lstlisting}[numbers=none]
> mysql -u root -p
Enter password: (何も入力せずに Enter)
\end{lstlisting}

xamppをインストールした場合、MySQL(MariaDB)にぱパスワードは設定されていない。
もし、パスワードを設定すれば、ここでパスワードを入力することになる。

パスワードが設定されている場合、以下のようにパスワードも含めて、
一行で入力することもできる。
パスワードは -p のあと、空白をはさまずに続けて書く。

 \fbox{mysql$>$ mysql -u root -proot}
 \footnote{\textsf{mysql} コマンドは、
 ``C:\yen xampp\yen mysql\yen bin'' の
 中の ``mysql.exe'' のことである。
 このフォルダには、他にも ``mysqldump.exe'' など
 いろいろなコマンドが置かれている。}


\section{データベースを設計する}

\subsection{扱うデータ}

以下のようなデータを扱うこととする。

\vspace{3mm}
 \begin{tabular}{|c|} \hline
  菅原文太 \\
  40歳 \\
  1933年生まれ \\
  総務部 \\ \hline
 \end{tabular}
 \quad
 \begin{tabular}{|c|} \hline
  千葉真一 \\
  34歳 \\
  1939年生まれ \\
  営業部 \\ \hline
 \end{tabular}
 \quad
 \begin{tabular}{|c|} \hline
  北大路欣也 \\
  30歳 \\
  1943年生まれ \\
  経理部 \\ \hline
 \end{tabular}
 \quad
 \begin{tabular}{|c|} \hline
  梶芽衣子 \\
  26歳 \\
  1947年生まれ \\
  営業部 \\ \hline
 \end{tabular}
\vspace{3mm}

あなたがプログラマで、上のような社員名簿アプリを作成することになったとする。
PHP か Java でアプリを作成することになる。クライアントの会社の総務部がこの
アプリを使うことになる。そのアプリには社員の登録画面、一覧画面、編集画面、
削除画面などがあるだろう。そういった画面と処理をあなたは作らなければならない。

そのときに、データを保存するしくみとして、データベースを使うことになる。
かりに PHP でプログラミングするならば、PHPという言語を使って
データベースを操作することになる。

\subsection{どのような表をつくるか?}

データベースは表の形でイメージすることができる。
しかし、上記のデータを見て、それをそのまま表にしてはいけない。

\vspace{3mm}
\includegraphics[width=10cm]{img/table01.png}
\vspace{3mm}

この表は、1件のデータが縦に配置されて、それが人数分横に続いている。これは良くない。

次の表のように、1件のデータを横に配置する。

\vspace{3mm}
\includegraphics[width=10cm]{img/table02.png}
\vspace{3mm}

そして、縦には同じ種類のデータが並ぶ。
だから、それぞれの列には、その列の内容を表す項目名をつけることができる。

この列のことを \textgt{カラム}(項目) という。(フィールドともいう)

そして、1件のデータを表す横1行を \textgt{レコード} という。
この表には4件のデータがあり、カラムは4である。

しかし、これだけではデータベースにはならない。
各レコードには、そのレコードの独自性を保証するデータが必要なのである。
それを プライマリー・キー という。

\subsection{primary key}

データベースにデータを格納する際には、そのデータに primary key (独自キー) が必要となる。
primary key とは、そのデータを他と区別するためのデータである。
菅原文太というデータは、この4つの中では独自であるが、他のデータを追加する際に、同じデータに出会う可能性
(同姓同名)を排除できない。
さらに日本語である以上、文字コードの問題を避けることもできない。つまり、同じ菅原文太という文字でも
UTF-8 と Shift\_JIS では別物と判定されるのである。

となると、この4つのデータには primary key となるものがないということになる。

このような場合、データベースの設計者が primary key を追加することになる。
ここでは 数字を primary key として追加する。
つまり、菅原文太は 1、千葉真一は 2 というふうにする。

そして、その項目名を ここでは id とした。

\vspace{3mm}
\includegraphics[width=10cm]{img/table03.png}
\vspace{3mm}

primary key には、数字やコードが使われる。
\footnote{'001'や'C001'など、固定長の文字列がよく使われる。また、整数もよく使われる。}


\section{データベースを作成する}

\subsection{データベースの作成}

''rensyu'' というデータベースを作成する。

\begin{tcolorbox}
 MariaDB [(none)]$>$ create database rensyu \ $<$Enterキー$>$
\end{tcolorbox}

このように入力すると、以下のようになる。

\begin{tcolorbox}
 MariaDB [(none)]$>$ create database rensyu \\
 \qquad -$>$ 
\end{tcolorbox}

これは、入力の終わりがまだないので、次の入力を受け付けているのである。

入力の終わりは ``;''(セミコロン) あるいは ``\textbackslash g'' である。
 
\begin{tcolorbox}
 MariaDB [(none)]$>$ create database rensyu \\
 \qquad -$>$ ;  \ $<$Enterキー$>$
\end{tcolorbox}

``;''(セミコロン) あるいは ''\textbackslash g'' を入力して <Enterキー>
を押す。

\subsection{データベースの確認}

データベースがちゃんと作成できたか、確認する。

\begin{tcolorbox}
 MariaDB [(none)]$>$ show databases; \hspace{3mm} (複数形)
\end{tcolorbox}

\begin{lstlisting}[numbers=none]
+--------------------+
| Database           |
+--------------------+
| information_schema |
| rensyu             |
+--------------------+
\end{lstlisting}


\subsection{データベースの使用宣言}

まず、使用宣言を行う。

\begin{tcolorbox}
 MariaDB [(none)]$>$ use rensyu;
\end{tcolorbox}

\fbox{Database changed} と表示される。



\include{end}

%% 修正時刻: Sat 2023/03/25 12:26:110
