\documentclass[uplatex, dvipdfmx]{jsarticle}


\usepackage{tcolorbox}
\usepackage{color}
\usepackage{listings, plistings}

%% ノート/latexメモ
%% http://pepper.is.sci.toho-u.ac.jp/pepper/index.php?%A5%CE%A1%BC%A5%C8%2Flatex%A5%E1%A5%E2

%% JavaScriptの設定
%% https://e8l.hatenablog.com/entry/2015/11/29/232800
\lstdefinelanguage{javascript}{
  morekeywords = [1]{ %keywords
    await, break, case, catch, class, const, continue, debugger, default, delete, 
    do, else, enum, export, extends, finally, for, function, function*, if, implements, import, in, 
    instanceof, interface, let, new, package, private, protected, public, return, static, super,
    switch, this, throw, try, typeof, var, void, while, with, yield, yield*
  },
  morekeywords = [2]{ %literal
    false, Infinity, NaN, null, true, undefined
  },
  morekeywords = [3] { %Classes
    Array, ArrayBuffer, Boolean, DataView, Date, Error, EvalError, Float32Array, Float64Array,
    Function, Generator, GeneratorFunction, Int16Array, Int32Array, Int8Array, InternalError,
    JSON, Map, Math, Number, Object, Promise, Proxy, RangeError, ReferenceError, Reflect,
    RegExp, Set, String, Symbol, SyntaxError, TypeError, URIError, Uint16Array, Uint32Array,
    Uint8Array, Uint8ClampedArray, WeakMap, WeakSet
  },
  morecomment = [l]{//},
  morecomment = [s]{/*}{*/},
  morestring = [b]{"},
  morestring = [b]{'},
  alsodigit = {-},
  sensitive = true
}

%% 修正時刻: Tue 2022/03/15 10:04:41


% Java
\lstset{% 
  frame=single,
  backgroundcolor={\color[gray]{.9}},
  stringstyle={\ttfamily \color[rgb]{0,0,1}},
  commentstyle={\itshape \color[cmyk]{1,0,1,0}},
  identifierstyle={\ttfamily}, 
  keywordstyle={\ttfamily \color[cmyk]{0,1,0,0}},
  basicstyle={\ttfamily},
  breaklines=true,
  xleftmargin=0zw,
  xrightmargin=0zw,
  framerule=.2pt,
  columns=[l]{fullflexible},
  numbers=left,
  stepnumber=1,
  numberstyle={\scriptsize},
  numbersep=1em,
  language={Java},
  lineskip=-0.5zw,
  morecomment={[s][{\color[cmyk]{1,0,0,0}}]{/**}{*/}},
  keepspaces=true,         % 空白の連続をそのままで
  showstringspaces=false,  % 空白字をOFF
}
%\usepackage[dvipdfmx]{graphicx}
\usepackage{url}
\usepackage[dvipdfmx]{hyperref}
\usepackage{amsmath, amssymb}
\usepackage{itembkbx}
\usepackage{eclbkbox}	% required for `\breakbox' (yatex added)
\usepackage{enumerate}
\usepackage[default]{cantarell}
\usepackage[T1]{fontenc}
\fboxrule=0.5pt
\parindent=1em
\definecolor{mygrey}{rgb}{0.97, 0.97, 0.97}

\makeatletter
\def\verbatim@font{\normalfont
\let\do\do@noligs
\verbatim@nolig@list}
\makeatother

\begin{document}

%\anaumeと入力すると穴埋め解答欄が作れるようにしてる。\anaumesmallで小さめの穴埋めになる。
\newcounter{mycounter} % カウンターを作る
\setcounter{mycounter}{0} % カウンターを初期化
\newcommand{\anaume}[1][]{\refstepcounter{mycounter}{#1}{\boxed{\phantom{aa}\textnormal{\themycounter}\phantom{aa}}}} %穴埋め問題の空欄作ってる。
\newcommand{\anaumesmall}[1][]{\refstepcounter{mycounter}{#1}{\boxed{\tiny{\phantom{a}\themycounter \phantom{a}}}}}%小さい版作ってる。色々改造できる。

%% 修正時刻: Tue 2022/03/15 10:04:411


\section{データの挿入}

それでは、1件分のデータを入力する。

入力データ

\begin{tabular}{|c|c|c|c|c|} \hline
 id & name & age & birthyear & dept \\ \hline
 1 & 菅原文太 & 40 & 1933 & 総務部 \\ \hline
\end{tabular}
\vspace{3mm}

\begin{lstlisting}[language=SQL]
mysql> insert into emp 
    -> (id, name, age, birthyear, dept)
    -> values
    -> (1, '菅原文太', 40, 1933, '総務部');
Query OK, 1 row affected (0.001 sec)
\end{lstlisting}

続いて、2つめのデータを入力する。

入力データ

\begin{tabular}{|c|c|c|c|c|} \hline
 id & name & age & birthyear & dept \\ \hline
 2 & 千葉真一 & 34 & 1939 & 営業部 \\ \hline
\end{tabular}
\vspace{3mm}

全項目を入力する場合、項目指定を省略できる。

\begin{lstlisting}[language=SQL]
mysql> insert into emp 
    -> values
    -> (2, '千葉真一', 34, 1939, '営業部');
Query OK, 1 row affected (0.001 sec)
\end{lstlisting}

残りの2件を一度に入力する。

入力データ

\begin{tabular}{|c|c|c|c|c|} \hline
 id & name & age & birthyear & dept \\ \hline
 3 & 北大路欣也 & 30 & 1943 & 経理部 \\ \hline
 4 & 梶芽衣子 & 26 & 1947 & 営業部 \\ \hline
\end{tabular}
\vspace{3mm}

\begin{lstlisting}[language=SQL]
mysql> insert into emp 
    -> values
    -> (3, '北大路欣也', 30, 1943, '経理部'),
    -> (4, '梶芽衣子', 26, 1947, '営業部');
Query OK, 2 rows affected (0.003 sec)
Records: 2  Duplicates: 0  Warnings: 0
\end{lstlisting}


\section{データの表示}

今までに入力したデータの一覧を表示する。

\begin{lstlisting}[language=SQL]
 mysql> select * from emp;
\end{lstlisting}

あるいは、次のように出力する項目を指定できる。
\footnote{ここでは全項目を指定しているが、
必要な項目だけに絞ることもできる。}

\begin{lstlisting}[language=SQL]
 mysql> select
     ->   id,
     ->   name,
     ->   age,
     ->   birdhyear,
     ->   dept
     -> from emp;
\end{lstlisting}


\begin{lstlisting}[language={}, numbers=none]
+----+-----------------+------+-----------+-----------+
| id | name            | age  | birthyear | dept      |
+----+-----------------+------+-----------+-----------+
|  1 | 菅原文太        |   40 |      1933 | 総務部    |
|  2 | 千葉真一        |   34 |      1939 | 営業部    |
|  3 | 北大路欣也      |   30 |      1943 | 経理部    |
|  4 | 梶芽衣子        |   26 |      1947 | 営業部    |
+----+-----------------+------+-----------+-----------+
4 rows in set (0.000 sec)
\end{lstlisting}


\section{データの修正}

データの修正(更新)をしてみる。
ここでは、千葉真一の 部署を''開発部'' に変更してみる。

\begin{lstlisting}[language=SQL]
 mysql> update emp
     -> set
     ->   dept = '開発部'
     -> where
     ->   id = 2;
\end{lstlisting}

\begin{lstlisting}[numbers=none, language={}]
mysql> select * from emp;
+----+-----------------+------+-----------+-----------+
| id | name            | age  | birthyear | dept      |
+----+-----------------+------+-----------+-----------+
|  1 | 菅原文太        |   40 |      1933 | 総務部    |
|  2 | 千葉真一        |   34 |      1939 | 開発部    |
|  3 | 北大路欣也      |   30 |      1943 | 経理部    |
|  4 | 梶芽衣子        |   26 |      1947 | 営業部    |
+----+-----------------+------+-----------+-----------+
4 rows in set (0.000 sec)
\end{lstlisting}


\section{データの削除}

データを1件削除する。
ここでは、北大路欣也を削除してみる。

\begin{lstlisting}[language=SQL]
 mysql> delete from emp
     -> where
     ->   id = 3;
\end{lstlisting}

\begin{lstlisting}[numbers=none, language={}]
mysql> select * from emp;
+----+-----------------+------+-----------+-----------+
| id | name            | age  | birthyear | dept      |
+----+-----------------+------+-----------+-----------+
|  1 | 菅原文太        |   40 |      1933 | 総務部    |
|  2 | 千葉真一        |   34 |      1939 | 開発部    |
|  4 | 梶芽衣子        |   26 |      1947 | 営業部    |
+----+-----------------+------+-----------+-----------+
4 rows in set (0.000 sec)
\end{lstlisting}

\section{CRUD}

データの挿入(作成)(insert)、表示(読込み)(select)、
修正(更新)(update)、削除(delete) は
基本処理である。

Create Read Update Delete という。





\end{document}

%% 修正時刻: Sat May  2 15:10:04 2020


%% 修正時刻: Sat 2022/10/01 08:45:160
