\documentclass[uplatex, dvipdfmx]{jsarticle}

\include{begin}

\section{データの挿入}

それでは、1件分のデータを入力する。

入力データ

\begin{tabular}{|c|c|c|c|c|} \hline
 id & name & age & birthyear & dept \\ \hline
 1 & 菅原文太 & 40 & 1933 & 総務部 \\ \hline
\end{tabular}
\vspace{3mm}

\begin{lstlisting}[language=SQL]
mysql> insert into emp 
    -> (id, name, age, birthyear, dept)
    -> values
    -> (1, '菅原文太', 40, 1933, '総務部');
Query OK, 1 row affected (0.001 sec)
\end{lstlisting}

続いて、2つめのデータを入力する。

入力データ

\begin{tabular}{|c|c|c|c|c|} \hline
 id & name & age & birthyear & dept \\ \hline
 2 & 千葉真一 & 34 & 1939 & 営業部 \\ \hline
\end{tabular}
\vspace{3mm}

全項目を入力する場合、項目指定を省略できる。

\begin{lstlisting}[language=SQL]
mysql> insert into emp 
    -> values
    -> (2, '千葉真一', 34, 1939, '営業部');
Query OK, 1 row affected (0.001 sec)
\end{lstlisting}

残りの2件を一度に入力する。

入力データ

\begin{tabular}{|c|c|c|c|c|} \hline
 id & name & age & birthyear & dept \\ \hline
 3 & 北大路欣也 & 30 & 1943 & 経理部 \\ \hline
 4 & 梶芽衣子 & 26 & 1947 & 営業部 \\ \hline
\end{tabular}
\vspace{3mm}

\begin{lstlisting}[language=SQL]
mysql> insert into emp 
    -> values
    -> (3, '北大路欣也', 30, 1943, '経理部'),
    -> (4, '梶芽衣子', 26, 1947, '営業部');
Query OK, 2 rows affected (0.003 sec)
Records: 2  Duplicates: 0  Warnings: 0
\end{lstlisting}


\section{データの表示}

今までに入力したデータの一覧を表示する。

\begin{lstlisting}[language=SQL]
 mysql> select * from emp;
\end{lstlisting}

あるいは、次のように出力する項目を指定できる。
\footnote{ここでは全項目を指定しているが、
必要な項目だけに絞ることもできる。}

\begin{lstlisting}[language=SQL]
 mysql> select
     ->   id,
     ->   name,
     ->   age,
     ->   birdhyear,
     ->   dept
     -> from emp;
\end{lstlisting}


\begin{lstlisting}[language={}, numbers=none]
+----+-----------------+------+-----------+-----------+
| id | name            | age  | birthyear | dept      |
+----+-----------------+------+-----------+-----------+
|  1 | 菅原文太        |   40 |      1933 | 総務部    |
|  2 | 千葉真一        |   34 |      1939 | 営業部    |
|  3 | 北大路欣也      |   30 |      1943 | 経理部    |
|  4 | 梶芽衣子        |   26 |      1947 | 営業部    |
+----+-----------------+------+-----------+-----------+
4 rows in set (0.000 sec)
\end{lstlisting}


\section{データの修正}

データの修正(更新)をしてみる。
ここでは、千葉真一の 部署を''開発部'' に変更してみる。

\begin{lstlisting}[language=SQL]
 mysql> update emp
     -> set
     ->   dept = '開発部'
     -> where
     ->   id = 2;
\end{lstlisting}

\begin{lstlisting}[numbers=none, language={}]
mysql> select * from emp;
+----+-----------------+------+-----------+-----------+
| id | name            | age  | birthyear | dept      |
+----+-----------------+------+-----------+-----------+
|  1 | 菅原文太        |   40 |      1933 | 総務部    |
|  2 | 千葉真一        |   34 |      1939 | 開発部    |
|  3 | 北大路欣也      |   30 |      1943 | 経理部    |
|  4 | 梶芽衣子        |   26 |      1947 | 営業部    |
+----+-----------------+------+-----------+-----------+
4 rows in set (0.000 sec)
\end{lstlisting}


\section{データの削除}

データを1件削除する。
ここでは、北大路欣也を削除してみる。

\begin{lstlisting}[language=SQL]
 mysql> delete from emp
     -> where
     ->   id = 3;
\end{lstlisting}

\begin{lstlisting}[numbers=none, language={}]
mysql> select * from emp;
+----+-----------------+------+-----------+-----------+
| id | name            | age  | birthyear | dept      |
+----+-----------------+------+-----------+-----------+
|  1 | 菅原文太        |   40 |      1933 | 総務部    |
|  2 | 千葉真一        |   34 |      1939 | 開発部    |
|  4 | 梶芽衣子        |   26 |      1947 | 営業部    |
+----+-----------------+------+-----------+-----------+
4 rows in set (0.000 sec)
\end{lstlisting}

\section{CRUD}

データの挿入(作成)(insert)、表示(読込み)(select)、
修正(更新)(update)、削除(delete) は
基本処理である。

Create Read Update Delete という。





\include{end}

%% 修正時刻: Sat 2022/10/01 08:45:160
