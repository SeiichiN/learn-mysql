\documentclass[dvipdfmx]{jsarticle}

\include{begin}

\section{さまざまなコマンド}

\subsection{表定義関連}

\subsubsection{表定義の変更}

以下のような表定義があったとする。

\vspace{3mm}
\newsavebox{\sqlbox}
\setbox\sqlbox=\vbox{\hsize 6cm
\begin{verbatim}
CREATE TABLE student (
  id char(3),
  name varchar(20)
)
\end{verbatim}
}
\fbox{\usebox{\sqlbox}}
\vspace{3mm}

この name カラムに NOT NULL 制約を追加したいとする。
以下のようにできる。

\begin{lstlisting}[numbers=none, language=SQL]
mysql> ALTER TABLE student
    -> MODIFY
    ->   name varchar(20) NOT NULL;
\end{lstlisting}


\vspace{3mm}
\begin{tabular}{|l|} \hline
 ALTER TABLE テーブル名 \\
 MODIFY カラム名 カラム定義 \\ \hline
\end{tabular}
\vspace{3mm}

\subsubsection{表定義にカラムを追加}

また、age (年齢) という項目を追加したいとする。以下のようにできる。

\begin{lstlisting}[numbers=none, language=SQL]
mysql> ALTER TABLE student
    -> ADD
    ->   age int NOT NULL;
\end{lstlisting}


\vspace{3mm}
\begin{tabular}{|l|} \hline
 ALTER TABLE テーブル名 \\
 ADD カラム名 カラム定義 \\ \hline
\end{tabular}
\vspace{3mm}

\subsubsection{表定義にプライマリーキーを追加}

あるいは、id に PRIMARY KEY を設定したいとする。以下のようにできる。

\begin{lstlisting}[numbers=none, language=SQL]
mysql> ALTER TABLE student
    -> ADD
    ->   PRIMARY KEY (id);
\end{lstlisting}

\vspace{3mm}
\begin{tabular}{|l|} \hline
 ALTER TABLE テーブル名 \\
 ADD PRIMARY KEY (カラム名) \\ \hline
\end{tabular}
\vspace{3mm}

\subsubsection{表定義に外部キー制約を追加}

外部キー制約を あとから設定するには、以下のようにする。

\begin{tcolorbox}
\verb!ALTER TABLE emp               -- empテーブル再定義! \\
\verb!   ADD                         -- 追加! \\
\verb!     CONSTRAINT fk_dept        -- 制約名 fk_dept! \\
\verb!     FOREIGN KEY (dept_id)     -- 外部キー dept_id! \\
\verb!     REFERENCES dept(id);      -- 参照  dept表の(id)!
\end{tcolorbox}

上の例は、emp表の中にある dept\_id という項目が、dept表の id という項目を参照しているという
設定である。

\subsection{表のデータの更新}

たとえば、emp表の id=1 の人の氏名を変更するには、以下のようにする。

\begin{lstlisting}[numbers=none, language=SQL]
mysql> UPDATE emp
    -> SET
    ->   name = '菅原文夫'
    -> WHERE
    ->   id = 1; 
\end{lstlisting}


\begin{tabular}{|l|} \hline
 \verb!UPDATE テーブル名 SET カラム名 = 訂正データ WHERE 条件! \\ \hline
\end{tabular}

\subsection{表のデータの削除}

\subsubsection{1件のデータを削除する}

id が 1 のデータを削除する。

\begin{lstlisting}[numbers=none, language=SQL]
mysql> DELETE FROM emp
    -> WHERE id = 1;
\end{lstlisting}

\begin{tabular}{|l|} \hline
 \verb!DELETE FROM テーブル名 WHERE 条件! \\ \hline
\end{tabular}

id が 1 のデータを削除すると、1 が欠番になる。

\begin{spacing}{0.8}        
\begin{verbatim}
+----+------------+-----+----------+---------+
| id | name       | age | birthday | dept_id |
+----+------------+-----+----------+---------+
|  2 | 千葉真一   |  34 |     1939 | 002     |
|  4 | 梶芽衣子   |  26 |     1947 | 002     |
|  3 | 北大路欣也 |  30 |     1943 | 003     |
+----+------------+-----+----------+---------+
\end{verbatim}
\end{spacing}

この場合、id の数字を振り直すことはしない。
なぜなら、id は順番を表す数字ではなく、そのデータを特定するための数字だからである。
それが プライマリー・キーの役割である。
順番になっているのは、データを作成するときに、AUTO\_INCREMENT を使ったからである。


\subsubsection{全データを削除する}

条件をつけなければ、全データが削除される。


\begin{lstlisting}[numbers=none, language=SQL]
mysql> DELETE FROM emp;
\end{lstlisting}

\begin{tabular}{|l|} \hline
 \verb!DELETE FROM テーブル名! \\ \hline
\end{tabular}


\subsection{表のデータの表示・検索}

SELECT で表のデータを表示するとき、WHERE で条件を指定する際に、''LIKE'' で
条件を指定できる。

\begin{lstlisting}[numbers=none, language=SQL]
 mysql> SELECT *
     ->   FROM emp
     -> WHERE
     ->   name LIKE '%葉%';
\end{lstlisting}

\begin{verbatim}
+----+--------------+-----+----------+---------+
| id | name         | age | birthday | dept_id |
+----+--------------+-----+----------+---------+
|  2 | 千葉真一      |  34 |     1939 | 002     |
+----+--------------+-----+----------+---------+
\end{verbatim}

\begin{tabular}{|l|} \hline
 \verb!SELECT [カラム名, ...] FROM テーブル名 WHERE カラム名 LIKE '%語句%';! \\ \hline
\end{tabular}
\vspace{3mm}

\noindent
{\em \%}(パーセント) は 0文字以上の文字列に合致する。\\
{\em \_}(アンダースコア) は 1文字以上の文字列に合致する。







\include{end}

%% 修正時刻: Sat 2022/10/01 18:25:282
