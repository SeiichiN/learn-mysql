\documentclass[dvipdfmx]{jsarticle}

\include{begin}

\section{MySQLの運用と管理}


\subsection{ユーザー一覧}

MySQLで今までに作成したユーザーの一覧を見たい場合、以下のコマンドでできる。
ただし、ルート・ユーザー(root)でないと権限がないと怒られる。

\begin{tcolorbox}
 \verb!mysql> SELECT user, host, db FROM mysql.db;!
\end{tcolorbox}

パスワードは暗号化されているので、見ても仕方がないが、以下のコマンドでできる。

\begin{tcolorbox}
 \verb!mysql> SELECT user, password FROM mysql.user;!
\end{tcolorbox}

\subsection{パスワードの再設定}

パスワードを忘れたなどで、パスワードを再設定したい場合は、以下のコマンドでできる。
(rootユーザーで実行)

\begin{tcolorbox}
 mysql$>$ SET \ PASSWORD \ FOR \ 'ユーザー名'@'localhost' \ = \ password('パスワード');
\end{tcolorbox}


\subsection{データベースのバックアップと復元}

\subsubsection{バックアップ}

データの入力が終わったら、データベースをバックアップする。

ここでは、\textsf{mysqldump} を使っておこなう。
これは mysql と同じく コマンドアプリである。

まず、MySQLをログアウトする。

\begin{tcolorbox}
 mysql$>$ exit (もしくは quit) \\
 C:\yen Users\yen XXXXXX\yen Documents\yen mysql$>$
\end{tcolorbox}

\rightline{※ XXXXXX は、各自のユーザー名}

コマンドプロンプトに戻る。

ここで以下のコマンドを実行する。

\begin{tcolorbox}
 $>$ mysqldump \ -u \ sampleuser \ -p \ -B \ sample \ $>$ \ sample\_db.dump
\end{tcolorbox}

\begin{tabular}{|l|} \hline
 mysqldump \ -u \ ユーザー名 \ -p \ -B \ データベース名 \ $>$
 保存ファイル名 \\ \hline
\end{tabular}

ここでは保存ファイル名を sample\_db.dump としたが、好みのファイル名を指定すればよい。

\textsf{dir} というコマンドを実行すると、ファイル一覧が見れる。

ファイル群の中に \textsf{sample\_db.dump} があるはず。

これは テキストファイルなので、TeraPad などのエディタで内容を見ることができる。
\footnote{また、{\em -B} オプションをつけずにバックアップを取ることもできる。 \\
\fbox{ $>$ mysqldump -u sampleuser -p sample $>$ sample\_db.dump } \\
この場合は、バックアップファイルの記述の中に、CREATE DATABASE 文がない。
}

\subsubsection{データのリストア(復元)}

以下のコマンドを実行する。
\footnote{-B オプションをつけずにバックアップファイルを作成した場合は、まず sample
データベースを作成してから \\
\fbox{ $>$ mysql -u sampleuser -p $<$ sample\_db.dump } \\
とするか、あるいは、sampleuserでログインしてから \\
\fbox{mysql$>$ source sample\_db.dump} とすればよい。
}
\begin{tcolorbox}
 $>$ mysql -u sampleuser -p $<$ sample\_db.dump \\
 password: ******
\end{tcolorbox}

\fbox{mysql \ -u \ ユーザー名 \ -p \ $<$ \ 保存したファイル名}
\vspace{3mm}

これでリストアができている。



\include{end}

%% 修正時刻: Mon Feb 21 18:01:02 2022
