\documentclass[dvipdfmx]{jsarticle}

\include{begin}

\section{2つのテーブルを結合する}

\subsection{内部結合(JOIN句)}

empテーブルの dept\_id は、deptテーブルの id である。

だから、dept\_idをキーにして、二つのテーブルを結合できる。

結合するには \textsf{JOIN}句を使う。

実行例

\begin{tcolorbox}
 mysql$>$ SELECT \ * \\
 \hspace{3mm} -$>$ FROM \ emp \\
 \hspace{3mm} -$>$ \underline{\textsf{INNER JOIN}} \ dept \\
 \hspace{3mm} -$>$ \underline{\textsf{ON}} \ emp.dept\_id \ = \ dept.id;
\end{tcolorbox}

\begin{lstlisting}[numbers=none, language=SQL]
 SELECT * 
 FROM メインの表
 [INNER] JOIN サブの表 
 ON メインの表.キーカラム名 = サブの表.キーカラム名
\end{lstlisting}


\vspace{3mm}
\begin{itemize}
 \item *(アスタリスク) --- メインの表、サブの表のすべての項目を表示する。
 \item emp.dept\_id --- emp表の dept\_id カラム名。ピリオドで区切って指定。
ピリオドの後に空白を入れてはいけない。
 \item dept.id --- dept表の id カラム名。
\end{itemize}


\vspace{3mm}
これで二つのテーブルが結合される。

% \begin{spacing}{0.8}        
% \begin{verbatim}
\begin{lstlisting}[numbers=none]
+----+------------+-----+----------+---------+-----+--------+
| id | name       | age | birthday | dept_id | id  | name   |
+----+------------+-----+----------+---------+-----+--------+
|  1 | 菅原文太    |  40 |     1933 | 001     | 001 | 総務部  |
|  2 | 千葉真一    |  34 |     1939 | 002     | 002 | 営業部  |
|  4 | 梶芽衣子    |  26 |     1947 | 002     | 002 | 営業部  |
|  3 | 北大路欣也  |  30 |     1943 | 003     | 003 | 経理部  |
+----+------------+-----+----------+---------+-----+--------+
\end{lstlisting}
% \end{verbatim}
% \end{spacing}

\textsf{JOIN} は \textsf{INNER JOIN} と記述できる。
``内部結合''と呼ばれている。

\textsf{ON emp.dept\_id = dept.id} は 
emp の dept\_id と dept の id が等しければ、そのレコードを抜き出す。

\rightline{※ emp.dept\_id は empテーブルの dept\_id という意味になる。}

\subsection{表示項目を絞る}

現在は項目を全て表示しているが、これを変更する。

emp表の id, name, age と dept表の name だけを表示させる。


\begin{lstlisting}[numbers=none, language=SQL]
 mysql> SELECT emp.id, emp.name, age, dept.name
     -> FROM emp
     -> INNER JOIN dept
     -> ON emp.dept_id = dept.id;
\end{lstlisting}


% \begin{spacing}{0.8}        
% \begin{verbatim}
\begin{lstlisting}[numbers=none]
+----+------------+-----+--------+
| id | name       | age | name   |
+----+------------+-----+--------+
|  1 | 菅原文太   |  40 | 総務部 |
|  2 | 千葉真一   |  34 | 営業部 |
|  4 | 梶芽衣子   |  26 | 営業部 |
|  3 | 北大路欣也 |  30 | 経理部 |
+----+------------+-----+--------+
\end{lstlisting}
% \end{verbatim}
% \end{spacing}        

このように必要な項目のみ表示させることができる。
ただ、name という項目が二つあったり、英語であったりするので、
これを適切な日本語に変えることにする。

それには、\textsf{AS句} というのが使える。

たとえば、``emp.name AS 名前'' とすると、``emp.neme'' は ``名前'' と表示される。

\begin{lstlisting}[numbers=none, language=SQL]
 mysql> SELECT
     ->   emp.id AS ID,
     ->   emp.name AS 名前,
     ->   age AS 年齢,
     ->   dept.name AS 部署名
     -> FROM emp
     -> INNER JOIN dept
     ->   ON emp.dept_id = dept.id;
\end{lstlisting}


% \begin{spacing}{0.8}        
% \begin{verbatim}
\begin{lstlisting}[numbers=none]
+----+------------+------+--------+
| ID | 名前       | 年齢 | 部署名 |
+----+------------+------+--------+
|  1 | 菅原文太   |   40 | 総務部 |
|  2 | 千葉真一   |   34 | 営業部 |
|  4 | 梶芽衣子   |   26 | 営業部 |
|  3 | 北大路欣也 |   30 | 経理部 |
+----+------------+------+--------+
\end{lstlisting}
% \end{verbatim}
% \end{spacing}

さらによく見てみると、この表は部署名の順に並んでいる。
これを ID順に並びかえる。

そのためには \textsf{ORDER句} というのが使える。

たとえば、今回の場合だと、\fbox{\textsf{ORDER BY emp.id [\! ASC\! ]}} とすることで、ID順になる。

\textsf{ASC} というのは''昇順''という意味で、省略すると ASC と指定したことになる。

また、\textsf{DESC} と指定すると ''降順'' で並びかえできる。


\begin{lstlisting}[numbers=none, language=SQL]
 mysql> SELECT
     ->   emp.id AS ID,
     ->   emp.name AS 名前,
     ->   age AS 年齢,
     ->   dept.name AS 部署名
     -> FROM emp
     -> INNER JOIN dept
     ->   ON emp.dept_id = dept.id
     -> ORDER BY ID;
\end{lstlisting}



\begin{lstlisting}[numbers=none]
+----+------------+------+--------+
| ID | 名前       | 年齢 | 部署名 |
+----+------------+------+--------+
|  1 | 菅原文太   |   40 | 総務部 |
|  2 | 千葉真一   |   34 | 営業部 |
|  3 | 北大路欣也 |   30 | 経理部 |
|  4 | 梶芽衣子   |   26 | 営業部 |
+----+------------+------+--------+
\end{lstlisting}


\textsf{ORDER BY emp.id} とするところを \textsf{ORDER BY ID} としている。

これは、1行目で \textsf{emp.id AS ID} としているので、ID を使うことができるのである。


\subsection{テーブルの指定を簡略化する}

emp とか dept とかのテーブルの指定も別名を使うことで簡略化できる。

\begin{lstlisting}[numbers=none, language=SQL]
 mysql> SELECT
     ->   e.id AS ID,
     ->   e.name AS 名前,
     ->   age AS 年齢,
     ->   d.name AS 部署名
     -> FROM emp AS e
     -> INNER JOIN dept AS d
     ->   ON e.dept_id = d.id
     -> ORDER BY ID;
\end{lstlisting}


さらに、FROM や JOIN の後では、すぐ後ろに 別名(ここでは e や d のこと)がくる場合、{\em AS} は
省略できる。

\begin{lstlisting}[numbers=none, language=SQL]
 mysql> SELECT
     ->   e.id AS ID,
     ->   e.name AS 名前,
     ->   age AS 年齢,
     ->   d.name AS 部署名
     -> FROM emp e
     -> INNER JOIN dept d
     ->   ON e.dept_id = d.id
     -> ORDER BY ID;
\end{lstlisting}







\newpage
\subsection{外部結合(LEFT OUTER JOIN / RIGHT OUTER JOIN)}

\subsubsection{左外部結合 LEFT OUTER JOIN}

この emp表に次のデータを追加する。

\begin{tcolorbox}
 \begin{tabular}{lcl}
  ID & : & 5 \\
  名前 & : & 成田三樹夫 \\
  年齢 & : &  38 \\
  誕生年 & : & 1935 \\
  部署ID & : & (なし) \\
 \end{tabular}
\end{tcolorbox}

\begin{lstlisting}[numbers=none, language=SQL]
 mysql> INSERT INTO emp
     ->   (name, age, birthday)
     -> VALUES
     ->   ('成田三樹夫', 38, 1935);
\end{lstlisting}


\begin{spacing}{0.8}        
\begin{verbatim}
mysql> SELECT * FROM emp;
+----+------------+-----+----------+---------+
| id | name       | age | birthday | dept_id |
+----+------------+-----+----------+---------+
|  1 | 菅原文太   |  40 |     1933 | 001     |
|  2 | 千葉真一   |  34 |     1939 | 002     |
|  3 | 北大路欣也 |  30 |     1943 | 003     |
|  4 | 梶芽衣子   |  26 |     1947 | 002     |
|  5 | 成田三樹夫 |  38 |     1935 | NULL    |
+----+------------+-----+----------+---------+    
\end{verbatim}
\end{spacing}


このデータには dept\_id、つまり部署ID がない。たとえば社長とかの場合である。

この状態で 内部結合 をすると、どうなるか?

\begin{verbatim}
mysql> SELECT
    ->   e.id AS ID,
    ->   e.name AS 名前,
    ->   e.age AS 年齢,
    ->   d.name AS 部署名
    -> FROM emp e
    -> JOIN dept d
    ->   ON e.dept_id = d.id
    -> ORDER BY ID;
\end{verbatim}

\begin{spacing}{0.8}        
\begin{verbatim}
+----+------------+------+--------+
| ID | 名前       | 年齢 | 部署名 |
+----+------------+------+--------+
|  1 | 菅原文太   |   40 | 総務部 |
|  2 | 千葉真一   |   34 | 営業部 |
|  3 | 北大路欣也 |   30 | 経理部 |
|  4 | 梶芽衣子   |   26 | 営業部 |
+----+------------+------+--------+    
\end{verbatim}
\end{spacing}

結合表には出てこない。

これを図であらわすと、このようになる。

\vspace{3mm}
\includegraphics[width=13cm]{../06-mysql/ketsugo.png}
\vspace{3mm}

成田三樹夫は部署IDがないので結合の対象ではない。

こんなときは \emph{左外部結合(LEFT [OUTER] JOIN)} を使う。

\begin{lstlisting}[numbers=none, language=SQL]
 mysql> SELECT
     ->   e.id AS ID,
     ->   e.name AS 名前,
     ->   e.age AS 年齢,
     ->   d.name as 部署名
     -> FROM emp e
     -> LEFT OUTER JOIN dept d
     ->   ON e.dept_id = d.id
     -> ORDER BY ID;
\end{lstlisting}

\rightline{※ LEFT JOIN と記述することもできる}

\begin{spacing}{0.8}        
\begin{verbatim}
+----+------------+------+--------+
| ID | 名前       | 年齢 | 部署名 |
+----+------------+------+--------+
|  1 | 菅原文太   |   40 | 総務部 |
|  2 | 千葉真一   |   34 | 営業部 |
|  3 | 北大路欣也 |   30 | 経理部 |
|  4 | 梶芽衣子   |   26 | 営業部 |
|  6 | 成田三樹夫 |   38 | NULL   |
+----+------------+------+--------+
\end{verbatim}
\end{spacing}

\subsubsection{右外部結合 RIGHT [OUTER] JOIN}

また、dept表をみてみると、\textsf{id : '004'} が  \textsf{開発部} であるが、
emp表には dept\_id が '004' である人はいない。

この状態で結合表をつくり、開発部という項目も表示させるには、次のようにする。

\begin{lstlisting}[numbers=none, language=SQL]
 mysql> SELECT
     ->   e.id AS ID,
     ->   e.name AS 名前,
     ->   e.age AS 年齢,
     ->   d.name AS 部署名
     -> FROM emp e
     -> RIGHT OUTER JOIN dept d
     ->   ON e.dept_id = d.id
     -> ORDER BY ID;
\end{lstlisting}

\begin{spacing}{0.8}        
\begin{verbatim}
+------+------------+------+--------+
| ID   | 名前       | 年齢 | 部署名 |
+------+------------+------+--------+
| NULL | NULL       | NULL | 開発部 |
|    1 | 菅原文太   |   40 | 総務部 |
|    2 | 千葉真一   |   34 | 営業部 |
|    3 | 北大路欣也 |   30 | 経理部 |
|    4 | 梶芽衣子   |   26 | 営業部 |
+------+------------+------+--------+
\end{verbatim}
\end{spacing}

\newpage
\section{練習問題}

\subsection{結合の練習}

\fbox{4.2 テーブル作成問題(2)} で作った4つの表を結合しましょう。

gender表、state表、course表、person表 を内部結合して、以下のように表示させてください。

\vspace{3mm}
\begin{tabular}{|c|l|c|l|l|l|} \hline
 ID & 名前            & 性別   & 誕生日     & 出身         & コース            \\ \hline\hline 
  1 & 染谷将太        & 男     & 1992-09-03 & 東京都       & JavaScriptコース  \\ \hline
  3 & 渡辺哲          & 男     & 1950-03-11 & 愛知県       & Javaコース        \\ \hline
  4 & 窪塚洋介        & 男     & 1979-05-07 & 神奈川県     & HTML/CSSコース    \\ \hline
  2 & 二階堂ふみ      & 女     & 1994-09-21 & 沖縄県       & PHPコース         \\ \hline
  5 & 吉高由里子      & 女     & 1988-07-22 & 東京都       & Javaコース        \\ \hline
\end{tabular}
\vspace{3mm}

\begin{lstlisting}[numbers=none, language=SQL]
mysql> SELECT
    ->   timestampdiff(YEAR, birthday, curdate()) AS 年齢
    -> FROM person;
\end{lstlisting}

とすると、以下のように出力できる。

\vspace{3mm}
\begin{tabular}{|c|} \hline
 年齢 \\ \hline\hline
  29  \\ \hline
  27  \\ \hline
  71  \\ \hline
  42  \\ \hline
  33  \\ \hline
\end{tabular}
\vspace{3mm}

このことから、以下のように表示させることができるはず。

\vspace{3mm}
\begin{tabular}{|c|l|c|l|l|l|} \hline
 ID & 名前            & 性別   & 年齢 & 出身         & コース            \\ \hline\hline 
  1 & 染谷将太        & 男     & 29  & 東京都       & JavaScriptコース  \\ \hline
  3 & 渡辺哲          & 男     & 71  & 愛知県       & Javaコース        \\ \hline
  4 & 窪塚洋介        & 男     & 42  & 神奈川県     & HTML/CSSコース    \\ \hline
  2 & 二階堂ふみ      & 女     & 27  & 沖縄県       & PHPコース         \\ \hline
  5 & 吉高由里子      & 女     & 33  & 東京都       & Javaコース        \\ \hline
\end{tabular}
\vspace{3mm}




\include{end}

%% 修正時刻: Sat 2022/10/01 18:25:282
