\documentclass[dvipdfmx]{jsarticle}

\include{begin}

\section{ちょっと複雑なデータベースを考える}

\subsection{データベースの例}

今度は以下のようなデータについて考えてみる。

\vspace{3mm}
 \begin{tabular}{|l|l|} \hline
  氏名 & 菅原文太 \\
  年齢 & 40歳 \\
  誕生年 & 1933年生まれ \\
  部署 & 総務部 \\ 
  趣味 & 釣り、油絵  \\ \hline
 \end{tabular}

\vspace{3mm}
 \begin{tabular}{|l|l|} \hline
  氏名 & 千葉真一 \\
  年齢 & 34歳 \\
  誕生年 &  1939年生まれ \\
  部署 & 営業部 \\
  趣味 & 空手、熱帯魚飼育、サッカー観戦  \\ \hline
 \end{tabular}

\vspace{3mm}
 \begin{tabular}{|l|l|} \hline
  氏名 & 北大路欣也 \\
  年齢 & 30歳 \\
  誕生年 &  1943年生まれ \\
  部署 & 経理部 \\
  趣味 & 茶道  \\ \hline
 \end{tabular}

\vspace{3mm}
 \begin{tabular}{|l|l|} \hline
  氏名 & 梶芽衣子 \\
  年齢 & 26歳 \\
  誕生年 &  1947年生まれ \\
  部署 & 営業部 \\
  趣味 & 登山、ヨガ  \\ \hline
 \end{tabular}
\vspace{3mm}

まず、このような表がイメージされる。

\begin{table}[h]
 \caption{emp}
 \begin{center}
  \begin{tabular}[h]{|c|l|c|c|c|l|}
   \hline
   ID & 名前       & 年齢 & 誕生年 & 部署 & 趣味 \\ \hline\hline
   1  & 菅原文太   & 40   & 1933   & 総務 & 釣り, 油絵  \\ \hline
   2  & 千葉真一   & 34   & 1939   & 営業 & 空手, 熱帯魚飼育, サッカー観戦 \\ \hline
   3  & 北大路欣也 & 30   & 1943   & 経理 & 茶道 \\ \hline
   4  & 梶芽衣子   & 26   & 1947   & 営業 & 登山, ヨガ  \\ \hline
  \end{tabular}
 \end{center}
\end{table}

しかしながら、この表の場合、趣味のフィールドには、複数のデータが含まれている。
これを解消したのが、次の表である。

\subsection{第1正規形}

\begin{table}[h]
 \caption{emp}
 \begin{center}
  \begin{tabular}[h]{|c|l|c|c|c|l|l|l|}
   \hline
   ID & 名前       & 年齢 & 誕生年 & 部署 & 趣味1 & 趣味2 & 趣味3 \\ \hline\hline
   1  & 菅原文太   & 40   & 1933   & 総務 & 釣り & 油絵 & \\ \hline
   2  & 千葉真一   & 34   & 1939   & 営業 & 空手 & 熱帯魚飼育 & サッカー観戦 \\ \hline
   3  & 北大路欣也 & 30   & 1943   & 経理 & 茶道 & & \\ \hline
   4  & 梶芽衣子   & 26   & 1947   & 営業 & 登山 & ヨガ & \\ \hline
  \end{tabular}
 \end{center}
\end{table}


趣味のフィールドが1つのもの、3つのものとバラバラなので、フィールドを1つにする。

\begin{table}[h]
 \caption{emp}
 \begin{center}
  \begin{tabular}[h]{|c|l|c|c|c|l|}
   \hline
   ID & 名前       & 年齢 & 誕生年 & 部署 & 趣味 \\ \hline\hline
   1  & 菅原文太   & 40   & 1933   & 総務 & 釣り \\ \hline
   1  & 菅原文太   & 40   & 1933   & 総務 & 油絵 \\ \hline
   2  & 千葉真一   & 34   & 1939   & 営業 & 空手  \\ \hline
   2  & 千葉真一   & 34   & 1939   & 営業 & 熱帯魚飼育 \\ \hline
   2  & 千葉真一   & 34   & 1939   & 営業 & サッカー観戦 \\ \hline
   3  & 北大路欣也 & 30   & 1943   & 経理 & 茶道 \\ \hline
   4  & 梶芽衣子   & 26   & 1947   & 営業 & 登山 \\ \hline
   4  & 梶芽衣子   & 26   & 1947   & 営業 & ヨガ \\ \hline
  \end{tabular}
 \end{center}
\end{table}

ここまでが「第1正規化」で、この表を「第1正規形」と呼ぶ。

\subsection{第2正規形}


上の表は縦方向にデータが繰り返されている。
これを表を分けることによって、解消する。

\begin{multicols}{2}

 emp表 \\
  \begin{tabular}[h]{|c|l|c|c|c|}
   \hline
   ID & 名前       & 年齢 & 誕生年 & 部署  \\ \hline\hline
   1  & 菅原文太   & 40   & 1933   & 総務  \\ \hline
   2  & 千葉真一   & 34   & 1939   & 営業  \\ \hline
   3  & 北大路欣也 & 30   & 1943   & 経理  \\ \hline
   4  & 梶芽衣子   & 26   & 1947   & 営業  \\ \hline
  \end{tabular}

 \columnbreak

 hobby表 \\
  \begin{tabular}[h]{|c|l|}
   \hline
   HID    & 趣味 \\ \hline\hline
   H01   & 釣り \\ \hline
   H02   & 油絵 \\ \hline
   H03   & 空手  \\ \hline
   H04   & 熱帯魚飼育 \\ \hline
   H05   & サッカー観戦 \\ \hline
   H06   & 茶道 \\ \hline
   H07   & 登山 \\ \hline
   H08   & ヨガ \\ \hline
  \end{tabular}
\end{multicols}


emp\_hobby表\\
\begin{tabular}{|c|c|c|} \hline
 \multicolumn{2}{|c|}{主キー} & \multirow{2}{*}{趣味} \\ \cline{1-2}
t ID & HID &      \\ \hline\hline
 1  & H01 & 釣り \\ \hline
 1  & H02 & 油絵 \\ \hline
 2  & H03 & 空手 \\ \hline
 2  & H04 & 熱帯魚飼育 \\ \hline
 2  & H05 & サッカー観戦 \\ \hline
 3  & H06 & 茶道 \\ \hline
 4  & H07 & 登山 \\ \hline
 4  & H08 & ヨガ \\ \hline
\end{tabular}


\subsection{第3正規形}

emp表には、表には現れていないが、id に従属せず、別のキーに従属すると
考えられるフィールドが存在する。それは「所属」である。
これを別の表(dept表)とする。

\begin{multicols}{2}
 emp表 \\
  \begin{tabular}[h]{|c|l|c|c|}
   \hline
   ID & 名前       & 年齢 & 誕生年 \\ \hline\hline
   1  & 菅原文太   & 40   & 1933   \\ \hline
   2  & 千葉真一   & 34   & 1939   \\ \hline
   3  & 北大路欣也 & 30   & 1943   \\ \hline
   4  & 梶芽衣子   & 26   & 1947   \\ \hline
  \end{tabular}

 \columnbreak

 dept表 \\
 \begin{tabular}{|c|c|} \hline
  DID & 部署名 \\ \hline\hline
  D01 & 総務部 \\ \hline
  D02 & 営業部 \\ \hline
  D03 & 経理部 \\ \hline
  D04 & 開発部 \\ \hline
 \end{tabular}
\end{multicols}
  
\begin{multicols}{2}
 hobby表 \\
  \begin{tabular}[h]{|c|l|}
   \hline
   HID   & 趣味 \\ \hline\hline
   H01   & 釣り \\ \hline
   H02   & 油絵 \\ \hline
   H03   & 空手  \\ \hline
   H04   & 熱帯魚飼育 \\ \hline
   H05   & サッカー観戦 \\ \hline
   H06   & 茶道 \\ \hline
   H07   & 登山 \\ \hline
   H08   & ヨガ \\ \hline
  \end{tabular}

 \columnbreak
 
 emp\_hobby表\\
 \begin{tabular}{|c|c|c|} \hline
  \multicolumn{2}{|c|}{主キー} & \multirow{2}{*}{趣味} \\ \cline{1-2}
  ID & HID &      \\ \hline\hline
  1  & H01 & 釣り \\ \hline
  1  & H02 & 油絵 \\ \hline
  2  & H03 & 空手 \\ \hline
  2  & H04 & 熱帯魚飼育 \\ \hline
  2  & H05 & サッカー観戦 \\ \hline
  3  & H06 & 茶道 \\ \hline
  4  & H07 & 登山 \\ \hline
  4  & H08 & ヨガ \\ \hline
 \end{tabular}
\end{multicols}


\section{ER図}

第3正規形ができたところで、ER図を作ってみる。

emp表\\
\begin{tabular}{|c|} \hline
 ID \\ \hline
 name \\ \hline
 age \\ \hline
 birthyear \\ \hline
\end{tabular}
 

\include{end}

%% 修正時刻: Fri Feb 18 15:07:09 2022
