\documentclass[dvipdfmx]{jsarticle}

\include{begin}

\section{ちょっと複雑なデータベースを考える}

\subsection{データベースの例}

今度は以下のようなデータについて考えてみる。

\vspace{3mm}
 \begin{tabular}{|l|l|} \hline
  氏名 & 菅原文太 \\
  性別 & 男性 \\
  年齢 & 40歳 \\
  誕生年 & 1933年生まれ \\
  部署 & 総務部 \\ 
  趣味 & 釣り、油絵、空手  \\ \hline
 \end{tabular}

\vspace{3mm}
 \begin{tabular}{|l|l|} \hline
  氏名 & 千葉真一 \\
  性別 & 男性 \\
  年齢 & 34歳 \\
  誕生年 &  1939年生まれ \\
  部署 & 営業部 \\
  趣味 & 空手、熱帯魚飼育、サッカー観戦、釣り  \\ \hline
 \end{tabular}

\vspace{3mm}
 \begin{tabular}{|l|l|} \hline
  氏名 & 北大路欣也 \\
  性別 & 男性 \\
  年齢 & 30歳 \\
  誕生年 &  1943年生まれ \\
  部署 & 経理部 \\
  趣味 & 茶道、空手  \\ \hline
 \end{tabular}

\vspace{3mm}
 \begin{tabular}{|l|l|} \hline
  氏名 & 梶芽衣子 \\
  性別 & 女性 \\
  年齢 & 26歳 \\
  誕生年 &  1947年生まれ \\
  部署 & 営業部 \\
  趣味 & 登山、ヨガ、サッカー観戦  \\ \hline
 \end{tabular}
\vspace{3mm}

まず、このような表がイメージされる。

 emp表 \\
  \begin{tabular}[h]{|c|l|c|c|c|c|l|}
   \hline
   ID & 名前       & 性別 & 年齢 & 誕生年 & 部署 & 趣味 \\ \hline\hline
   1  & 菅原文太   & 男性 & 40   & 1933   & 総務 & 釣り, 油絵, 空手  \\ \hline
   2  & 千葉真一   & 男性 & 34   & 1939   & 営業 & 空手, 熱帯魚飼育, サッカー観戦, 釣り \\ \hline
   3  & 北大路欣也 & 男性 & 30   & 1943   & 経理 & 茶道, 空手 \\ \hline
   4  & 梶芽衣子   & 女性 & 26   & 1947   & 営業 & 登山, ヨガ, サッカー観戦  \\ \hline
  \end{tabular}

しかしながら、この表の場合、趣味のフィールドには、複数のデータが含まれている。
これを解消したのが、次の表である。


\subsection{第1正規形}

\vspace{3mm}
emp表 \\
  \begin{tabular}[h]{|c|l|c|c|c|c|l|l|l|l|}
   \hline
   \underline{ID}
      & 名前       & 性別 & 年齢 & 誕生年 & 部署 & 趣味1 & 趣味2     & 趣味3       & 趣味4 \\ \hline\hline
   1  & 菅原文太   & 男性 & 40   & 1933   & 総務 & 釣り & 油絵       & 空手        &      \\ \hline
   2  & 千葉真一   & 男性 & 34   & 1939   & 営業 & 空手 & 熱帯魚飼育 & サッカー観戦 & 釣り \\ \hline
   3  & 北大路欣也 & 男性 & 30   & 1943   & 経理 & 茶道 & 空手       &             & \\ \hline
   4  & 梶芽衣子   & 女性 & 26   & 1947   & 営業 & 登山 & ヨガ       & サッカー観戦 & \\ \hline
  \end{tabular}
\vspace{3mm}


趣味のフィールドが1つのもの、3つのものとバラバラなので、フィールドを1つにする。

\vspace{3mm}
  emp表 \\
  \begin{tabular}[h]{|c|l|c|c|c|c|l|}
   \hline
   \underline{ID}
      & 名前        & 性別 & 年齢 & 誕生年 & 部署 & 趣味 \\ \hline\hline
   1  & 菅原文太    & 男性 & 40   & 1933   & 総務 & 釣り \\ \hline
   1  & 菅原文太    & 男性 & 40   & 1933   & 総務 & 油絵 \\ \hline
   1  & 菅原文太    & 男性 & 40   & 1933   & 総務 & 空手 \\ \hline
   2  & 千葉真一    & 男性 & 34   & 1939   & 営業 & 空手  \\ \hline
   2  & 千葉真一    & 男性 & 34   & 1939   & 営業 & 熱帯魚飼育 \\ \hline
   2  & 千葉真一    & 男性 & 34   & 1939   & 営業 & サッカー観戦 \\ \hline
   2  & 千葉真一    & 男性 & 34   & 1939   & 営業 & 釣り \\ \hline
   3  & 北大路欣也  & 男性 & 30   & 1943   & 経理 & 茶道 \\ \hline
   3  & 北大路欣也  & 男性 & 30   & 1943   & 経理 & 空手 \\ \hline
   4  & 梶芽衣子    & 女性 & 26   & 1947   & 営業 & 登山 \\ \hline
   4  & 梶芽衣子    & 女性 & 26   & 1947   & 営業 & ヨガ \\ \hline
   4  & 梶芽衣子    & 女性 & 26   & 1947   & 営業 & サッカー観戦 \\ \hline
  \end{tabular}
\vspace{3mm}

ここまでが「第1正規化」で、この表を「第1正規形」と呼ぶ。

\vspace{5mm}
\fbox{\large \textgt{第1正規形} :\quad 1つのセルには1つの値しか含まない}

\newpage
\subsection{第2正規形}


上の表は縦方向にデータが繰り返されている。
これを表を分けることによって、解消する。
名前、性別等の列は、主キーである \textsf{ID} に従属している。
趣味の列は、この表には隠れている別の主キー に従属していると考える。
今回の場合は、emp表の \textsf{ID} と hobby表の \textsf{HID} である。
この2つのキーを主キー(複合キー)として、それに従属していると考える。

\begin{multicols}{2}

 emp表 \\
  \begin{tabular}[h]{|c|l|c|c|c|c|}
   \hline
   \underline{ID}
      & 名前       & 性別 & 年齢 & 誕生年 & 部署  \\ \hline\hline
   1  & 菅原文太   & 男性 & 40   & 1933   & 総務  \\ \hline
   2  & 千葉真一   & 男性 & 34   & 1939   & 営業  \\ \hline
   3  & 北大路欣也 & 男性 & 30   & 1943   & 経理  \\ \hline
   4  & 梶芽衣子   & 女性 & 26   & 1947   & 営業  \\ \hline
  \end{tabular}

 \columnbreak

 hobby表 \\
  \begin{tabular}[h]{|c|l|}
   \hline
   \underline{HID}
         & 趣味 \\ \hline\hline
   H01   & 釣り \\ \hline
   H02   & 油絵 \\ \hline
   H03   & 空手  \\ \hline
   H04   & 熱帯魚飼育 \\ \hline
   H05   & サッカー観戦 \\ \hline
   H06   & 茶道 \\ \hline
   H07   & 登山 \\ \hline
   H08   & ヨガ \\ \hline
  \end{tabular}
\end{multicols}

\noindent
emp\_hobby表\\
\begin{tabular}{|c|c|cc} \cline{1-2}
 \multicolumn{2}{|c|}{主キー} & \multirow{2}{*}{名前} & \multirow{2}{*}{趣味} \\ \cline{1-2}
 \underline{ID} & \underline{HID} &   &   \\ \cline{1-2}\cline{1-2}
 1  & H01 & 菅原文太 &  釣り \\ \cline{1-2}
 1  & H02 & 菅原文太 & 油絵 \\ \cline{1-2}
 1  & H03 & 菅原文太 & 空手 \\ \cline{1-2}
 2  & H03 & 千葉真一 & 空手 \\ \cline{1-2}
 2  & H04 & 千葉真一 & 熱帯魚飼育 \\ \cline{1-2}
 2  & H05 & 千葉真一 & サッカー観戦 \\ \cline{1-2}
 2  & H01 & 千葉真一 & 釣り \\ \cline{1-2}
 3  & H06 & 北大路欣也 & 茶道 \\ \cline{1-2}
 3  & H03 & 北大路欣也 & 空手 \\ \cline{1-2}
 4  & H07 & 梶芽衣子 & 登山 \\ \cline{1-2}
 4  & H08 & 梶芽衣子 & ヨガ \\ \cline{1-2}
 4  & H05 & 梶芽衣子 & サッカー観戦 \\ \cline{1-2}
\end{tabular}

\vspace{5mm}
\fbox{\large \textgt{第2正規形} :\quad 単一列の主キーであるか、もしくは、すべての列が主キーの一部である。}



\newpage
\subsection{第3正規形}

emp表には、表には現れていないが、id に従属せず、別のキーに従属すると
考えられるフィールドが存在する。それは「性別」と「所属」である。
これを別の表(gender表、dept表)とする。

\begin{multicols}{2}
 emp表 \\
  \begin{tabular}[h]{|c|l|c|c|}
   \hline
   \underline{ID}
      & 名前       & 年齢 & 誕生年 \\ \hline\hline
   1  & 菅原文太   & 40   & 1933   \\ \hline
   2  & 千葉真一   & 34   & 1939   \\ \hline
   3  & 北大路欣也 & 30   & 1943   \\ \hline
   4  & 梶芽衣子   & 26   & 1947   \\ \hline
  \end{tabular}

 \columnbreak

 gender表 \\
 \begin{tabular}{|c|c|} \hline
  \underline{GID}
      & 性別 \\  \hline\hline
  0   & 不明 \\ \hline
  1   & 男性 \\ \hline
  2   & 女性 \\ \hline
  3   & その他 \\ \hline
 \end{tabular}
\end{multicols}

\begin{multicols}{3}
 dept表 \\
 \begin{tabular}{|c|c|} \hline
  \underline{DID}
      & 部署名 \\ \hline\hline
  D01 & 総務部 \\ \hline
  D02 & 営業部 \\ \hline
  D03 & 経理部 \\ \hline
  D04 & 開発部 \\ \hline
 \end{tabular}
  
 \columnbreak

 hobby表 \\
  \begin{tabular}[h]{|c|l|}
   \hline
   \underline{HID}
         & 趣味 \\ \hline\hline
   H01   & 釣り \\ \hline
   H02   & 油絵 \\ \hline
   H03   & 空手  \\ \hline
   H04   & 熱帯魚飼育 \\ \hline
   H05   & サッカー観戦 \\ \hline
   H06   & 茶道 \\ \hline
   H07   & 登山 \\ \hline
   H08   & ヨガ \\ \hline
  \end{tabular}

 \columnbreak

 \noindent
 emp\_hobby表\\
 \begin{tabular}{|c|c|} \hline
  \multicolumn{2}{|c|}{主キー} \\ \hline
  \underline{ID} & \underline{HID}  \\ \hline\hline
  1  & H01  \\ \hline
  1  & H02  \\ \hline
  1  & H03  \\ \hline
  2  & H03  \\ \hline
  2  & H04  \\ \hline
  2  & H05  \\ \hline
  2  & H01  \\ \hline
  3  & H06  \\ \hline
  3  & H03  \\ \hline
  4  & H07  \\ \hline
  4  & H08  \\ \hline
  4  & H05  \\ \hline
 \end{tabular}
\end{multicols}

\vspace{5mm}
\fbox{\large \textgt{第3正規形} :\quad すべての列が主キーに従属している。}


\newpage
\section{ER図}

第3正規形ができたところで、ER図を作ってみる。

\vspace{3mm}
\includegraphics[width=14cm]{img/ER.png}
\vspace{3mm}







\include{end}

%% 修正時刻: Sat Feb 19 19:08:31 2022
