\documentclass[dvipdfmx]{jsarticle}


\usepackage{tcolorbox}
\usepackage{color}
\usepackage{listings, plistings}

%% ノート/latexメモ
%% http://pepper.is.sci.toho-u.ac.jp/pepper/index.php?%A5%CE%A1%BC%A5%C8%2Flatex%A5%E1%A5%E2

%% JavaScriptの設定
%% https://e8l.hatenablog.com/entry/2015/11/29/232800
\lstdefinelanguage{javascript}{
  morekeywords = [1]{ %keywords
    await, break, case, catch, class, const, continue, debugger, default, delete, 
    do, else, enum, export, extends, finally, for, function, function*, if, implements, import, in, 
    instanceof, interface, let, new, package, private, protected, public, return, static, super,
    switch, this, throw, try, typeof, var, void, while, with, yield, yield*
  },
  morekeywords = [2]{ %literal
    false, Infinity, NaN, null, true, undefined
  },
  morekeywords = [3] { %Classes
    Array, ArrayBuffer, Boolean, DataView, Date, Error, EvalError, Float32Array, Float64Array,
    Function, Generator, GeneratorFunction, Int16Array, Int32Array, Int8Array, InternalError,
    JSON, Map, Math, Number, Object, Promise, Proxy, RangeError, ReferenceError, Reflect,
    RegExp, Set, String, Symbol, SyntaxError, TypeError, URIError, Uint16Array, Uint32Array,
    Uint8Array, Uint8ClampedArray, WeakMap, WeakSet
  },
  morecomment = [l]{//},
  morecomment = [s]{/*}{*/},
  morestring = [b]{"},
  morestring = [b]{'},
  alsodigit = {-},
  sensitive = true
}

%% 修正時刻: Tue 2022/03/15 10:04:41


% Java
\lstset{% 
  frame=single,
  backgroundcolor={\color[gray]{.9}},
  stringstyle={\ttfamily \color[rgb]{0,0,1}},
  commentstyle={\itshape \color[cmyk]{1,0,1,0}},
  identifierstyle={\ttfamily}, 
  keywordstyle={\ttfamily \color[cmyk]{0,1,0,0}},
  basicstyle={\ttfamily},
  breaklines=true,
  xleftmargin=0zw,
  xrightmargin=0zw,
  framerule=.2pt,
  columns=[l]{fullflexible},
  numbers=left,
  stepnumber=1,
  numberstyle={\scriptsize},
  numbersep=1em,
  language={Java},
  lineskip=-0.5zw,
  morecomment={[s][{\color[cmyk]{1,0,0,0}}]{/**}{*/}},
  keepspaces=true,         % 空白の連続をそのままで
  showstringspaces=false,  % 空白字をOFF
}
%\usepackage[dvipdfmx]{graphicx}
\usepackage{url}
\usepackage[dvipdfmx]{hyperref}
\usepackage{amsmath, amssymb}
\usepackage{itembkbx}
\usepackage{eclbkbox}	% required for `\breakbox' (yatex added)
\usepackage{enumerate}
\usepackage[default]{cantarell}
\usepackage[T1]{fontenc}
\fboxrule=0.5pt
\parindent=1em
\definecolor{mygrey}{rgb}{0.97, 0.97, 0.97}

\makeatletter
\def\verbatim@font{\normalfont
\let\do\do@noligs
\verbatim@nolig@list}
\makeatother

\begin{document}

%\anaumeと入力すると穴埋め解答欄が作れるようにしてる。\anaumesmallで小さめの穴埋めになる。
\newcounter{mycounter} % カウンターを作る
\setcounter{mycounter}{0} % カウンターを初期化
\newcommand{\anaume}[1][]{\refstepcounter{mycounter}{#1}{\boxed{\phantom{aa}\textnormal{\themycounter}\phantom{aa}}}} %穴埋め問題の空欄作ってる。
\newcommand{\anaumesmall}[1][]{\refstepcounter{mycounter}{#1}{\boxed{\tiny{\phantom{a}\themycounter \phantom{a}}}}}%小さい版作ってる。色々改造できる。

%% 修正時刻: Tue 2022/03/15 10:04:411


\section{ちょっと複雑なデータベースを考える}

\subsection{データベースの例}

今度は以下のようなデータについて考えてみる。

\vspace{3mm}
 \begin{tabular}{|l|l|} \hline
  氏名 & 菅原文太 \\
  年齢 & 40歳 \\
  誕生年 & 1933年生まれ \\
  部署 & 総務部 \\ 
  趣味 & 釣り、油絵  \\ \hline
 \end{tabular}

\vspace{3mm}
 \begin{tabular}{|l|l|} \hline
  氏名 & 千葉真一 \\
  年齢 & 34歳 \\
  誕生年 &  1939年生まれ \\
  部署 & 営業部 \\
  趣味 & 空手、熱帯魚飼育、サッカー観戦  \\ \hline
 \end{tabular}

\vspace{3mm}
 \begin{tabular}{|l|l|} \hline
  氏名 & 北大路欣也 \\
  年齢 & 30歳 \\
  誕生年 &  1943年生まれ \\
  部署 & 経理部 \\
  趣味 & 茶道  \\ \hline
 \end{tabular}

\vspace{3mm}
 \begin{tabular}{|l|l|} \hline
  氏名 & 梶芽衣子 \\
  年齢 & 26歳 \\
  誕生年 &  1947年生まれ \\
  部署 & 営業部 \\
  趣味 & 登山、ヨガ  \\ \hline
 \end{tabular}
\vspace{3mm}

まず、このような表がイメージされる。

\begin{table}[h]
 \caption{emp}
 \begin{center}
  \begin{tabular}[h]{|c|l|c|c|c|l|}
   \hline
   ID & 名前       & 年齢 & 誕生年 & 部署 & 趣味 \\ \hline\hline
   1  & 菅原文太   & 40   & 1933   & 総務 & 釣り, 油絵  \\ \hline
   2  & 千葉真一   & 34   & 1939   & 営業 & 空手, 熱帯魚飼育, サッカー観戦 \\ \hline
   3  & 北大路欣也 & 30   & 1943   & 経理 & 茶道 \\ \hline
   4  & 梶芽衣子   & 26   & 1947   & 営業 & 登山, ヨガ  \\ \hline
  \end{tabular}
 \end{center}
\end{table}

しかしながら、この表の場合、趣味のフィールドには、複数のデータが含まれている。
これを解消したのが、次の表である。

\subsection{第1正規形}

\begin{table}[h]
 \caption{emp}
 \begin{center}
  \begin{tabular}[h]{|c|l|c|c|c|l|l|l|}
   \hline
   ID & 名前       & 年齢 & 誕生年 & 部署 & 趣味1 & 趣味2 & 趣味3 \\ \hline\hline
   1  & 菅原文太   & 40   & 1933   & 総務 & 釣り & 油絵 & \\ \hline
   2  & 千葉真一   & 34   & 1939   & 営業 & 空手 & 熱帯魚飼育 & サッカー観戦 \\ \hline
   3  & 北大路欣也 & 30   & 1943   & 経理 & 茶道 & & \\ \hline
   4  & 梶芽衣子   & 26   & 1947   & 営業 & 登山 & ヨガ & \\ \hline
  \end{tabular}
 \end{center}
\end{table}


趣味のフィールドが1つのもの、3つのものとバラバラなので、フィールドを1つにする。

\begin{table}[h]
 \caption{emp}
 \begin{center}
  \begin{tabular}[h]{|c|l|c|c|c|l|}
   \hline
   ID & 名前       & 年齢 & 誕生年 & 部署 & 趣味 \\ \hline\hline
   1  & 菅原文太   & 40   & 1933   & 総務 & 釣り \\ \hline
   1  & 菅原文太   & 40   & 1933   & 総務 & 油絵 \\ \hline
   2  & 千葉真一   & 34   & 1939   & 営業 & 空手  \\ \hline
   2  & 千葉真一   & 34   & 1939   & 営業 & 熱帯魚飼育 \\ \hline
   2  & 千葉真一   & 34   & 1939   & 営業 & サッカー観戦 \\ \hline
   3  & 北大路欣也 & 30   & 1943   & 経理 & 茶道 \\ \hline
   4  & 梶芽衣子   & 26   & 1947   & 営業 & 登山 \\ \hline
   4  & 梶芽衣子   & 26   & 1947   & 営業 & ヨガ \\ \hline
  \end{tabular}
 \end{center}
\end{table}

ここまでが「第1正規化」で、この表を「第1正規形」と呼ぶ。

\subsection{第2正規形}


上の表は縦方向にデータが繰り返されている。
これを表を分けることによって、解消する。

\begin{multicols}{2}

 emp表 \\
  \begin{tabular}[h]{|c|l|c|c|c|}
   \hline
   ID & 名前       & 年齢 & 誕生年 & 部署  \\ \hline\hline
   1  & 菅原文太   & 40   & 1933   & 総務  \\ \hline
   2  & 千葉真一   & 34   & 1939   & 営業  \\ \hline
   3  & 北大路欣也 & 30   & 1943   & 経理  \\ \hline
   4  & 梶芽衣子   & 26   & 1947   & 営業  \\ \hline
  \end{tabular}

 \columnbreak

 hobby表 \\
  \begin{tabular}[h]{|c|l|}
   \hline
   HID    & 趣味 \\ \hline\hline
   H01   & 釣り \\ \hline
   H02   & 油絵 \\ \hline
   H03   & 空手  \\ \hline
   H04   & 熱帯魚飼育 \\ \hline
   H05   & サッカー観戦 \\ \hline
   H06   & 茶道 \\ \hline
   H07   & 登山 \\ \hline
   H08   & ヨガ \\ \hline
  \end{tabular}
\end{multicols}


emp\_hobby表\\
\begin{tabular}{|c|c|c|} \hline
 \multicolumn{2}{|c|}{主キー} & \multirow{2}{*}{趣味} \\ \cline{1-2}
t ID & HID &      \\ \hline\hline
 1  & H01 & 釣り \\ \hline
 1  & H02 & 油絵 \\ \hline
 2  & H03 & 空手 \\ \hline
 2  & H04 & 熱帯魚飼育 \\ \hline
 2  & H05 & サッカー観戦 \\ \hline
 3  & H06 & 茶道 \\ \hline
 4  & H07 & 登山 \\ \hline
 4  & H08 & ヨガ \\ \hline
\end{tabular}


\subsection{第3正規形}

emp表には、表には現れていないが、id に従属せず、別のキーに従属すると
考えられるフィールドが存在する。それは「所属」である。
これを別の表(dept表)とする。

\begin{multicols}{2}
 emp表 \\
  \begin{tabular}[h]{|c|l|c|c|}
   \hline
   ID & 名前       & 年齢 & 誕生年 \\ \hline\hline
   1  & 菅原文太   & 40   & 1933   \\ \hline
   2  & 千葉真一   & 34   & 1939   \\ \hline
   3  & 北大路欣也 & 30   & 1943   \\ \hline
   4  & 梶芽衣子   & 26   & 1947   \\ \hline
  \end{tabular}

 \columnbreak

 dept表 \\
 \begin{tabular}{|c|c|} \hline
  DID & 部署名 \\ \hline\hline
  D01 & 総務部 \\ \hline
  D02 & 営業部 \\ \hline
  D03 & 経理部 \\ \hline
  D04 & 開発部 \\ \hline
 \end{tabular}
\end{multicols}
  
\begin{multicols}{2}
 hobby表 \\
  \begin{tabular}[h]{|c|l|}
   \hline
   HID   & 趣味 \\ \hline\hline
   H01   & 釣り \\ \hline
   H02   & 油絵 \\ \hline
   H03   & 空手  \\ \hline
   H04   & 熱帯魚飼育 \\ \hline
   H05   & サッカー観戦 \\ \hline
   H06   & 茶道 \\ \hline
   H07   & 登山 \\ \hline
   H08   & ヨガ \\ \hline
  \end{tabular}

 \columnbreak
 
 emp\_hobby表\\
 \begin{tabular}{|c|c|c|} \hline
  \multicolumn{2}{|c|}{主キー} & \multirow{2}{*}{趣味} \\ \cline{1-2}
  ID & HID &      \\ \hline\hline
  1  & H01 & 釣り \\ \hline
  1  & H02 & 油絵 \\ \hline
  2  & H03 & 空手 \\ \hline
  2  & H04 & 熱帯魚飼育 \\ \hline
  2  & H05 & サッカー観戦 \\ \hline
  3  & H06 & 茶道 \\ \hline
  4  & H07 & 登山 \\ \hline
  4  & H08 & ヨガ \\ \hline
 \end{tabular}
\end{multicols}


\section{ER図}

第3正規形ができたところで、ER図を作ってみる。

emp表\\
\begin{tabular}{|c|} \hline
 ID \\ \hline
 name \\ \hline
 age \\ \hline
 birthyear \\ \hline
\end{tabular}
 

\end{document}

%% 修正時刻: Sat May  2 15:10:04 2020


%% 修正時刻: Fri Feb 18 15:07:09 2022
