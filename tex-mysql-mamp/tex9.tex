
\documentclass[dvipdfmx]{jsarticle}

\include{begin}

\section{練習問題}

次の各表からデータベースを設計してください。ER図を書くのをゴールとします。

データベース名は \textsf{mountain}。
ユーザー名は 'moutain\_user'。パスワードは '1234' です。

また、その中に、teizanテーブルがあります。

\begin{multicols}{2}
\vspace{3mm}
\noindent
\begin{tabular}{|l|l|l|l|} \hline
id & 名前       & 性別 & 誕生日 \\ \hline\hline 
1 & 染谷将太    & 男性 & 1992-09-03 \\ \hline
2 & 二階堂ふみ  & 女性 & 1994-09-21 \\ \hline
3 & 渡辺哲      & 男性 & 1950-03-11 \\ \hline
4 & 窪塚洋介    & 男性 & 1979-05-07 \\ \hline
5 & 吉高由里子  & 女性 & 1988-07-22  \\ \hline
\end{tabular}
\vspace{3mm}

\columnbreak
 
\vspace{3mm}
\noindent
\begin{tabular}{|l|l|l|l|l|} \hline
 id & 山岳名   & かな           &  標高(m) & 所在地   \\ \hline\hline
 20 & 八溝山    & やみぞさん      &   1022  & 福島県   \\ \hline 
 30 & 荒船山    & あらふねやま    &   1423  & 群馬県    \\ \hline
 50 & 大山      & おおやま        &   1252  & 神奈川県  \\ \hline
 44 & 高尾山    & たかおさん      &    599  & 東京都    \\ \hline
 50 & 大山      & おおやま        &   1252  & 神奈川県  \\ \hline
 82 & 比叡山    & ひえいざん      &    848  & 京都府    \\ \hline
 53 & 駒ガ岳    & こまがたけ      &   1438  & 神奈川県  \\ \hline
 58 & 弥彦山    & やひこやま      &    634  & 新潟県    \\ \hline
 66 & 身延山    & みのぶさん      &   1153  & 山梨県    \\ \hline
 67 & 高社山    & たかやしろやま  &   1351  & 長野県    \\ \hline
 26 & 榛名山    & はるなさん      &   1449  & 群馬県    \\ \hline
 30 & 荒船山    & あらふねやま    &   1423  & 群馬県    \\ \hline
 43 & 権現山    & ごんげんやま    &   1312  & 山梨県    \\ \hline
 66 & 身延山    & みのぶさん      &   1153  & 山梨県    \\ \hline
 30 & 荒船山    & あらふねやま    &   1423  & 群馬県    \\ \hline
 50 & 大山      & おおやま        &   1252  & 神奈川県  \\ \hline
 53 & 駒ガ岳    & こまがたけ      &   1438  & 神奈川県  \\ \hline
 80 & 伊吹山    & いぶきやま      &   1377  & 滋賀県    \\ \hline
 82 & 比叡山    & ひえいざん      &    848  & 京都府    \\ \hline
\end{tabular}
\vspace{3mm}
\end{multicols}

\vspace{3mm}
\noindent
\begin{tabular}{|l|l|l|} \hline
id & 名前       & 登った山 \\ \hline\hline
1 & 染谷将太    & 八溝山, 荒船山, 大山  \\ \hline
2 & 二階堂ふみ  & 榛名山, 荒船山, 権現山, 身延山 \\ \hline
3 & 渡辺哲      & 高尾山, 大山, 比叡山 \\ \hline
4 & 窪塚洋介    & 駒ガ岳, 弥彦山, 身延山, 高社山 \\ \hline
5 & 吉高由里子  & 荒船山, 大山, 駒ガ岳, 伊吹山, 比叡山 \\ \hline
\end{tabular}
\vspace{3mm}




\include{end}

%% 修正時刻: Sat Feb 19 22:19:46 2022
