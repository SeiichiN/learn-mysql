\documentclass[dvipdfmx]{jsarticle}

\include{begin}

\section{データベースをバックアップする}

\subsection{バックアップ}

データの入力が終わったら、データベースをバックアップする。

ここでは、\textsf{mysqldump} を使っておこなう。

まず、MySQLをログアウトする。

\begin{tcolorbox}
 MariaDB [sample]$>$ exit (もしくは quit) \\
 C:\yen Users\yen XXXXXX\yen Documents\yen mysql$>$
\end{tcolorbox}

\rightline{※ XXXXXX は、各自のユーザー名}

コマンドプロンプトに戻る。

ここで以下のコマンドを実行する。

\begin{tcolorbox}
 $>$ mysqldump -u sampleuser -p --databases sample $>$ sample\_db.dump
\end{tcolorbox}

\textsf{dir} というコマンドを実行すると、ファイル一覧が見れる。

ファイル群の中に \textsf{sample\_db.dump} があるはず。

これは テキストファイルなので、TeraPad などのエディタで内容を見ることができる。
\footnote{また、{\em --databases} オプションをつけずにバックアップを取ることもできる。 \\
\fbox{ $>$ mysqldump -u sampleuser -p sample $>$ sample\_db.dump } \\
この場合は、バックアップファイルの記述の中に、CREATE DATABASE 文がない。
}

\subsection{データのリストア(復元)}

以下のコマンドを実行する。
\footnote{--databases オプションをつけずにバックアップファイルを作成した場合は、まず sample
データベースを作成してから \\
\fbox{ $>$ mysql -u sampleuser -p $<$ sample\_db.dump } \\
とするか、あるいは、sampleuserでログインしてから \\
\fbox{MariaDB[sample] $>$ source sample\_db.dump} とすればよい。
}
\begin{tcolorbox}
 $>$ mysql -u sampleuser -p $<$ sample\_db.dump
\end{tcolorbox}



これでリストアができている。



\include{end}

%% 修正時刻: Fri Oct  1 18:59:18 2021
