\documentclass[dvipdfmx]{jsarticle}


\usepackage{tcolorbox}
\usepackage{color}
\usepackage{listings, plistings}

%% ノート/latexメモ
%% http://pepper.is.sci.toho-u.ac.jp/pepper/index.php?%A5%CE%A1%BC%A5%C8%2Flatex%A5%E1%A5%E2

%% JavaScriptの設定
%% https://e8l.hatenablog.com/entry/2015/11/29/232800
\lstdefinelanguage{javascript}{
  morekeywords = [1]{ %keywords
    await, break, case, catch, class, const, continue, debugger, default, delete, 
    do, else, enum, export, extends, finally, for, function, function*, if, implements, import, in, 
    instanceof, interface, let, new, package, private, protected, public, return, static, super,
    switch, this, throw, try, typeof, var, void, while, with, yield, yield*
  },
  morekeywords = [2]{ %literal
    false, Infinity, NaN, null, true, undefined
  },
  morekeywords = [3] { %Classes
    Array, ArrayBuffer, Boolean, DataView, Date, Error, EvalError, Float32Array, Float64Array,
    Function, Generator, GeneratorFunction, Int16Array, Int32Array, Int8Array, InternalError,
    JSON, Map, Math, Number, Object, Promise, Proxy, RangeError, ReferenceError, Reflect,
    RegExp, Set, String, Symbol, SyntaxError, TypeError, URIError, Uint16Array, Uint32Array,
    Uint8Array, Uint8ClampedArray, WeakMap, WeakSet
  },
  morecomment = [l]{//},
  morecomment = [s]{/*}{*/},
  morestring = [b]{"},
  morestring = [b]{'},
  alsodigit = {-},
  sensitive = true
}

%% 修正時刻: Tue 2022/03/15 10:04:41


% Java
\lstset{% 
  frame=single,
  backgroundcolor={\color[gray]{.9}},
  stringstyle={\ttfamily \color[rgb]{0,0,1}},
  commentstyle={\itshape \color[cmyk]{1,0,1,0}},
  identifierstyle={\ttfamily}, 
  keywordstyle={\ttfamily \color[cmyk]{0,1,0,0}},
  basicstyle={\ttfamily},
  breaklines=true,
  xleftmargin=0zw,
  xrightmargin=0zw,
  framerule=.2pt,
  columns=[l]{fullflexible},
  numbers=left,
  stepnumber=1,
  numberstyle={\scriptsize},
  numbersep=1em,
  language={Java},
  lineskip=-0.5zw,
  morecomment={[s][{\color[cmyk]{1,0,0,0}}]{/**}{*/}},
  keepspaces=true,         % 空白の連続をそのままで
  showstringspaces=false,  % 空白字をOFF
}
%\usepackage[dvipdfmx]{graphicx}
\usepackage{url}
\usepackage[dvipdfmx]{hyperref}
\usepackage{amsmath, amssymb}
\usepackage{itembkbx}
\usepackage{eclbkbox}	% required for `\breakbox' (yatex added)
\usepackage{enumerate}
\usepackage[default]{cantarell}
\usepackage[T1]{fontenc}
\fboxrule=0.5pt
\parindent=1em
\definecolor{mygrey}{rgb}{0.97, 0.97, 0.97}

\makeatletter
\def\verbatim@font{\normalfont
\let\do\do@noligs
\verbatim@nolig@list}
\makeatother

\begin{document}

%\anaumeと入力すると穴埋め解答欄が作れるようにしてる。\anaumesmallで小さめの穴埋めになる。
\newcounter{mycounter} % カウンターを作る
\setcounter{mycounter}{0} % カウンターを初期化
\newcommand{\anaume}[1][]{\refstepcounter{mycounter}{#1}{\boxed{\phantom{aa}\textnormal{\themycounter}\phantom{aa}}}} %穴埋め問題の空欄作ってる。
\newcommand{\anaumesmall}[1][]{\refstepcounter{mycounter}{#1}{\boxed{\tiny{\phantom{a}\themycounter \phantom{a}}}}}%小さい版作ってる。色々改造できる。

%% 修正時刻: Tue 2022/03/15 10:04:411


\section{データベースをバックアップする}

\subsection{バックアップ}

データの入力が終わったら、データベースをバックアップする。

ここでは、\textsf{mysqldump} を使っておこなう。

まず、MySQLをログアウトする。

\begin{tcolorbox}
 MariaDB [sample]$>$ exit (もしくは quit) \\
 C:\yen Users\yen XXXXXX\yen Documents\yen mysql$>$
\end{tcolorbox}

\rightline{※ XXXXXX は、各自のユーザー名}

コマンドプロンプトに戻る。

ここで以下のコマンドを実行する。

\begin{tcolorbox}
 $>$ mysqldump \ -u \ sampleuser \ -p \ -\! -databases \ sample \ $>$ \ sample\_db.dump
\end{tcolorbox}

\begin{tabular}{|l|} \hline
 mysqldump \ -u \ ユーザー名 \ -p \ -\! -databases \ データベース名 \ $>$
 保存ファイル名 \\ \hline
\end{tabular}

\rightline{※ -\! -databases は -(ハイフン)2つ}

ここでは保存ファイル名を sample\_db.dump としたが、好みのファイル名を指定すればよい。

\textsf{dir} というコマンドを実行すると、ファイル一覧が見れる。

ファイル群の中に \textsf{sample\_db.dump} があるはず。

これは テキストファイルなので、TeraPad などのエディタで内容を見ることができる。
\footnote{また、{\em --databases} オプションをつけずにバックアップを取ることもできる。 \\
\fbox{ $>$ mysqldump -u sampleuser -p sample $>$ sample\_db.dump } \\
この場合は、バックアップファイルの記述の中に、CREATE DATABASE 文がない。
}

\subsection{データのリストア(復元)}

以下のコマンドを実行する。
\footnote{--databases オプションをつけずにバックアップファイルを作成した場合は、まず sample
データベースを作成してから \\
\fbox{ $>$ mysql -u sampleuser -p $<$ sample\_db.dump } \\
とするか、あるいは、sampleuserでログインしてから \\
\fbox{MariaDB[sample] $>$ source sample\_db.dump} とすればよい。
}
\begin{tcolorbox}
 $>$ mysql -u sampleuser -p $<$ sample\_db.dump
\end{tcolorbox}

\fbox{mysql \ -u \ ユーザー名 \ -p \ $<$ \ 保存したファイル名}
\vspace{3mm}

これでリストアができている。



\end{document}

%% 修正時刻: Sat May  2 15:10:04 2020


%% 修正時刻: Sat Oct  9 07:53:15 2021
