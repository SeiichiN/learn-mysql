\documentclass[dvipdfmx]{jsarticle}

\include{begin}

\section{データベースをバックアップする}

\subsection{バックアップ}

データの入力が終わったら、データベースをバックアップする。

ここでは、\textsf{mysqldump} を使っておこなう。

まず、MySQLをログアウトする。

\begin{tcolorbox}
 MariaDB [sample]$>$ exit (もしくは quit) \\
 C:\yen Users\yen XXXXXX\yen Documents\yen mysql$>$
\end{tcolorbox}

\rightline{※ XXXXXX は、各自のユーザー名}

コマンドプロンプトに戻る。

ここで以下のコマンドを実行する。

\begin{tcolorbox}
 $>$ mysqldump -u sampleuser -p sample $>$ sample\_db.dump
\end{tcolorbox}

\textsf{dir} というコマンドを実行すると、ファイル一覧が見れる。

ファイル群の中に \textsf{sample\_db.dump} があるはず。

これは テキストファイルなので、TeraPad などのエディタで内容を見ることができる。

\subsection{データのリストア(復元)}

データをリストアするには、まず、そのデータベースが存在していなければならない。

今回でいえば、sampleuserユーザーを作成し、sampleデータベースを作っておく。
\footnote{
sampleuser というユーザーが存在するか調べたいので、ユーザー一覧を表示させる。\\
\textsf{$>$ select host, name from mysql.user;} \\
sampleuser というユーザーを作成し、なおかつ sample データベースへの権限を付与する。\\
\textsf{$>$ grant all on sample.* to 'sampleuser'@'localhost' identified by 'sampleuser';} \\
sample データベースの作成。\\
\textsf{$>$ create database sample;}
}

それから、以下のコマンドを実行する。

\begin{tcolorbox}
 $>$ mysql -u sampleuser -p sample $<$ sample\_db.dump
\end{tcolorbox}

これでリストアができている。



\include{end}

%% 修正時刻: Wed Aug 11 09:16:28 2021
