\documentclass[dvipdfmx]{jsarticle}


\usepackage{tcolorbox}
\usepackage{color}
\usepackage{listings, plistings}

%% ノート/latexメモ
%% http://pepper.is.sci.toho-u.ac.jp/pepper/index.php?%A5%CE%A1%BC%A5%C8%2Flatex%A5%E1%A5%E2

%% JavaScriptの設定
%% https://e8l.hatenablog.com/entry/2015/11/29/232800
\lstdefinelanguage{javascript}{
  morekeywords = [1]{ %keywords
    await, break, case, catch, class, const, continue, debugger, default, delete, 
    do, else, enum, export, extends, finally, for, function, function*, if, implements, import, in, 
    instanceof, interface, let, new, package, private, protected, public, return, static, super,
    switch, this, throw, try, typeof, var, void, while, with, yield, yield*
  },
  morekeywords = [2]{ %literal
    false, Infinity, NaN, null, true, undefined
  },
  morekeywords = [3] { %Classes
    Array, ArrayBuffer, Boolean, DataView, Date, Error, EvalError, Float32Array, Float64Array,
    Function, Generator, GeneratorFunction, Int16Array, Int32Array, Int8Array, InternalError,
    JSON, Map, Math, Number, Object, Promise, Proxy, RangeError, ReferenceError, Reflect,
    RegExp, Set, String, Symbol, SyntaxError, TypeError, URIError, Uint16Array, Uint32Array,
    Uint8Array, Uint8ClampedArray, WeakMap, WeakSet
  },
  morecomment = [l]{//},
  morecomment = [s]{/*}{*/},
  morestring = [b]{"},
  morestring = [b]{'},
  alsodigit = {-},
  sensitive = true
}

%% 修正時刻: Tue 2022/03/15 10:04:41


% Java
\lstset{% 
  frame=single,
  backgroundcolor={\color[gray]{.9}},
  stringstyle={\ttfamily \color[rgb]{0,0,1}},
  commentstyle={\itshape \color[cmyk]{1,0,1,0}},
  identifierstyle={\ttfamily}, 
  keywordstyle={\ttfamily \color[cmyk]{0,1,0,0}},
  basicstyle={\ttfamily},
  breaklines=true,
  xleftmargin=0zw,
  xrightmargin=0zw,
  framerule=.2pt,
  columns=[l]{fullflexible},
  numbers=left,
  stepnumber=1,
  numberstyle={\scriptsize},
  numbersep=1em,
  language={Java},
  lineskip=-0.5zw,
  morecomment={[s][{\color[cmyk]{1,0,0,0}}]{/**}{*/}},
  keepspaces=true,         % 空白の連続をそのままで
  showstringspaces=false,  % 空白字をOFF
}
%\usepackage[dvipdfmx]{graphicx}
\usepackage{url}
\usepackage[dvipdfmx]{hyperref}
\usepackage{amsmath, amssymb}
\usepackage{itembkbx}
\usepackage{eclbkbox}	% required for `\breakbox' (yatex added)
\usepackage{enumerate}
\usepackage[default]{cantarell}
\usepackage[T1]{fontenc}
\fboxrule=0.5pt
\parindent=1em
\definecolor{mygrey}{rgb}{0.97, 0.97, 0.97}

\makeatletter
\def\verbatim@font{\normalfont
\let\do\do@noligs
\verbatim@nolig@list}
\makeatother

\begin{document}

%\anaumeと入力すると穴埋め解答欄が作れるようにしてる。\anaumesmallで小さめの穴埋めになる。
\newcounter{mycounter} % カウンターを作る
\setcounter{mycounter}{0} % カウンターを初期化
\newcommand{\anaume}[1][]{\refstepcounter{mycounter}{#1}{\boxed{\phantom{aa}\textnormal{\themycounter}\phantom{aa}}}} %穴埋め問題の空欄作ってる。
\newcommand{\anaumesmall}[1][]{\refstepcounter{mycounter}{#1}{\boxed{\tiny{\phantom{a}\themycounter \phantom{a}}}}}%小さい版作ってる。色々改造できる。

%% 修正時刻: Tue 2022/03/15 10:04:411


\section{制約}

\subsection{外部キー制約}

\subsubsection{表定義に制約をつけてみる}

最初の表を入力するところに戻る。
以下の表を作成して入力するのだった。

\begin{table}[h]
 \caption{emp}
 \begin{center}
  \begin{tabular}[h]{|c|l|c|c|c|}
   \hline
   ID & 名前       & 年齢 & 誕生年 & 部署ID \\ \hline\hline
   1  & 菅原文太   & 40   & 1933   & 001    \\ \hline
   2  & 千葉真一   & 34   & 1939   & 002    \\ \hline
   3  & 北大路欣也 & 30   & 1943   & 003    \\ \hline
   4  & 梶芽衣子   & 26   & 1947   & 002    \\ \hline
  \end{tabular}
 \end{center}
\end{table}

\begin{table}[h]
 \caption{dep}
 \begin{center}
  \begin{tabular}{|l|c|} \hline
   ID   & 部署名 \\ \hline\hline
   001  & 総務部 \\ \hline
   002  & 営業部 \\ \hline
   003  & 経理部 \\ \hline
   004  & 開発部 \\ \hline
  \end{tabular}
 \end{center}
\end{table}

emp表のデータの ''部署ID'' を入力するとき、dept表にない番号を入力すると
まずいことになる。

そこで、emp表の ''部署ID'' を入力するときに、dept表にある番号だけを
入力するように 制限 をかけることができる。

これを ''外部キー制約'' という。

emp表の dept\_id に入力する値は dept表にある値に制限するのであるから、
emp表を定義する前に dept表が定義されていなくてはならない。

\textgt{dept表の定義(再掲)}

\begin{tcolorbox}
 MariaDB [sample]$>$ create table dept ( \\
 \hspace{6mm} \verb!->! id char(3) primary key, (カンマ) \\
 \hspace{6mm} \verb!->! name varchar(20) not null (カンマなし) \\
 \hspace{6mm} \verb!->! );
\end{tcolorbox}


\textgt{emp表の定義(外部キー制約)}

\begin{tcolorbox}
 MariaDB [sample]$>$ create table emp ( \\
 \hspace{6mm} \verb!->! id int primary key auto\_increment,  \\
 \hspace{6mm} \verb!->! name varchar(20) not null, \\
 \hspace{6mm} \verb!->! age int not null, \\
 \hspace{6mm} \verb!->! birthday year not null, \\
 \hspace{6mm} \verb!->! dept\_id char(3),  (カンマをつける)\\
 \hspace{6mm} \verb!->! \underline{foreign key(dept\_id) references dept(id)} \\
 \hspace{6mm} \verb!->! );
\end{tcolorbox}

現在の empテーブル、deptテーブルを削除して、再定義、初期データを
入力する。
そのためのスクリプトは、以下である。

\begin{lstlisting}[caption=reinit\_data.sql]
 -- もし存在していなかったら sample データベースを作成する
CREATE DATABASE IF NOT EXISTS sample;

-- sample データベースを使用
USE sample;


-- もし empテーブルが存在していたら削除する。。
DROP TABLE IF EXISTS emp;


-- もし deptテーブルが存在したら削除する。
-- empテーブルが存在していたら削除できないので、
-- empテーブルを先に削除しなくてはならない。
DROP TABLE IF EXISTS dept;

-- dept テーブルの作成
CREATE TABLE IF NOT EXISTS dept (
  id   CHAR(3)     PRIMARY KEY,
  name VARCHAR(20) NOT NULL
);


-- emp テーブルの作成
CREATE TABLE IF NOT EXISTS emp (
  id       INT         PRIMARY KEY AUTO_INCREMENT,
  name     VARCHAR(20) NOT NULL,
  age      INT         NOT NULL,
  birthday YEAR        NOT NULL,
  dept_id  CHAR(3),
  FOREIGN KEY(dept_id) REFERENCES dept(id)
);


-- 自動連番を初期化する。
ALTER TABLE emp AUTO_INCREMENT = 1;

-- dept表の初期データ
INSERT INTO dept (id, name) VALUES
('001', '総務部'),
('002', '営業部'),
('003', '経理部'),
('004', '開発部');

-- emp表の初期データ
INSERT INTO emp (name, age, birthday, dept_id) VALUES
('菅原文太',   40, 1933, '001'),
('千葉真一',   34, 1939, '002') ,
('北大路欣也', 30, 1943, '003'),
('梶芽衣子',   26, 1947, '002');


SELECT * FROM dept;
SELECT * FROM emp;
\end{lstlisting}

このファイルを C:\yen Users\yen XXXXX\yen Documents\yen mysql に
作成する。

そのフォルダで コマンドプロンプトを起動し、sampleuserユーザーで mysql にログインする。

\begin{tcolorbox}
 $>$ mysql -u sampleuser -p \\
 Enter password: ********
 MariaDB [(none)]$>$
\end{tcolorbox}

今作成したファイルを 読み込む。

\begin{tcolorbox}
 MariaDB [(none)]$>$ source reinit\_data.sql
\end{tcolorbox}


\begin{spacing}{0.8}        
\begin{verbatim}
+-----+--------+
| id  | name   |
+-----+--------+
| 001 | 総務部 |
| 002 | 営業部 |
| 003 | 経理部 |
| 004 | 開発部 |
+-----+--------+
\end{verbatim}
\end{spacing}

\begin{spacing}{0.8}        
\begin{verbatim}
+----+------------+-----+----------+---------+
| id | name       | age | birthday | dept_id |
+----+------------+-----+----------+---------+
|  1 | 菅原文太   |  40 |     1933 | 001     |
|  2 | 千葉真一   |  34 |     1939 | 002     |
|  3 | 北大路欣也 |  30 |     1943 | 003     |
|  4 | 梶芽衣子   |  26 |     1947 | 002     |
+----+------------+-----+----------+---------+
\end{verbatim}
\end{spacing}

さて、この emp表に 以下のように dept\_id の項目に dept表にない値を指定して
データを入力してみる。

\begin{tcolorbox}
 MariaDB [sample]$>$ insert into emp (name, age, birthday, dept\_id) values \\
 \hspace{6mm} \verb!->! ('成田三樹夫', 38, 1935, '005');
\end{tcolorbox}

すると、次のようなエラーメッセージが出て、入力に失敗する。

\begin{quote}
\begin{verbatim}
ERROR 1452 (23000): Cannot add or update a child row:
a foreign key constraint fails (`sample`.`emp`, CONSTRAINT `emp_ibfk_1`
FOREIGN KEY (`dept_id`) REFERENCES `dept` (`id`))
\end{verbatim}
\end{quote}

dept表にある値にして入力する。

\begin{tcolorbox}
 MariaDB [sample]$>$ insert into emp (name, age, birthday, dept\_id) values \\
 \hspace{6mm} \verb!->! ('成田三樹夫', 38, 1935, '004');
\end{tcolorbox}

するとうまく入力できることがわかる。

\vspace{5mm}
※ もし dept表の id が、たとえば 営業部が 2 から 5 に変更になったとすると
どうなるか?

emp表の dept\_id も修正しなくてはならなくなる。

こんなときのために、外部キー制約のところに以下のような記述をすることができる。


\begin{tcolorbox}
 \textsf{
CREATE TABLE IF NOT EXISTS emp ( \\
\hspace{3mm}  id       INT         PRIMARY KEY AUTO\_INCREMENT, \\
\hspace{3mm}  name     VARCHAR(20) NOT NULL, \\
\hspace{3mm}  age      INT         NOT NULL, \\
\hspace{3mm}  birthday YEAR        NOT NULL, \\
\hspace{3mm}  dept\_id  CHAR(3), \\
\hspace{3mm}  FOREIGN KEY(dept\_id) REFERENCES dept(id) \\
\hspace{3mm}  \underline{ON DELETE SET NULL ON UPDATE CASCADE} \\
);}
 \end{tcolorbox}
\noindent
ON DELETE SET NULL --- 参照先を delete すると、参照元が null になる。\\
ON UPDATE CASCADE  --- 参照先を update すると、参照元も update される。
\footnote{
\href{https://qiita.com/SLEAZOIDS/items/d6fb9c2d131c3fdd1387}
{参考: https://qiita.com/SLEAZOIDS/items/d6fb9c2d131c3fdd1387}
}
しかし、dept表が頻繁に修正されるというのはあってほしくないことである。
その表を参照している表に大きな影響を与えることになるからである。


\subsubsection{参照している表を変更してみる}

\vspace{5mm}
この状態で dept表を変更してみる。

表の変更(更新)は、以下の構文を使うことでできる。

\begin{tcolorbox}
 \verb!UPDATE <テーブル名> SET <変更するカラム名> = <新しい値> WHERE <条件となるカラム> = <値>!
\end{tcolorbox}

たとえば、''営業部'' の ''002'' を ''005'' に変更してみる。

\begin{tcolorbox}
 MariaDB [sample]$>$ \textsf{update dept set id = '005' where id = '002';}
\end{tcolorbox}

\begin{tabular}{|l|} \hline
\verb!MariaDB [sample]> select * from dept;! \\ \hline
\end{tabular}

\begin{spacing}{0.8}        
\begin{quote}
 \begin{verbatim}
+-----+--------+
| id  | name   |
+-----+--------+
| 001 | 総務部 |
| 003 | 経理部 |
| 004 | 開発部 |
| 005 | 営業部 |
+-----+--------+
\end{verbatim}
\end{quote}
\end{spacing}

\begin{tabular}{|l|} \hline
\verb!MariaDB [sample]> select * from emp;! \\ \hline
\end{tabular}

\begin{spacing}{0.8}        
\begin{quote}
 \begin{verbatim}
+----+------------+-----+----------+---------+
| id | name       | age | birthday | dept_id |
+----+------------+-----+----------+---------+
|  1 | 菅原文太   |  40 |     1933 | 001     |
|  2 | 千葉真一   |  34 |     1939 | 005     |
|  3 | 北大路欣也 |  30 |     1943 | 003     |
|  4 | 梶芽衣子   |  26 |     1947 | 005     |
+----+------------+-----+----------+---------+
\end{verbatim}
\end{quote}
\end{spacing}

dept表の id が変更されたら、emp表の dept\_id も更新されているのがわかる。

これは、emp表を定義したときの \framebox[5cm][c]{ON UPDATE CASCADE} の働きによる。


\subsubsection{参照している表のデータを削除してみる}

今度は、参照している表のデータを削除してみる。削除は、以下の構文を使う。

\begin{tcolorbox}
 $>$ DELETE FROM <テーブル名> WHERE <削除カラム名> = <値>;
\end{tcolorbox}

dept表の id:'003' name:'経理部' を削除してみる。

\begin{tcolorbox}
 MariaDB [sample]$>$ delete from dept where id = '003';
\end{tcolorbox}

\begin{tabular}{|l|} \hline
\verb!MariaDB [sample]> select * from dept;! \\ \hline
\end{tabular}

\begin{spacing}{0.8}        
\begin{quote}
 \begin{verbatim}
+-----+--------+
| id  | name   |
+-----+--------+
| 001 | 総務部 |
| 004 | 開発部 |
| 005 | 営業部 |
+-----+--------+
\end{verbatim}
\end{quote}
\end{spacing}

\begin{tabular}{|l|} \hline
\verb!MariaDB [sample]> select * from emp;! \\ \hline
\end{tabular}

\begin{spacing}{0.8}        
\begin{quote}
 \begin{verbatim}
+----+------------+-----+----------+---------+
| id | name       | age | birthday | dept_id |
+----+------------+-----+----------+---------+
|  1 | 菅原文太   |  40 |     1933 | 001     |
|  2 | 千葉真一   |  34 |     1939 | 005     |
|  3 | 北大路欣也 |  30 |     1943 | NULL    |
|  4 | 梶芽衣子   |  26 |     1947 | 005     |
+----+------------+-----+----------+---------+
\end{verbatim}
\end{quote}
\end{spacing}

このように、dept表で削除されたデータを参照していた emp表の項目は ''NULL'' に
なっていることが確認できる。

これは emp表の定義の中の \framebox[5cm][c]{ON DELETE SET NULL} の働きによる。

\subsubsection{もっと厳しく制限をかける}

今までの制限は、緩い制限で、本来変更してはいけないデータの変更を許すものであった。

そこで、もっと厳しい制限をかけたほうがいい場合もある。

\begin{tcolorbox}
 \textsf{
CREATE TABLE IF NOT EXISTS emp ( \\
\hspace{3mm}  id       INT         PRIMARY KEY AUTO\_INCREMENT, \\
\hspace{3mm}  name     VARCHAR(20) NOT NULL, \\
\hspace{3mm}  age      INT         NOT NULL, \\
\hspace{3mm}  birthday YEAR        NOT NULL, \\
\hspace{3mm}  dept\_id  CHAR(3), \\
\hspace{3mm}  FOREIGN KEY(dept\_id) REFERENCES dept(id) \\
\hspace{3mm}  \underline{ON DELETE RESTRICT ON UPDATE RESTRICT} \\
);}
\end{tcolorbox}

\noindent
NO DELETE RESTRICT --- 参照している表(dept)のデータを削除するときエラーにする。\\
NO UPDATE RESTRICT --- 参照している表(dept)のデータを変更するときエラーにする。

ファイル ''reinit\_data.sql'' の emp表の定義部分を上記のように書き変えたのち、
''source reinit\_data.sql'' でファイル reinit\_data.sql を読み込む。

その後、以下のように dept表のデータを変更してみる。

\begin{tcolorbox}
 MariaDB [sample]$>$ update dept set id = '005' where id = '003';
\end{tcolorbox}

このようにエラーが出て、dept表のデータは変更できない。

\begin{quote}
 ERROR 1451 (23000): Cannot delete or update a parent row: a foreign key constraint fails 
 (`sample`.`emp`, CONSTRAINT `emp\_ibfk\_1` FOREIGN KEY (`dept\_id`) REFERENCES 
 `dept` (`id`))
\end{quote}

今度は dept表のデータを削除してみる。

\begin{tcolorbox}
 MariaDB [sample]$>$ delete from dept where id = '002';
\end{tcolorbox}

このように dept表のデータは削除もできなくなっている。

\begin{quote}
 ERROR 1451 (23000): Cannot delete or update a parent row: a foreign key constraint fails 
 (`sample`.`emp`, \underline{CONSTRAINT `emp\_ibfk\_1`} FOREIGN KEY (`dept\_id`) REFERENCES 
 `dept` (`id`)) 
\end{quote}

このように、参照している表は、削除したり変更したりできないほうが保守しやすい。
しかし、その時々で適切に対応するしかない。

\subsubsection{制約名}

ところで、エラーメッセージ中に \textsf{CONSTRAINT `emp\_ibfk\_1`} という部分があるが、
`emp\_ibfk\_1` は、MySQLが勝手につけたこの制約の名前である。

制約には名前をつけることができる。
名前をつけておくと、エラーが出たときに、どの部分の制約か判別しやすい。

今回の emp表定義中の制約に名前をつけてみる。
ファイル ''reinit\_data.sql'' の emp表定義の部分を以下のように修正する。

\begin{tcolorbox}
 \textsf{
 CREATE TABLE IF NOT EXISTS emp ( \\
 \hspace{3mm}  id       INT         PRIMARY KEY AUTO\_INCREMENT, \\
 \hspace{3mm}  name     VARCHAR(20) NOT NULL, \\
 \hspace{3mm}  age      INT         NOT NULL, \\
 \hspace{3mm}  birthday YEAR        NOT NULL, \\
 \hspace{3mm}  dept\_id  CHAR(3), \\
 \hspace{3mm}  \underline{CONSTRAINT fk\_dept\_id} \\
 \hspace{3mm}  FOREIGN KEY(dept\_id) REFERENCES dept(id) \\
 \hspace{3mm}  ON DELETE RESTRICT ON UPDATE RESTRICT \\
 );}
\end{tcolorbox}
\rightline{※ CONSTRAINT は ''制約'' という意味。}

修正したら、\framebox[5cm][c]{$>$ source reinit\_data.sql} とする。

そののち、dept表のひとつのデータを削除してみる。

\begin{tcolorbox}
 MariaDB [samle]$>$ delete from dept where id = '002';
\end{tcolorbox}

以下のように 制約名 ''fk\_dept\_id'' が出力されている。

\begin{quote}
ERROR 1451 (23000): Cannot delete or update a parent row: a foreign key constraint fails (`sample`.`emp`, CONSTRAINT `fk\_dept\_id` FOREIGN KEY (`dept\_id`) REFERENCES `dept` (`id`))
\end{quote}




 






\end{document}

%% 修正時刻: Sat May  2 15:10:04 2020


%% 修正時刻: Fri Oct  1 19:33:56 2021
