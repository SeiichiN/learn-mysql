\documentclass[dvipdfmx]{jsarticle}

\include{begin}

\section{2つのテーブルを結合する}

\subsection{内部結合(JOIN句)}

empテーブルの dept\_id は、deptテーブルの id である。

だから、dept\_idをキーにして、二つのテーブルを結合できる。

結合するには \textsf{JOIN}句を使う。

実行例

\begin{tcolorbox}
 MariaDB [sample]$>$ select * from emp \underline{\textsf{join}} dept
 \underline{\textsf{on}} emp.dept\_id = dept.id;
\end{tcolorbox}

これで二つのテーブルが結合される。

\begin{spacing}{0.8}        
\begin{verbatim}
+----+------------+-----+----------+---------+-----+--------+
| id | name       | age | birthday | dept_id | id  | name   |
+----+------------+-----+----------+---------+-----+--------+
|  1 | 菅原文太   |  40 |     1933 | 001     | 001 | 総務部 |
|  2 | 千葉真一   |  34 |     1939 | 002     | 002 | 営業部 |
|  4 | 梶芽衣子   |  26 |     1947 | 002     | 002 | 営業部 |
|  3 | 北大路欣也 |  30 |     1943 | 003     | 003 | 経理部 |
+----+------------+-----+----------+---------+-----+--------+
\end{verbatim}
\end{spacing}

\textsf{join} は \textsf{inner join} と記述できる。
``内部結合''と呼ばれている。

\textsf{on emp.dept\_id = dept.id} は 
emp の dept\_id と dept の id が等しければ、そのレコードを抜き出す。

\rightline{※ emp.dept\_id は empテーブルの dept\_id という意味になる。}

\subsection{表示項目を絞る}

現在は項目を全て表示しているが、これを変更する。

emp表の id, name, age と dept表の name だけを表示させる。


\begin{tcolorbox}
 MariaDB [sample]$>$ select \underline{emp.id, emp.name, age, dept.name}
 from emp join dept \\
 \hspace{6mm} \verb!->! on emp.dept\_id = dept.id;
\end{tcolorbox}

\begin{spacing}{0.8}        
\begin{verbatim}
+----+------------+-----+--------+
| id | name       | age | name   |
+----+------------+-----+--------+
|  1 | 菅原文太   |  40 | 総務部 |
|  2 | 千葉真一   |  34 | 営業部 |
|  4 | 梶芽衣子   |  26 | 営業部 |
|  3 | 北大路欣也 |  30 | 経理部 |
+----+------------+-----+--------+
\end{verbatim}
\end{spacing}        

このように必要な項目のみ表示させることができる。
ただ、name という項目が二つあったり、英語であったりするので、
これを適切な日本語に変える。

それには、\textsf{as句} というのが使える。

たとえば、``emp.name as 名前'' とすると、``emp.neme'' は ``名前'' と表示される。

\begin{tcolorbox}
 MariaDB [sample]$>$ select \underline{emp.id as ID},
 \underline{emp.name as 名前}, \underline{age as 年齢},
 \underline{dept.name as 部署名} \\
 \hspace{6mm} \verb!->! from emp join dept \\
 \hspace{6mm} \verb!->! on emp.dept\_id = dept.id;
\end{tcolorbox}

\begin{spacing}{0.8}        
\begin{verbatim}
+----+------------+------+--------+
| ID | 名前       | 年齢 | 部署名 |
+----+------------+------+--------+
|  1 | 菅原文太   |   40 | 総務部 |
|  2 | 千葉真一   |   34 | 営業部 |
|  4 | 梶芽衣子   |   26 | 営業部 |
|  3 | 北大路欣也 |   30 | 経理部 |
+----+------------+------+--------+
\end{verbatim}
\end{spacing}

さらによく見てみると、この表は部署名の順に並んでいる。
これを ID順に並びかえる。

そのためには \textsf{order句} というのが使える。

たとえば、今回の場合だと、\fbox{\textsf{order by emp.id [asc]}} とすることで、ID順になる。

\textsf{asc} というのは''昇順''という意味で、省略すると asc と指定したことになる。

また、\textsf{desc} と指定すると ''降順'' で並びかえできる。


\begin{tcolorbox}
 MariaDB [sample]$>$ select emp.id as ID, emp.name as 名前, age as 年齢,
 dept.name as 部署名 \\
 \hspace{6mm} \verb!->! from emp join dept \\
 \hspace{6mm} \verb!->! on emp.dept\_id = dept.id \\
 \hspace{6mm} \verb!->! \underline{order by ID};
\end{tcolorbox}

\begin{spacing}{0.8}        
\begin{verbatim}
+----+------------+------+--------+
| ID | 名前       | 年齢 | 部署名 |
+----+------------+------+--------+
|  1 | 菅原文太   |   40 | 総務部 |
|  2 | 千葉真一   |   34 | 営業部 |
|  3 | 北大路欣也 |   30 | 経理部 |
|  4 | 梶芽衣子   |   26 | 営業部 |
+----+------------+------+--------+
\end{verbatim}
\end{spacing}

\textsf{order by emp.id} とするところを \textsf{order by ID} としている。

これは、1行目で \textsf{emp.id as ID} としているので、ID を使うことができるのである。

\newpage
\subsection{外部結合(left outer join / right outer join)}

\subsubsection{左外部結合 left outer join}

この emp表に次のデータを追加する。

\begin{tcolorbox}
 \begin{tabular}{lcl}
  ID & : & 5 \\
  名前 & : & 成田三樹夫 \\
  年齢 & : &  38 \\
  誕生年 & : & 1935 \\
  部署ID & : & (なし) \\
 \end{tabular}
\end{tcolorbox}

\begin{tcolorbox}
 MariaDB [sample]$>$ \textsf{insert into emp (name, age, birthday) values} \\
 \hspace{6mm} \verb!->! \textsf{('成田三樹夫', 38, 1935);}
\end{tcolorbox}

\begin{spacing}{0.8}        
\begin{verbatim}
MariaDB [sample]> select * from emp;
+----+------------+-----+----------+---------+
| id | name       | age | birthday | dept_id |
+----+------------+-----+----------+---------+
|  1 | 菅原文太   |  40 |     1933 | 001     |
|  2 | 千葉真一   |  34 |     1939 | 002     |
|  3 | 北大路欣也 |  30 |     1943 | 003     |
|  4 | 梶芽衣子   |  26 |     1947 | 002     |
|  5 | 成田三樹夫 |  38 |     1935 | NULL    |
+----+------------+-----+----------+---------+    
\end{verbatim}
\end{spacing}


このデータには dept\_id、つまり部署ID がない。たとえば社長とかの場合である。

この状態で 内部結合 をすると、どうなるか?

\begin{verbatim}
MariaDB [sample]> select emp.id as ID, emp.name as 名前, age as 年齢,
    -> dept.name as 部署名 from emp join dept
    -> on emp.dept_id = dept.id
    -> order by ID;
\end{verbatim}

\begin{spacing}{0.8}        
\begin{verbatim}
+----+------------+------+--------+
| ID | 名前       | 年齢 | 部署名 |
+----+------------+------+--------+
|  1 | 菅原文太   |   40 | 総務部 |
|  2 | 千葉真一   |   34 | 営業部 |
|  3 | 北大路欣也 |   30 | 経理部 |
|  4 | 梶芽衣子   |   26 | 営業部 |
+----+------------+------+--------+    
\end{verbatim}
\end{spacing}

結合表には出てこない。

これを図であらわすと、このようになる。

\vspace{3mm}
\includegraphics[width=13cm]{../06-mysql/ketsugo.png}
\vspace{3mm}

成田三樹夫は部署IDがないので結合の対象ではない。

こんなときは ''左外部結合(left outer join)'' を使う。

\begin{tcolorbox}
 MariaDB [sample]$>$ select emp.id as ID, emp.name as 名前, age as 年齢, \\
 \hspace{6mm} \verb!->! dept.name as 部署名 from emp \underline{\textsf{left join}} dept \\
 \hspace{6mm} \verb!->! \textsf{on} emp.dept\_id = dept.id \\
 \hspace{6mm} \verb!->! order by ID;
\end{tcolorbox}

\begin{spacing}{0.8}        
\begin{verbatim}
+----+------------+------+--------+
| ID | 名前       | 年齢 | 部署名 |
+----+------------+------+--------+
|  1 | 菅原文太   |   40 | 総務部 |
|  2 | 千葉真一   |   34 | 営業部 |
|  3 | 北大路欣也 |   30 | 経理部 |
|  4 | 梶芽衣子   |   26 | 営業部 |
|  6 | 成田三樹夫 |   38 | NULL   |
+----+------------+------+--------+
\end{verbatim}
\end{spacing}

\subsubsection{右外部結合 right outer join}

また、dept表をみてみると、\textsf{id : '004'} が  \textsf{開発部} であるが、
emp表には dept\_id が '004' である人はいない。

この状態で結合表をつくり、開発部という項目も表示させるには、次のようにする。

\begin{tcolorbox}
 MariaDB [sample]$>$ select emp.id as ID, emp.name as 名前, age as 年齢, \\
 \hspace{6mm} \verb!->! dept.name as 部署名 from emp \underline{\textsf{right join}} dept \\
 \hspace{6mm} \verb!->! \textsf{on} emp.dept\_id = dept.id \\
 \hspace{6mm} \verb!->! order by ID;
\end{tcolorbox}

\begin{spacing}{0.8}        
\begin{verbatim}
+------+------------+------+--------+
| ID   | 名前       | 年齢 | 部署名 |
+------+------------+------+--------+
| NULL | NULL       | NULL | 開発部 |
|    1 | 菅原文太   |   40 | 総務部 |
|    2 | 千葉真一   |   34 | 営業部 |
|    3 | 北大路欣也 |   30 | 経理部 |
|    4 | 梶芽衣子   |   26 | 営業部 |
+------+------------+------+--------+
\end{verbatim}
\end{spacing}



\include{end}

%% 修正時刻: Fri Oct  1 19:11:24 2021
