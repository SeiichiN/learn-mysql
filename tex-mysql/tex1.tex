\documentclass[dvipdfmx]{jsarticle}

\include{begin}

\section{データベースを設計する}

\subsection{扱うデータ}

以下のようなデータを扱うこととする。

\vspace{3mm}
 \begin{tabular}{|c|} \hline
  菅原文太 \\
  40歳 \\
  1933年生まれ \\
  総務部 \\ \hline
 \end{tabular}
 \quad
 \begin{tabular}{|c|} \hline
  千葉真一 \\
  34歳 \\
  1939年生まれ \\
  営業部 \\ \hline
 \end{tabular}
 \quad
 \begin{tabular}{|c|} \hline
  北大路欣也 \\
  30歳 \\
  1943年生まれ \\
  経理部 \\ \hline
 \end{tabular}
 \quad
 \begin{tabular}{|c|} \hline
  梶芽衣子 \\
  26歳 \\
  1947年生まれ \\
  営業部 \\ \hline
 \end{tabular}
\vspace{3mm}

あなたがプログラマで、上のような社員名簿アプリを作成することになったとする。
PHP か Java でアプリを作成することになる。クライアントの会社の総務部がこの
アプリを使うことになる。そのアプリには社員の登録画面、一覧画面、編集画面、
削除画面などがあるだろう。そういった画面と処理をあなたは作らなければならない。

そのときに、データを保存するしくみとして、データベースを使うことになる。
かりに PHP でプログラミングするならば、PHPという言語を使って
データベースを操作することになる。

\subsection{primary key}

データベースにデータを格納する際には、そのデータに primary key (独自キー) が必要となる。
primary key とは、そのデータを他と区別するためのデータである。
菅原文太というデータは、この4つの中では独自であるが、他のデータを追加する際に、同じデータに出会う可能性
(同姓同名)を排除できない。
さらに日本語である以上、文字コードの問題を避けることもできない。つまり、同じ菅原文太という文字でも
UTF-8 と Shift\_JIS では別物と判定されるのである。

となると、この4つのデータには primary key となるものがないということになる。

このような場合、データベースの設計者が primary key を追加することになる。
ここでは 数字を primary key として追加する。
つまり、菅原文太は 1、千葉真一は 2 というふうにする。

\subsection{表を分ける}

4つの各データには、総務部、営業部、経理部 という部署名がはいっている。
これらは、部署データとして、別の表で管理するのが自然である。
そして、各人のデータから部署データを参照しているというふうにするのがよい。

\vspace{3mm}
XX会社 部署一覧 \\
\hspace{5mm}
\begin{tabular}{|c|} \hline
 総務部 \\
 営業部 \\
 経理部 \\
 開発部 \\ \hline
\end{tabular}\vspace{3mm}


\newpage
\section{データベースを作成する}

\subsection{ユーザーを作成して、データベースを作成する}

\subsubsection{MySQLの起動}

まず、MySQLを起動しなくてはならない。
\begin{enumerate}
 \item XAMPP コントロールパネルを管理者として起動する。
 \item MySQL の行の Start ボタンをクリックして MySQLを起動する。
\end{enumerate}

\vspace{3mm}
\includegraphics[width=10cm]{../06-mysql/02-mysql-start.png}
\vspace{3mm}


\subsubsection{rootユーザーでログインする}

データベースを作成するために、まずそのデータベースを扱うことのできるユーザーを作成する。

ユーザーを作成するために、まず管理者(root) でログインする。
MariaDBの場合、以下の手順でログインできる。

コマンドプロンプトを起動して、以下のコマンドを入力する。

\begin{tcolorbox}
 $>$ mysql -u root -p (Enterキー)\\
 $>$ Enter password: (何も入力せず、Enterキー)
\end{tcolorbox}

これで、4行ほどのメッセージと、次のプロンプトが表示される。

\begin{tcolorbox}
 MariaDB [(none)]$>$
\end{tcolorbox}

\newpage
\subsubsection{ユーザーの作成と権限の付与}

以下のコマンドで \textsf{sampleuser} を作成し、\textsf{sampleuser} というパスワードを設定し、
\textsf{sample} データベースへの権限を与えることができる。
\footnote{ユーザーの作成と権限付与を別々にすることもできる。\\
sampleuser というユーザーの作成と sampleuser というパスワードの設定\\
MariaDB [(note)]$>$ CREATE USER 'sampleuser'@'localhost' IDENTIFIED BY 'sampleuser'; \\
sampleuser に sample データベーYスへの権限を付与する。\\
MariaDB [(none)]$>$ GRANT ALL [PRIVILEGES] ON sample.* TO 'sampleuser'@'localhost';
}
ここでは、ユーザー名を sampleuser、パスワードを sampleuser としている。

\begin{tcolorbox}
MariaDB [(note)]$>$ GRANT \ ALL \ ON \ sample.* \ TO \ 'sampleuser'@'localhost' \\
\hspace{20mm} -$>$ IDENTIFIED \ BY \ 'sampleuser';
\end{tcolorbox}

\fbox{GRANT \ ALL \ ON \ データベース名.* \ TO \ 'ユーザー名'@'localhost'
 \ IDENTIFIED \ BY \ 'パスワード';}
\vspace{3mm}


\textsf{sample} というデータベースを作成すると、いくつかファイルを作成することになるので、
それら全部に権限を与えるため、\textsf{sample.*} としている。

\rightline{※ \textsf{sample.(ドット)*(アスタリスク)}}


これで root としての仕事は終了である。\textsf{exit} あるいは \textsf{quit} でログアウトする。

\begin{tcolorbox}
 MariaDB [(none)]$>$ \textsf{exit}
\end{tcolorbox}

\subsubsection{作成したユーザーでログインし、データベースを作成する}

作成したユーザー \textsf{sampleuser} でログインする。

\begin{tcolorbox}
 $>$ mysql -u sampleuser -p (Enterキー)\\
 $>$ Enter password: ********** (sampleuser と入力) \\
 (... 省略 ...) \\
 MariaDB [(none)]$>$ 
\end{tcolorbox}

\fbox{mysql \ -u \ ユーザー名 \ -p} \\
Enter password: \fbox{パスワード}
\vspace{3mm}

データベース \textsf{sample} を作成する。

\begin{tcolorbox}
 MariaDB [(none)]$>$ \textsf{create database sample;}
\end{tcolorbox}

\fbox{CREATE \ DATABASE \ データベース名 \,;}

これで、この作成したデータベース sample は、sampleuserユーザーでアクセスできる。
(もちろん root ユーザーもアクセスできる)







\include{end}

%% 修正時刻: Sat Oct  9 19:19:59 2021
