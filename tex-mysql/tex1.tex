\documentclass[dvipdfmx]{jsarticle}

\include{begin}

\section{データベースを作成する}

\subsection{ユーザーを作成して、データベースを作成する}

\subsubsection{MySQLの起動}

まず、MySQLを起動しなくてはならない。
\begin{enumerate}
 \item XAMPP コントロールパネルを管理者として起動する。
 \item MySQL の行の Start ボタンをクリックして MySQLを起動する。
\end{enumerate}

\vspace{3mm}
\includegraphics[width=10cm]{../06-mysql/02-mysql-start.png}
\vspace{3mm}


\subsubsection{rootユーザーでログインする}

データベースを作成するために、まずそのデータベースを扱うことのできるユーザーを作成する。

ユーザーを作成するために、まず管理者(root) でログインする。
MariaDBの場合、以下の手順でログインできる。

コマンドプロンプトを起動して、以下のコマンドを入力する。

\begin{tcolorbox}
 $>$ mysql -u root -p (Enterキー)\\
 $>$ Enter password: (何も入力せず、Enterキー)
\end{tcolorbox}

これで、4行ほどのメッセージと、次のプロンプトが表示される。

\begin{tcolorbox}
 MariaDB [(none)]$>$
\end{tcolorbox}

\newpage
\subsubsection{ユーザーの作成と権限の付与}

以下のコマンドで \textsf{sampleuser} というユーザーを作成する。
パスワードも \textsf{sampleuser} としておく。(今回は練習のため)

\begin{tcolorbox}
 MariaDB [(none)]$>$ \textsf{create user 'sampleuser'@'localhost' identified by 'sampleuser';}
\end{tcolorbox}
\rightline{※ 末尾の ; (セミコロン) を忘れないように}

次に以下のコマンドで \textsf{sampleuser} に \textsf{sample} データベースへの権限を
付与しておく。

\begin{tcolorbox}
 MariaDB [(none)]$>$ \textsf{grant all privileges on sample.* to 'sampleuser'@'localhost';}
\end{tcolorbox}
\footnote{ユーザーの作成と権限付与を同時にすることもできる。\\
MariaDB [(note)]$>$ \textsf{grant all on sample.* to 'sampleuser'@'localhost' identified by 'sampleuser';}}

\textsf{sample} というデータベースを作成すると、いくつかファイルを作成することになるので、
それら全部に権限を与えるため、\textsf{sample.*} としている。

\rightline{※ \textsf{sample.(ドット)*(アスタリスク)}}


これで root としての仕事は終了である。\textsf{exit} あるいは \textsf{quit} でログアウトする。

\begin{tcolorbox}
 MariaDB [(none)]$>$ \textsf{exit}
\end{tcolorbox}

\subsubsection{作成したユーザーでログインし、データベースを作成する}

作成したユーザー \textsf{sampleuser} でログインする。

\begin{tcolorbox}
 $>$ mysql -u sampleuser -p (Enterキー)\\
 $>$ Enter password: ********** (sampleuser と入力) \\
 (... 省略 ...) \\
 MariaDB [(none)]$>$ 
\end{tcolorbox}

データベース \textsf{sample} を作成する。

\begin{tcolorbox}
 MariaDB [(none)]$>$ \textsf{create database sample;}
\end{tcolorbox}

これで、この作成したデータベース sample は、sampleuserユーザーでアクセスできる。
(もちろん root ユーザーもアクセスできる)







\include{end}

%% 修正時刻: Tue Aug 10 06:53:35 2021
