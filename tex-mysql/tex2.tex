\documentclass[dvipdfmx]{jsarticle}

\include{begin}

\section{テーブル(表)を作成する}

\subsection{作成する表のイメージ}

以下のような表を作成することとする。

\begin{table}[h]
 \caption{emp}
 \begin{center}
  \begin{tabular}[h]{|c|l|c|c|c|}
   \hline
   ID & 名前       & 年齢 & 誕生年 & 部署ID \\ \hline\hline
   1  & 菅原文太   & 40   & 1933   & 001    \\ \hline
   2  & 千葉真一   & 34   & 1939   & 002    \\ \hline
   3  & 北大路欣也 & 30   & 1943   & 003    \\ \hline
   4  & 梶芽衣子   & 26   & 1947   & 002    \\ \hline
  \end{tabular}
 \end{center}
\end{table}

\begin{table}[h]
 \caption{dept}
 \begin{center}
  \begin{tabular}{|l|c|} \hline
   ID   & 部署名 \\ \hline\hline
   001  & 総務部 \\ \hline
   002  & 営業部 \\ \hline
   003  & 経理部 \\ \hline
   004  & 開発部 \\ \hline
  \end{tabular}
 \end{center}
\end{table}

そして、上の2つの表から、以下の結合表を表示することとする。

\begin{table}[h]
 \begin{center}
  \begin{tabular}[h]{|c|l|c|c|c|}
   \hline
   ID & 名前       & 年齢  & 部署名 \\ \hline\hline
   1  & 菅原文太   & 40    & 総務部    \\ \hline
   2  & 千葉真一   & 34    & 営業部    \\ \hline
   3  & 北大路欣也 & 30    & 経理部    \\ \hline
   4  & 梶芽衣子   & 26    & 営業部    \\ \hline
  \end{tabular}
 \end{center}
\end{table}
 
\newpage
 
\subsection{テーブルの定義}

テーブルの定義を決める。

\begin{table}[ht]
 \caption{empテーブルの定義}
 \begin{center}
  \begin{tabular}{|l|l|l|} \hline
   項目 & 型 & オプション \\ \hline
   id   & int & primary key auto\_increment \\ 
   name & varchar(20) & not null \\ 
   age  & int & not null \\ 
   birthday & year & not null \\ 
   dept\_id & char(3) & \\ \hline
  \end{tabular}
 \end{center}
\end{table}

\textgt{型}

\begin{tabular}{ll}
 int型 & 整数。これがよく使われる。 \\
 varchar型 & 可変長の文字列型。ここでは最大20文字としている。(全角文字を使った場合) \\
 year型 & 年のみを扱う型。誕生の年だけを入力する。 \\
 char型 & 固定長の文字型。ここで半角で3文字としている。
\end{tabular}

\vspace{3mm}
\textgt{オプション}

\begin{tabular}{ll}
 primary key & 項目 id をデータの識別に使う。重複する値がないことが保証される。 \\
 auto\_increment & 自動連番。自動的に順に番号を振ってくれる機能を使う。 \\
 not null & 入力が必須。もしも入力しなかったら エラー になる。 \\
\end{tabular}

\vspace{3mm}
最後の dept\_id を not null にしなかったのは、部署ID のない社員もいるかもしれないからである。

\vspace{3mm}
deptテーブルは、このような定義になる。
 
\begin{table}[h]
 \caption{deptテーブルの定義}
 \begin{center}
  \begin{tabular}{|l|l|l|} \hline
   項目 & 型 & オプション \\ \hline
   id   & char(3) & primary key \\ 
   name & varchar(20) & not null \\ \hline
  \end{tabular}
 \end{center}
\end{table}

今回の場合、empテーブルには dept\_id が入っている。これは、deptテーブルの id のことである。
このことにより、empテーブルとdeptテーブルを結合させることができる。

このときの empテーブルの dept\_id のことを ``\textgt{外部キー}'' という。


\subsection{テーブルの作成}

テーブルを作成する前に、データベースの使用を宣言する。

\begin{tcolorbox}
 MariaDB [(none)]$>$ use sample; 
\end{tcolorbox}

以下のコマンドにより empテーブルを作成できる。

\begin{tcolorbox}
 MariaDB [sample]$>$ create \ table \ emp \ ( \\
 \hspace{6mm} \verb!->! id \ int \ auto\_increment\,, (カンマ) \\
 \hspace{6mm} \verb!->! name \ varchar(20) \ not \ null\,, \\
 \hspace{6mm} \verb!->! age \ int \ not \ null\,, \\
 \hspace{6mm} \verb!->! birthday \ year \ not \ null\,, \\
 \hspace{6mm} \verb!->! dept\_id \ char(3), \\
 \hspace{6mm} \verb!->! primary \ key\  (id) (カンマなし) \\
 \hspace{6mm} \verb!->! );
\end{tcolorbox}

同様に deptテーブルも作成する。
\footnote{primary key の指定は、id の指定のときに書くこともできるし、別項目に分けて書くこともできる。}

\begin{tcolorbox}
 MariaDB [sample]$>$ create \ table \ dept \ ( \\
 \hspace{6mm} \verb!->! id \ char(3) \ primary \ key\,, (カンマ) \\
 \hspace{6mm} \verb!->! name \ varchar(20) \ not \ null (カンマなし) \\
 \hspace{6mm} \verb!->! );
\end{tcolorbox}

\vspace{3mm}
\noindent ※
\quad 作成したテーブルの構造は以下のコマンドで確認できる。 \\
MariaDB [sample]$>$ \textsf{desc emp;}
\begin{verbatim}
+----------+-------------+------+-----+---------+----------------+
| Field    | Type        | Null | Key | Default | Extra          |
+----------+-------------+------+-----+---------+----------------+
| id       | int(11)     | NO   | PRI | NULL    | auto_increment |
| name     | varchar(20) | NO   |     | NULL    |                |
| age      | int(11)     | NO   |     | NULL    |                |
| birthday | year(4)     | NO   |     | NULL    |                |
| dept_id  | char(3)     | YES  |     | NULL    |                |
+----------+-------------+------+-----+---------+----------------+
\end{verbatim}

また、テーブルを作成したときのコマンドは以下で確認できる。\\
MariaDB [sample]$>$ \textsf{show create table emp;}
\begin{verbatim}
...(省略)... 
CREATE TABLE `emp` ( 
  `id` int(11) NOT NULL AUTO\_INCREMENT,
  `name` varchar(20) NOT NULL,
  `age` int(11) NOT NULL,
  `birthday` year(4) NOT NULL,
  `dept\_id` char(3) DEFAULT NULL,
  PRIMARY KEY (`id`)
) ENGINE=InnoDB DEFAULT CHARSET=utf8mb4
\end{verbatim}

\subsection{データの登録}

データの登録は、以下のコマンドでできる。

\begin{tcolorbox}
 MariaDB [sample]$>$ \textsf{insert into emp (name, age, birthday, dept\_id) values ('菅原文太', 40, 1933, '001');}
\end{tcolorbox}
\rightline{※ 各データの区切りは \textsf{,}(カンマ)}
\noindent
''id'' は auto\_increment なので、指定しない。\\
また、dept\_id は char(3) なので、\textsf{'001'} シングルクォーテーションを使って入力する。

画面の関係で一行で入力しづらければ、次のように二行で入力することもできる。

\begin{tcolorbox}
 MariaDB [sample]$>$ \textsf{insert into emp (name, age, birthday, dept\_id)} \\
 \hspace{6mm} \verb!->! \textsf{values ('千葉真一', 34, 1939, '002');}
\end{tcolorbox}

TeraPad などのエディタで記述しておいて、コピー\&貼り付け ですることもできる。

以下のようにすると、一度で入力できてしまう。

\begin{tcolorbox}
 MariaDB [sample]$>$ \textsf{insert into emp (name, age, birthday, dept\_id) values} \\
 \hspace{6mm} \verb!->! \textsf{('北大路欣也', 30, 1943, '003'),}   (カンマ) \\
 \hspace{6mm} \verb!->! \textsf{('梶芽衣子', 26, 1947, '002');}
\end{tcolorbox}

これも、エディタに記述しておいて、コピー\&貼り付けでも できる。

データの確認は次のコマンドでできる。

\begin{tcolorbox}
 MariaDB [sample]$>$ \textsf{select * from emp;}
\end{tcolorbox}

\begin{verbatim}
+----+------------+-----+----------+---------+
| id | name       | age | birthday | dept_id |
+----+------------+-----+----------+---------+
|  1 | 菅原文太   |  40 |     1933 | 001     |
|  2 | 千葉真一   |  34 |     1939 | 002     |
|  3 | 北大路欣也 |  30 |     1943 | 003     |
|  4 | 梶芽衣子   |  26 |     1947 | 002     |
+----+------------+-----+----------+---------+
\end{verbatim}

\subsection{ファイル読込みによるデータの登録}

同様に、deptテーブルについてもデータを登録する。

ただ、今度は 登録のための SQL文を外部ファイルに記述しておいて、
それを読み込むという方法で登録してみる。

\subsubsection{作業のためのフォルダを用意して、そこでファイルをつくる。}

ファイルを置くためのフォルダを用意する。
ここでは仮に、ドキュメントフォルダに mysql というフォルダを作成したとする。

そこに、以下の内容のファイル ''insert\_dept.sql'' を作成する。

\begin{lstlisting}[caption=insert\_dept.sql]
-- dept テーブル

INSERT INTO dept (id, name) VALUES
('001', '総務部'),
('002', '営業部'),
('003', '経理部'),
('004', '開発部');
\end{lstlisting}

{-}{-} で始まる行は、コメントである。
\footnote{ほかに、\# で始まる行もコメント。また複数行は、/* ... */ が使える。}

また、SQLのコマンドは大文字で記述したほうがよい。
コマンドプロンプトでは、小文字でかまわないが、このようにファイルとして記述する場合は、
SQLのコマンド文字列は大文字で記述し、ユーザーが用意した変数などは小文字で記述しておく。
あとで見なおしたりする場合にわかりやすい。


\subsubsection{そのフォルダでコマンドプロンプトを起動する。}

そのフォルダでコマンドプロンプトを起動する。
次の図のようにする。

\vspace{3mm}
\includegraphics[width=10cm]{../06-mysql/03-cmd.png}
\vspace{3mm}

上の図のように、エクスプローラのアドレス欄の余白部分をクリックする。
すると、現在のフォルダをあらわす文字列が青く反転する。

\vspace{3mm}
\includegraphics[width=10cm]{../06-mysql/04-cmd.png}
\vspace{3mm}

上の図のように、そこに ''\textsf{cmd}'' と入力して Enterキーを押下する。
すると、そのフォルダでコマンドプロンプトが起動する。

\vspace{3mm}
\includegraphics[width=11cm]{../06-mysql/05-cmd.png}
\vspace{3mm}

上の図の赤い線の部分が、現在のフォルダになっている。

\begin{tcolorbox}
 C:\yen Users\yen (ユーザー名)\yen Documents\yen mysql$>$ \textsf{dir}
\end{tcolorbox}

\textsf{dir} というコマンドを実行すると、現フォルダのファイルが一覧できる。
\textsf{insert\_dept.sql} があることがわかる。

\begin{verbatim}
 2021/08/09  22:05   <DIR>        .                (このフォルダ)
 2021/08/09  22:05   <DIR>        ..               (ひとつ上の階層)
 2021/08/09  22:05            222 dept.sql  
\end{verbatim}

\newpage

\subsubsection{ファイルを読み込んで、SQL文を実行する}

ここで、mysql を起動する。

\begin{tcolorbox}
 C:\yen Users\yen (ユーザー名)\yen Documents\yen mysql$>$ \textsf{mysql -u sampleuser -p} \\
 \textsf{Enter password: **********}
\end{tcolorbox}

\textsf{sample}データベースの使用を宣言する。

\begin{tcolorbox}
 MariaDB [(none)]$>$ \textsf{use sample;} \\
 MariaDB [sample]$>$
\end{tcolorbox}

次に、以下のコマンドで \textsf{insert\_dept.sql} を実行できる。

\begin{tcolorbox}
 MariaDB [sample]$>$ \textsf{source insert\_dept.sql}
\end{tcolorbox}

あるいは、以下のような省略形もある。

\begin{tcolorbox}
 MariaDB [sample]$>$ \textsf{\yen . insert\_dept.sql}
\end{tcolorbox}

確認する。

\begin{tcolorbox}
 MariaDB [sample]$>$ \textsf{select * from dept;}
\end{tcolorbox}

\begin{verbatim}
+-----+--------+
| id  | name   |
+-----+--------+
| 001 | 総務部 |
| 002 | 営業部 |
| 003 | 経理部 |
| 004 | 開発部 |
+-----+--------+
\end{verbatim}

読み込めているのがわかる。


\subsection{テーブル作成からデータの登録までを自動化する}

このことを応用して、テーブルの作成からデータの登録までを、ファイル読込みによって
自動化することができる。

以下のような記述が考えられる。

\begin{lstlisting}[caption=init\_data.sql]
 -- もし存在していなかったら sample データベースを作成する
 CREATE DATABASE IF NOT EXISTS sample;

 -- sample データベースを使用
 USE sample;
 
 -- emp テーブルの作成
 -- もし empテーブルが存在しなかったら作成する。
 -- もし存在したら、このSQL文は実行されない。
 
 CREATE TABLE IF NOT EXISTS emp (
   id INT PRIMARY KEY AUTO_INCREMENT,
   name VARCHAR(20) NOT NULL,
   age INT NOT NULL,
   birthday YEAR NOT NULL,
   dept_id CHAR(3)
 );

 -- dept テーブルの作成
 -- もし deptテーブルが存在しなかったら作成する。
 -- もし存在したら、このSQL文は実行されない。
 
 CREATE TABLE IF NOT EXISTS dept (
   id CHAR(3) PRIMARY KEY,
   name VARCHAR(20) NOT NULL
 );

 -- もし、データが存在していたら、削除する。
 DELETE FROM emp WHERE (SELECT COUNT(id) FROM emp) > 0;

 -- 自動連番を初期化する。
 ALTER TABLE emp AUTO_INCREMENT = 1;
 
 INSERT INTO emp (name, age, birthday, dept_id) VALUES
 ('菅原文太', 40, 1933, '001'),
 ('千葉真一', 34, 1939, '002') ,
 ('北大路欣也', 30, 1943, '003'),
 ('梶芽衣子', 26, 1947, '002');

 -- もし、データが存在していたら、削除する。
 DELETE FROM dept WHERE (SELECT COUNT(id) FROM dept) > 0;
 
 INSERT INTO dept (id, name) VALUES
 ('001', '総務部'),
 ('002', '営業部'),
 ('003', '経理部'),
 ('004', '開発部');

 SELECT * FROM emp;
 SELECT * FROM dept;
\end{lstlisting}

これを実行する。

\begin{tcolorbox}
 MariaDB [sample]$>$ \yen . init\_data.sql
\end{tcolorbox}

これにより、いつでもデータを初期状態に戻すことができるようになった。





\include{end}

%% 修正時刻: Fri Oct  1 09:27:52 2021
