\documentclass[dvipdfmx]{jsarticle}

\include{begin}

\section{IISの動作を止める}

XAMPPのコントロールパネルでApacheを起動しようと Start ボタンをクリック
しても、Apacheが起動しないことがある。

\vspace{3mm}
\includegraphics[width=10cm]{img/apache-error.png}
\vspace{3mm}

これは、Apacheが使用する80番と443番ポートが、すでに他のアプリによって
使用されているためかもしれない。

Windowsのタスクバーの検索で、``Windowsの機能の有効化または無効化''を
検索する。

すると、以下のようなダイアログが表示されるので、
``インターネット・インフォーメーション・サービス''の項目を見てみる。

黒くなっていたら、その''+''をクリックして展開し、
``World Wide Webサービス''の黒をクリックして白くする。
そして、OKとする。

\vspace{3mm}
\includegraphics[width=10cm]{img/win-iis.png}
\vspace{3mm}

すると、インターネット・インフォーメーション・サービスを停止できるので、
80番ポート、443番ポートが開放されるので、Apacheが起動できる。


\section{MySQLをサービスに登録する}

MariaDB(MySQL)を停止させずに Windowsを終了させると、MariaDBのテーブルが
壊れやすくなると言われている。また、身近にもそういうケースが見られる。

そこで、mysqlをWindowsのサービスに登録する。
サービスに登録すれば、Windowsは終了時にサービスに登録しているプロセスを
停止させるだろうからである。

MySQLをサービスに登録するには、\textbf{C:\yen xampp\yen mysql} にある
``mysql\_installservice.bat''を使う。これは、mysqlをサービスに登録する
ためのスクリプトである。

ただ、このファイルの中を修正しなければならない点が1つある。
このファイルをTerapadなどのエディタで開く。

\begin{lstlisting}[title=mysql\_installservice.bat, language=bash]
 ... (略) ...
 
28 :MainNT
29 echo Installing MySQL as an Service 
30 copy "%cd%\bin\my.cnf" /-y %windir%\my.ini
31 bin\mysqld --install mysql --defaults-file="%cd%\bin\my.ini"
32 echo Try to start the MySQL deamon as service ... 
33 net start MySQL

 ... (略) ...
\end{lstlisting}

これの31行目の ``\verb!bin\mysqld!'' を
``\verb!bin\mysqld.exe!'' に変更する。

それから、コマンドプロンプトを 管理者権限で起動し、``C:\yen xampp\yen mysql''
に移動する。

\begin{lstlisting}
 C:\Widows\System32> cd C:\xampp\mysql
 C:\xampp\mysql>
\end{lstlisting}

そこで、``mysql\_installservice.bat''を実行する。

\begin{lstlisting}
 C:\xampp\mysql> mysql_installservice.bat
\end{lstlisting}

これで、mysql をサービスに登録できた。

\vspace{3mm}
\includegraphics[width=10cm]{../02-install/23-mysql-installservice.png}
\vspace{3mm}


\begin{tcolorbox}[title=注意]
 mysql\_installservice.bat を右クリック--- 管理者権限で実行、では、
 うまくいかない。
\end{tcolorbox}

XAMPPコントロールパネルを起動しなおすと、mysqlが動作しているのがわかる。


\include{end}

%% 修正時刻: Fri 2024/03/22 21:11:150
