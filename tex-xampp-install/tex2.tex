\documentclass[dvipdfmx]{jsarticle}

\include{begin}

\newpage

\section{XAMPPの設定}

\subsection{TeraPad の登録}

''XAMPPコントロールパネル'' が起動したら、まず、右上の ``Config'' をクリックする。

\vspace{3mm}
\includegraphics[width=12cm]{../03-config/21-config.png}
\vspace{3mm}

次に開いた窓で、''Editor'' の設定を変更する。

メモ帳 が初期値になっているので、それを ``TeraPad'' に変更する。

\textsf{C:\yen Program Files (x86)\yen terapad\yen TeraPad.exe} を指定する。

\vspace{3mm}
\includegraphics[width=10cm]{../03-config/22-terapad.png}
\vspace{3mm}

\newpage
\subsection{php.ini の設定}

XAMPPコントロールパネルの ''Apache'' の行の ``Config'' をクリックして、
表示されたサブメニューから \textsf{PHP (pho.ini)} を選択する。

\vspace{3mm}
\includegraphics[width=14cm]{../04-php-ini/11-php-ini.png}
\vspace{3mm}

php.ini が TeraPad で 開くので、上の虫メガネの左端のアイコンをクリックして、
出てきたウィンドウで、\textsf{timezone} と入力して、 ``先頭から検索'' を
クリックする。

\vspace{3mm}
\includegraphics[width=13cm]{../04-php-ini/12-timezone.png}
\vspace{3mm}

上の虫メガネの右端のアイコンを 2回 あるいは 3回 クリックすると、
1972行目あたりに \textsf{date.timezone=Europe/Berlin} という行が
見つかる。

\vspace{3mm}
\includegraphics[width=13cm]{../04-php-ini/13-find-timezone.png}
\vspace{3mm}


その Europe/Berlin を \textsf{Asia/Tokyo} に変更する。

\vspace{3mm}
\includegraphics[width=13cm]{../04-php-ini/14-asia-tokyo.png}
\vspace{3mm}

これで、php.ini の設定は終了である。

\subsection{余談 ------ mbstring の設定}

『PHPノート』(p.25)に載っている mbstring の設定は、PHP5.6以降(だったかな)は不要(非推奨)である。
現在では、\textsf{timezone} の設定だけでいける。

\href{https://www.php.net/manual/ja/mbstring.configuration.php}{https://www.php.net/manual/ja/mbstring.configuration.php}

\subsection{my.iniの設定}

XAMPPコントロールパネルの''MySQL''の行の''Config''をクリックして、
表示されたサブメニューから ``my.ini'' を選択する。

\vspace{3mm}
\includegraphics[width=13cm]{img/mysql-config.png}
\vspace{3mm}

my.iniがTeraPadで開くので、

\begin{lstlisting}[numbers=none]
 [client]
 ...
 default-character-set=utf8mb4             # <-
 ...
 [mysqld]
\end{lstlisting}

\vspace{3mm}
\includegraphics[width=13cm]{img/my-ini-01.png}
\vspace{3mm}


\begin{lstlisting}[numbers=none]
 ...
 [mysqld]
 ...
 character-set-server=utf8mb4               # <-
 collation-server=utf8mb4_general_ci        # <-
 ...
 [mysqldump]
\end{lstlisting}

\vspace{3mm}
\includegraphics[width=13cm]{img/my-ini-02.png}
\vspace{3mm}

上記のように記述を追加したのち、
mysql を ``stop'' してから ``start'' と再起動する。

\vspace{6mm}
\textgt{確認方法}

コマンドプロンプトを起動して、MySQLにログインする。

\framebox{$>$ mysql -u root -p}

以下のコマンドを実行する。


\begin{tabular}{|l|} \hline
 \verb!MariaDB[(nome)]> show variables like '%char%';! \\ \hline
\end{tabular}

以下のように表示されればよい。

\vspace{3mm}
\includegraphics[width=13cm]{img/charset.png}
\vspace{3mm}



\vspace{3cm}

\section{環境変数への登録}

\subsection{php を環境変数 Path に登録する}

\subsubsection{システム環境変数の編集}

システム環境変数の \textsf{PATH} に、php.exe のある場所を登録する。

スタートボタン右の 虫メガネ に、''システム'' と入力し、表示された候補から
''システム環境変数の編集'' を選択する。

\vspace{3mm}
\includegraphics[width=14cm]{../05-path/01-system-variable.png}
\vspace{3mm}

開いたウィンドウで、''環境変数'' をクリックする。

\vspace{3mm}
\includegraphics[width=8cm]{../05-path/02-env-variable.png}
\vspace{3mm}

開いたウィンドウで、下の ``システム環境変数'' の ''\textsf{Path}'' を
選択する。

そして、下の ''編集'' をクリックする。

\vspace{3mm}
\includegraphics[width=11cm]{../05-path/03-path.png}
\vspace{3mm}

''環境変数名の編集''画面で、''新規'' を選択する。

\vspace{3mm}
\includegraphics[width=13cm]{../05-path/04-new-path.png}
\vspace{3mm}

空欄の項目ができるので、そこに
\begin{tcolorbox}
 C:\yen xampp\yen php
\end{tcolorbox}

と、入力する。

\vspace{3mm}
\includegraphics[width=13cm]{../05-path/05-path-xampp-php.png}
\vspace{3mm}

あとは、''OK'' をクリックして閉じていく。
``×'' や ''キャンセル'' をクリックすると、設定が反映されない。
必ず ''OK'' をクリックする。

\subsubsection{確認}

もし、今、コマンドプロンプトの黒い画面が開いていたら、いったん閉じる。

それから、コマンドプロンプトを開いて、以下のコマンドを入力する。

\begin{tcolorbox}
 $>$ php -v
\end{tcolorbox}

PHPのバージョンが表示されれば成功である。

\newpage

\subsection{mysql を環境変数 Path に登録する}

phpのときと同様にして、mysql を環境変数 Path に登録する。

登録するパスは \textsf{C:\yen xampp\yen mysql\yen bin} である。

\vspace{3mm}
\includegraphics[width=13cm]{../06-mysql/01-mysql-path.png}
\vspace{3mm}

確認は、コマンドプロンプトを開きなおしてから、以下を入力する。

\begin{tcolorbox}
 $>$ mysql {-}{-}version
\end{tcolorbox}
\rightline{※ {-}{-} は ハイフン2つ}

mysql (MariaDB) のバージョンが表示される。


\include{end}

%% 修正時刻: Fri 2024/09/27 09:15:021
