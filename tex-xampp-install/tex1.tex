\documentclass[dvipdfmx]{jsarticle}

\include{begin}

\section{XAMPPのダウンロードとインストール}

\subsection{ダウンロード}

google などで、``xampp''で検索。

\vspace{3mm}
\includegraphics[width=13cm]{../01-download/01-xampp.png}
\vspace{3mm}

XAMPP のページが開くので、''Windows版'' をダウンロードする。

\vspace{3mm}
\includegraphics[width=13cm]{../01-download/02-xampp-download.png}
\vspace{3mm}

ファイルを保存する。``ダウンロード''フォルダに保存される。

\vspace{3mm}
\includegraphics[width=10cm]{../01-download/03-xampp.png}
\vspace{3mm}

\subsection{インストール}

``ダウンロード'' フォルダの ``\textsf{xampp-windows-x64-8.0.9-0-VS16-install.exe}''
をダブルクリックしてインストールを実行する。

○○が Windows の変更を許可しますか? みたいなことが表示されたら、``はい'' を選択する。

次に英語で ''Warning'' が表示される。

\vspace{3mm}
\includegraphics{../02-install/01-install.png}
\vspace{3mm}

Google翻訳
\begin{tcolorbox}
 重要! システムでアクティブ化されたユーザーアカウント制御(UAC)が原因で、XAMPPの一部の機能が制限されている可能性があります。 \\
 UACでは、XAMPPをC:\yen Program Filesにインストールしないでください(書き込み権限がありません)。 \\
 または、この設定後にmsconfigを使用してUACを非アクティブ化します。
\end{tcolorbox}

C:\yen Program Files にはインストールしないから、大丈夫。

次は、コンポーネントの選択画面である。

既定値では、すべてのコンポーネントが選択されているが、``FileZilla'' と ``Merucry'' と
``Tomcat'' は いらない。
インストールしても、使うことはまずない。

\vspace{3mm}
\includegraphics{../02-install/02-check.png}
\vspace{3mm}

インストール先の確認画面である。

\vspace{3mm}
\includegraphics[width=9cm]{../02-install/03-folder.png}
\vspace{3mm}

\textsf{C:\yen XAMPP} にインストールされる。(覚えておく)

次は、\textsf{XAMPP Control Panel} ではどの言語を使うかを選択できる。
が、日本語はない。

\vspace{3mm}
\includegraphics[width=10cm]{../02-install/04-english.png}
\vspace{3mm}

次の画面では、チェックをはずすが、これはチェックがはいったままでも
かまわない。
Bitnami for XAMPP についての情報へのリンクをつくるかどうかを
尋ねているだけである。

\vspace{3mm}
\includegraphics[width=10cm]{../02-install/05-no.png}
\vspace{3mm}

さて、これでインストールが実行される。しばらくかかる。

\subsection{インストール後のメニューの設定と起動}

インストールが終了すると、スタートメニューに XAMPP のメニューができている。

\textsf{XAMPP Control Panel} の項目を右クリックして、''スタートにピン留めする'' を
クリックして、スタートパネルから呼び出せるようにしておく。

\vspace{3mm}
\includegraphics[width=13cm]{../02-install/06-start-panel.png}
\vspace{3mm}

\newpage
XAMPPコントロールパネルを起動するときは、``管理者として実行'' をする必要がある。
右クリックして、``その他'' --- ``管理者として実行'' を選択する。


\vspace{3mm}
\includegraphics[width=14cm]{../02-install/07-kanrisya.png}
\vspace{3mm}










\include{end}

%% 修正時刻: Mon Aug  9 09:11:14 2021
