\documentclass[dvipdfmx]{jsarticle}

\include{begin}

\section{データベースをバックアップする}

\subsection{バックアップ}

データの入力が終わったら、データベースをバックアップする。

ここでは、mysqldump というコマンドを使っておこなう。

まず、MySQLをログアウトする。

\begin{tcolorbox}
 MariaDB [sample]$>$ exit (もしくは quit) \\
 C:\yen Users\yen XXXXXX\yen Documents\yen mysql$>$
\end{tcolorbox}

\rightline{※ XXXXXX は、各自のユーザー名}

コマンドプロンプトに戻る。

ここで以下のコマンドを実行する。

\begin{tcolorbox}
 $>$ mysqldump -u root -p -B sample $>$ sample.dump
\end{tcolorbox}

\begin{tabular}{|l|} \hline
 mysqldump \ -u \ ユーザー名 \ -p \ -B \ データベース名 \ $>$
 保存ファイル名 \\ \hline
\end{tabular}

ここでは保存ファイル名を sample.dump としたが、好みのファイル名を指定すればよい。

dir というコマンドを実行すると、ファイル一覧が見れる。

ファイル群の中に sample.dump があるはず。

これは テキストファイルなので、TeraPad などのエディタで内容を見ることができる。
\footnote{また、{\em -B} オプションをつけずにバックアップを取ることもできる。 \\
\fbox{ $>$ mysqldump -u sampleuser -p sample $>$ sample\_db.dump } \\
この場合は、バックアップファイルの記述の中に、CREATE DATABASE 文がない。
}

\subsection{データのリストア(復元)}

以下のコマンドを実行する。
\footnote{-B オプションをつけずにバックアップファイルを作成した場合は、まず sampleデータベースを作成してから \\
\fbox{ $>$ mysql -u sampleuser -p $<$ sample.dump } \\
とするか、あるいは、sampleuserでログインしてから \\
\fbox{MariaDB[sample] $>$ source sample.dump} とすればよい。
}

\begin{tcolorbox}
 $>$ mysql -u sampleuser -p $<$ sample.dump
\end{tcolorbox}

\fbox{mysql \ -u \ ユーザー名 \ -p \ $<$ \ 保存したファイル名}
\vspace{3mm}

これでリストアができている。



\include{end}

%% 修正時刻: Tue 2023/03/28 17:33:091
