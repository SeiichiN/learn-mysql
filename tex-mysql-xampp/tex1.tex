\documentclass[dvipdfmx]{jsarticle}


\usepackage{tcolorbox}
\usepackage{color}
\usepackage{listings, plistings}

%% ノート/latexメモ
%% http://pepper.is.sci.toho-u.ac.jp/pepper/index.php?%A5%CE%A1%BC%A5%C8%2Flatex%A5%E1%A5%E2

%% JavaScriptの設定
%% https://e8l.hatenablog.com/entry/2015/11/29/232800
\lstdefinelanguage{javascript}{
  morekeywords = [1]{ %keywords
    await, break, case, catch, class, const, continue, debugger, default, delete, 
    do, else, enum, export, extends, finally, for, function, function*, if, implements, import, in, 
    instanceof, interface, let, new, package, private, protected, public, return, static, super,
    switch, this, throw, try, typeof, var, void, while, with, yield, yield*
  },
  morekeywords = [2]{ %literal
    false, Infinity, NaN, null, true, undefined
  },
  morekeywords = [3] { %Classes
    Array, ArrayBuffer, Boolean, DataView, Date, Error, EvalError, Float32Array, Float64Array,
    Function, Generator, GeneratorFunction, Int16Array, Int32Array, Int8Array, InternalError,
    JSON, Map, Math, Number, Object, Promise, Proxy, RangeError, ReferenceError, Reflect,
    RegExp, Set, String, Symbol, SyntaxError, TypeError, URIError, Uint16Array, Uint32Array,
    Uint8Array, Uint8ClampedArray, WeakMap, WeakSet
  },
  morecomment = [l]{//},
  morecomment = [s]{/*}{*/},
  morestring = [b]{"},
  morestring = [b]{'},
  alsodigit = {-},
  sensitive = true
}

%% 修正時刻: Tue 2022/03/15 10:04:41


% Java
\lstset{% 
  frame=single,
  backgroundcolor={\color[gray]{.9}},
  stringstyle={\ttfamily \color[rgb]{0,0,1}},
  commentstyle={\itshape \color[cmyk]{1,0,1,0}},
  identifierstyle={\ttfamily}, 
  keywordstyle={\ttfamily \color[cmyk]{0,1,0,0}},
  basicstyle={\ttfamily},
  breaklines=true,
  xleftmargin=0zw,
  xrightmargin=0zw,
  framerule=.2pt,
  columns=[l]{fullflexible},
  numbers=left,
  stepnumber=1,
  numberstyle={\scriptsize},
  numbersep=1em,
  language={Java},
  lineskip=-0.5zw,
  morecomment={[s][{\color[cmyk]{1,0,0,0}}]{/**}{*/}},
  keepspaces=true,         % 空白の連続をそのままで
  showstringspaces=false,  % 空白字をOFF
}
%\usepackage[dvipdfmx]{graphicx}
\usepackage{url}
\usepackage[dvipdfmx]{hyperref}
\usepackage{amsmath, amssymb}
\usepackage{itembkbx}
\usepackage{eclbkbox}	% required for `\breakbox' (yatex added)
\usepackage{enumerate}
\usepackage[default]{cantarell}
\usepackage[T1]{fontenc}
\fboxrule=0.5pt
\parindent=1em
\definecolor{mygrey}{rgb}{0.97, 0.97, 0.97}

\makeatletter
\def\verbatim@font{\normalfont
\let\do\do@noligs
\verbatim@nolig@list}
\makeatother

\begin{document}

%\anaumeと入力すると穴埋め解答欄が作れるようにしてる。\anaumesmallで小さめの穴埋めになる。
\newcounter{mycounter} % カウンターを作る
\setcounter{mycounter}{0} % カウンターを初期化
\newcommand{\anaume}[1][]{\refstepcounter{mycounter}{#1}{\boxed{\phantom{aa}\textnormal{\themycounter}\phantom{aa}}}} %穴埋め問題の空欄作ってる。
\newcommand{\anaumesmall}[1][]{\refstepcounter{mycounter}{#1}{\boxed{\tiny{\phantom{a}\themycounter \phantom{a}}}}}%小さい版作ってる。色々改造できる。

%% 修正時刻: Tue 2022/03/15 10:04:411


\section{データベースを設計する}

\subsection{扱うデータ}

以下のようなデータを扱うこととする。

\vspace{3mm}
 \begin{tabular}{|c|} \hline
  菅原文太 \\
  40歳 \\
  1933年生まれ \\
  総務部 \\ \hline
 \end{tabular}
 \quad
 \begin{tabular}{|c|} \hline
  千葉真一 \\
  34歳 \\
  1939年生まれ \\
  営業部 \\ \hline
 \end{tabular}
 \quad
 \begin{tabular}{|c|} \hline
  北大路欣也 \\
  30歳 \\
  1943年生まれ \\
  経理部 \\ \hline
 \end{tabular}
 \quad
 \begin{tabular}{|c|} \hline
  梶芽衣子 \\
  26歳 \\
  1947年生まれ \\
  営業部 \\ \hline
 \end{tabular}
\vspace{3mm}

また、その会社には以下のような部署があるとする。

\vspace{3mm}
\begin{tabular}{|l|} \hline
 総務部 \\
 経理部 \\
 営業部 \\
 開発部 \\
 人事部 \\
 情報システム部 \\ \hline
\end{tabular}
\vspace{3mm}



\newpage
\section{データベースを作成する}

\subsection{rootユーザーでログインする}

作業をするために、rootユーザーでログインする。

\begin{tcolorbox}
 $>$ mysql -u root -p \\
 $>$ Enter password: 
\end{tcolorbox}


\subsection{データベースを作成する}

データベース名を \textsf{sample} とする。

\begin{tcolorbox}
 MariaDB [(none)]$>$ CREATE DATABASE sample;
\end{tcolorbox}


\subsection{sample専用ユーザーを作成する}

アプリケーションがこのデータベースにアクセスするためには、
アプリケーションがこのデータベースにログインできなければならない。

ふつう、アプリケーションには rootではログインさせない。
そのアプリケーション専用のユーザーアカウントを作成し、
そのアカウントでログインさせるようにする。

\vspace{3mm}
\begin{tabular}{|l|l|} \hline
 ユーザー名 & sampleuser \\
 パスワード & sampleuser \\ \hline
\end{tabular}
\vspace{3mm}

ユーザーの作成は以下のコマンドでできる。

\begin{tcolorbox}
 MariaDB [(none)]$>$ CREATE USER 'sampleuser'@'localhost' \\
 \qquad --$>$ IDENTIFIED BY 'sampleuser'; 
\end{tcolorbox}

そのユーザーに、sampleデータベースへの権限を与える。

\begin{tcolorbox}
 MariaDB [(none)]$>$ GRANT ALL ON sample.* TO 'sampleuser'@'localhost'; 
\end{tcolorbox}

``sample.*'' というのは、``sampleというデータベースの全てのテーブル''
に対してという意味である。

ユーザー名は文字列であることを明示するために 'sampleuser' とする。

次に ``@'' を記述し、そのあとに ホスト名を 文字列で指定する。
そのユーザーの属しているホストの名前である。
'\%' と指定すると、全てのホストから接続可能となるが、通常はそのような指定は
しない。
'localhost'というのは、このデータベースに接続しようとしているアプリの所属
するホストである。つまり、このデータベースとアプリとは同じホストに属している
ということになる。

\vspace{3mm}
\begin{tabular}{|l|} \hline
 ユーザーの作成と権限の設定は、以下のコマンドで一度にできる。\\
  MariaDB [(none)]$>$ GRANT ALL ON sample.* TO 'sampleuser'@'localhost' \\
 \qquad --$>$ IDENTIFIED BY 'sampleuser'; \\ \hline
\end{tabular}


\subsubsection{作成したユーザーでログインし、データベースを作成する}

このまま rootユーザーとして作業をしてもいいのであるが、
ここでは、今作成したユーザーでログインして作業することにする。

\textsf{exit} あるいは \textsf{quit} でログアウトする。

\begin{tcolorbox}
 MariaDB [(none)]$>$ exit
\end{tcolorbox}


作成したユーザー \textsf{sampleuser} でログインする。

\begin{tcolorbox}
 $>$ mysql -u sampleuser -p (Enterキー)\\
 Enter password: ********** (sampleuser と入力)
\end{tcolorbox}






\end{document}

%% 修正時刻: Sat May  2 15:10:04 2020


%% 修正時刻: Tue 2023/03/28 17:33:091
