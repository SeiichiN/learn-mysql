\documentclass[dvipdfmx]{jsarticle}

\include{begin}

\section{データベースを設計する}

\subsection{扱うデータ}

以下のようなデータを扱うこととする。

\vspace{3mm}
 \begin{tabular}{|c|} \hline
  菅原文太 \\
  40歳 \\
  1933年生まれ \\
  総務部 \\ \hline
 \end{tabular}
 \quad
 \begin{tabular}{|c|} \hline
  千葉真一 \\
  34歳 \\
  1939年生まれ \\
  営業部 \\ \hline
 \end{tabular}
 \quad
 \begin{tabular}{|c|} \hline
  北大路欣也 \\
  30歳 \\
  1943年生まれ \\
  経理部 \\ \hline
 \end{tabular}
 \quad
 \begin{tabular}{|c|} \hline
  梶芽衣子 \\
  26歳 \\
  1947年生まれ \\
  営業部 \\ \hline
 \end{tabular}
\vspace{3mm}

また、その会社には以下のような部署があるとする。

\vspace{3mm}
\begin{tabular}{|l|} \hline
 総務部 \\
 経理部 \\
 営業部 \\
 開発部 \\
 人事部 \\
 情報システム部 \\ \hline
\end{tabular}
\vspace{3mm}



\newpage
\section{データベースを作成する}

\subsection{rootユーザーでログインする}

作業をするために、rootユーザーでログインする。

\begin{tcolorbox}
 $>$ mysql -u root -p \\
 $>$ Enter password: 
\end{tcolorbox}


\subsection{データベースを作成する}

データベース名を \textsf{sample} とする。

\begin{tcolorbox}
 MariaDB [(none)]$>$ CREATE DATABASE sample;
\end{tcolorbox}


\subsection{sample専用ユーザーを作成する}

アプリケーションがこのデータベースにアクセスするためには、
アプリケーションがこのデータベースにログインできなければならない。

ふつう、アプリケーションには rootではログインさせない。
そのアプリケーション専用のユーザーアカウントを作成し、
そのアカウントでログインさせるようにする。

\vspace{3mm}
\begin{tabular}{|l|l|} \hline
 ユーザー名 & sampleuser \\
 パスワード & sampleuser \\ \hline
\end{tabular}
\vspace{3mm}

ユーザーの作成は以下のコマンドでできる。

\begin{tcolorbox}
 MariaDB [(none)]$>$ CREATE USER 'sampleuser'@'localhost' \\
 \qquad --$>$ IDENTIFIED BY 'sampleuser'; 
\end{tcolorbox}

そのユーザーに、sampleデータベースへの権限を与える。

\begin{tcolorbox}
 MariaDB [(none)]$>$ GRANT ALL ON sample.* TO 'sampleuser'@'localhost'; 
\end{tcolorbox}

``sample.*'' というのは、``sampleというデータベースの全てのテーブル''
に対してという意味である。

ユーザー名は文字列であることを明示するために 'sampleuser' とする。

次に ``@'' を記述し、そのあとに ホスト名を 文字列で指定する。
そのユーザーの属しているホストの名前である。
'\%' と指定すると、全てのホストから接続可能となるが、通常はそのような指定は
しない。
'localhost'というのは、このデータベースに接続しようとしているアプリの所属
するホストである。つまり、このデータベースとアプリとは同じホストに属している
ということになる。

\vspace{3mm}
\begin{tabular}{|l|} \hline
 ユーザーの作成と権限の設定は、以下のコマンドで一度にできる。\\
  MariaDB [(none)]$>$ GRANT ALL ON sample.* TO 'sampleuser'@'localhost' \\
 \qquad --$>$ IDENTIFIED BY 'sampleuser'; \\ \hline
\end{tabular}


\subsubsection{作成したユーザーでログインし、データベースを作成する}

このまま rootユーザーとして作業をしてもいいのであるが、
ここでは、今作成したユーザーでログインして作業することにする。

\textsf{exit} あるいは \textsf{quit} でログアウトする。

\begin{tcolorbox}
 MariaDB [(none)]$>$ exit
\end{tcolorbox}


作成したユーザー \textsf{sampleuser} でログインする。

\begin{tcolorbox}
 $>$ mysql -u sampleuser -p (Enterキー)\\
 Enter password: ********** (sampleuser と入力)
\end{tcolorbox}






\include{end}

%% 修正時刻: Tue 2023/03/28 17:33:091
