\documentclass[dvipdfmx]{jsarticle}


\usepackage{tcolorbox}
\usepackage{color}
\usepackage{listings, plistings}

%% ノート/latexメモ
%% http://pepper.is.sci.toho-u.ac.jp/pepper/index.php?%A5%CE%A1%BC%A5%C8%2Flatex%A5%E1%A5%E2

%% JavaScriptの設定
%% https://e8l.hatenablog.com/entry/2015/11/29/232800
\lstdefinelanguage{javascript}{
  morekeywords = [1]{ %keywords
    await, break, case, catch, class, const, continue, debugger, default, delete, 
    do, else, enum, export, extends, finally, for, function, function*, if, implements, import, in, 
    instanceof, interface, let, new, package, private, protected, public, return, static, super,
    switch, this, throw, try, typeof, var, void, while, with, yield, yield*
  },
  morekeywords = [2]{ %literal
    false, Infinity, NaN, null, true, undefined
  },
  morekeywords = [3] { %Classes
    Array, ArrayBuffer, Boolean, DataView, Date, Error, EvalError, Float32Array, Float64Array,
    Function, Generator, GeneratorFunction, Int16Array, Int32Array, Int8Array, InternalError,
    JSON, Map, Math, Number, Object, Promise, Proxy, RangeError, ReferenceError, Reflect,
    RegExp, Set, String, Symbol, SyntaxError, TypeError, URIError, Uint16Array, Uint32Array,
    Uint8Array, Uint8ClampedArray, WeakMap, WeakSet
  },
  morecomment = [l]{//},
  morecomment = [s]{/*}{*/},
  morestring = [b]{"},
  morestring = [b]{'},
  alsodigit = {-},
  sensitive = true
}

%% 修正時刻: Tue 2022/03/15 10:04:41


% Java
\lstset{% 
  frame=single,
  backgroundcolor={\color[gray]{.9}},
  stringstyle={\ttfamily \color[rgb]{0,0,1}},
  commentstyle={\itshape \color[cmyk]{1,0,1,0}},
  identifierstyle={\ttfamily}, 
  keywordstyle={\ttfamily \color[cmyk]{0,1,0,0}},
  basicstyle={\ttfamily},
  breaklines=true,
  xleftmargin=0zw,
  xrightmargin=0zw,
  framerule=.2pt,
  columns=[l]{fullflexible},
  numbers=left,
  stepnumber=1,
  numberstyle={\scriptsize},
  numbersep=1em,
  language={Java},
  lineskip=-0.5zw,
  morecomment={[s][{\color[cmyk]{1,0,0,0}}]{/**}{*/}},
  keepspaces=true,         % 空白の連続をそのままで
  showstringspaces=false,  % 空白字をOFF
}
%\usepackage[dvipdfmx]{graphicx}
\usepackage{url}
\usepackage[dvipdfmx]{hyperref}
\usepackage{amsmath, amssymb}
\usepackage{itembkbx}
\usepackage{eclbkbox}	% required for `\breakbox' (yatex added)
\usepackage{enumerate}
\usepackage[default]{cantarell}
\usepackage[T1]{fontenc}
\fboxrule=0.5pt
\parindent=1em
\definecolor{mygrey}{rgb}{0.97, 0.97, 0.97}

\makeatletter
\def\verbatim@font{\normalfont
\let\do\do@noligs
\verbatim@nolig@list}
\makeatother

\begin{document}

%\anaumeと入力すると穴埋め解答欄が作れるようにしてる。\anaumesmallで小さめの穴埋めになる。
\newcounter{mycounter} % カウンターを作る
\setcounter{mycounter}{0} % カウンターを初期化
\newcommand{\anaume}[1][]{\refstepcounter{mycounter}{#1}{\boxed{\phantom{aa}\textnormal{\themycounter}\phantom{aa}}}} %穴埋め問題の空欄作ってる。
\newcommand{\anaumesmall}[1][]{\refstepcounter{mycounter}{#1}{\boxed{\tiny{\phantom{a}\themycounter \phantom{a}}}}}%小さい版作ってる。色々改造できる。

%% 修正時刻: Tue 2022/03/15 10:04:411


\section{さまざまなコマンド}

\subsection{表定義関連}

\subsubsection{表定義の変更}

以下のような表定義があったとする。

\vspace{3mm}
\newsavebox{\sqlbox}
\setbox\sqlbox=\vbox{\hsize 6cm
\begin{verbatim}
CREATE TABLE students (
  id char(3),
  name varchar(20)
)
\end{verbatim}
}
\fbox{\usebox{\sqlbox}}
\vspace{3mm}

この name カラムに NOT NULL 制約を追加したいとする。
以下のようにできる。

\begin{tcolorbox}
\verb!MariaDB> ALTER TABLE students! \\
\verb!    -> MODIFY name varchar(20) NOT NULL;!
\end{tcolorbox}

\vspace{3mm}
\begin{tabular}{|l|} \hline
 ALTER TABLE テーブル名 \\
 MODIFY カラム名 カラム定義 \\ \hline
\end{tabular}
\vspace{3mm}


\subsubsection{カラム名の変更}

nameカラムの名称を st\_name に変更したいときは、以下のようにする。

\begin{tcolorbox}
 MariaDB$>$ ALTER \ TABLE \ students \\
 \qquad --$>$ CHANGE \ name \ st\_name VARCHAR(20);
\end{tcolorbox}

\vspace{3mm}
\begin{tabular}{|l|} \hline
 ALTER TABLE テーブル名 \\
 CHANGE カラム名 新しいカラム名 新しいデータ型 \\ \hline
\end{tabular}
\vspace{3mm}


\subsubsection{表定義にカラムを追加}

また、age (年齢) という項目を追加したいとする。以下のようにできる。

\begin{tcolorbox}
\verb!MariaDB> ALTER TABLE students! \\
\verb!    -> ADD age int NOT NULL;!
\end{tcolorbox}

\vspace{3mm}
\begin{tabular}{|l|} \hline
 ALTER TABLE テーブル名 \\
 ADD カラム名 カラム定義 \\ \hline
\end{tabular}
\vspace{3mm}

\subsubsection{表定義にプライマリーキーを追加}

あるいは、id に PRIMARY KEY を設定したいとする。以下のようにできる。

\begin{tcolorbox}
\verb!MariaDB> ALTER TABLE students! \\
\verb!    -> ADD PRIMARY KEY (id);!
\end{tcolorbox}

\vspace{3mm}
\begin{tabular}{|l|} \hline
 ALTER TABLE テーブル名 \\
 ADD PRIMARY KEY (カラム名) \\ \hline
\end{tabular}
\vspace{3mm}

\subsubsection{表定義に外部キー制約を追加}

外部キー制約を あとから設定するには、以下のようにする。

\begin{tcolorbox}
\verb!ALTER TABLE emp               -- empテーブル再定義! \\
\verb!   ADD                         -- 追加! \\
\verb!     CONSTRAINT fk_dept        -- 制約名 fk_dept! \\
\verb!     FOREIGN KEY (dept_id)     -- 外部キー dept_id! \\
\verb!     REFERENCES dept(id);      -- 参照  dept表の(id)!
\end{tcolorbox}

上の例は、emp表の中にある dept\_id という項目が、dept表の id という項目を参照しているという
設定である。

\subsection{表のデータの更新}

たとえば、emp表の id=1 の人の氏名を変更するには、以下のようにする。

\begin{tcolorbox}
 \verb!MariaDB> UPDATE emp SET name = '菅原文夫' WHERE id = 1;! 
\end{tcolorbox}

\begin{tabular}{|l|} \hline
 \verb!UPDATE テーブル名 SET カラム名 = 訂正データ WHERE 条件! \\ \hline
\end{tabular}

\subsection{表のデータの削除}

\subsubsection{1件のデータを削除する}

id が 1 のデータを削除する。

\begin{tcolorbox}
 \verb!MariaDB> DELETE FROM emp WHERE id = 1;!
\end{tcolorbox}

\begin{tabular}{|l|} \hline
 \verb!DELETE FROM テーブル名 WHERE 条件! \\ \hline
\end{tabular}

id が 1 のデータを削除すると、1 が欠番になる。

\begin{spacing}{0.8}        
\begin{verbatim}
+----+------------+-----+----------+---------+
| id | name       | age | birthday | dept_id |
+----+------------+-----+----------+---------+
|  2 | 千葉真一   |  34 |     1939 | 002     |
|  4 | 梶芽衣子   |  26 |     1947 | 002     |
|  3 | 北大路欣也 |  30 |     1943 | 003     |
+----+------------+-----+----------+---------+
\end{verbatim}
\end{spacing}

この場合、id の数字を振り直すことはしない。
なぜなら、id は順番を表す数字ではなく、そのデータを特定するための数字だからである。
それが プライマリー・キーの役割である。
順番になっているのは、データを作成するときに、AUTO\_INCREMENT を使ったからである。


\subsubsection{全データを削除する}

条件をつけなければ、全データが削除される。

\begin{tcolorbox}
 \verb!MariaDB> DELETE FROM emp;!
\end{tcolorbox}

\begin{tabular}{|l|} \hline
 \verb!DELETE FROM テーブル名! \\ \hline
\end{tabular}


\subsection{表のデータの表示・検索}

SELECT で表のデータを表示するとき、WHERE で条件を指定する際に、''LIKE'' で
条件を指定できる。

\begin{tcolorbox}
 \verb!MariaDB> SELECT * FROM emp WHERE name LIKE '%葉%';!
\end{tcolorbox}

\begin{verbatim}
+----+--------------+-----+----------+---------+
| id | name         | age | birthday | dept_id |
+----+--------------+-----+----------+---------+
|  2 | 千葉真一     |  34 |     1939 | 002     |
+----+--------------+-----+----------+---------+
\end{verbatim}

\begin{tabular}{|l|}
 \verb!SELECT [カラム名, ...] FROM テーブル名 WHERE カラム名 LIKE '%語句%';! \\
\end{tabular}

{\em \%} は 0文字以上の文字列に合致する。\\
{\em \_} は 1文字以上の文字列に合致する。


\subsection{MySQLの管理}

\subsubsection{MySQLの文字コードの確認}

MySQLで文字コードを確認するには、以下のコマンドを使う。

\begin{tcolorbox}
 \verb!MariaDB> SHOW VARIABLES LIKE '%char%';!
\end{tcolorbox}

Windowsのコマンドプロンプトで実行すると、以下のようになる。

\vspace{3mm}
\begin{tabular}{|l|} \hline
\verb!+--------------------------+----------------------------+! \\
\verb!| Variable_name            | Value                      |! \\
\verb!+--------------------------+----------------------------+! \\
\verb!| character_set_client     | cp932                      |! \\
\verb!| character_set_connection | cp932                      |! \\
\verb!| character_set_database   | utf8mb4                    |! \\
\verb!| character_set_filesystem | binary                     |! \\
\verb!| character_set_results    | cp932                      |! \\
\verb!| character_set_server     | utf8mb4                    |! \\
\verb!| character_set_system     | utf8                       |! \\
\verb!| character_sets_dir       | /usr/share/mysql/charsets/ |! \\ 
\verb!+--------------------------+----------------------------+! \\ \hline
\end{tabular}
\vspace{3mm}

Macのターミナルや、同じWindowsでも、コマンドプロンプトが UTF-8(cp65001)に
なっていると、以下のようになる。

\vspace{3mm}
\begin{tabular}{|l|} \hline
\verb!+--------------------------+----------------------------+! \\
\verb!| Variable_name            | Value                      |! \\
\verb!+--------------------------+----------------------------+! \\
\verb!| character_set_client     | utf8mb4                    |! \\
\verb!| character_set_connection | utf8mb4                    |! \\
\verb!| character_set_database   | utf8mb4                    |! \\
\verb!| character_set_filesystem | binary                     |! \\
\verb!| character_set_results    | utf8mb4                    |! \\
\verb!| character_set_server     | utf8mb4                    |! \\
\verb!| character_set_system     | utf8                       |! \\
\verb!| character_sets_dir       | /usr/share/mysql/charsets/ |! \\ 
\verb!+--------------------------+----------------------------+! \\ \hline
\end{tabular}
\vspace{3mm}

{\em mysql} というのはクライアント・アプリで、{\em mysqld} というのが
サーバー・アプリなのだが、mysqlクライアントが mysqldサーバーに接続
するときに、自身の環境の文字コードを mysqldサーバーに送るのである。
mysqldサーバーは、SQLコマンドの結果を mysqlクライアントに送るときに
そのクライアントの文字コードにして送るのである。

ただ、mysqldumpコマンドの場合、指定しなければ UTF-8 の文字コードで
送られる。だから、mysqldumpで取得できたダンプファイルの文字コードは
UTF-8 である。

これは、リストアする時に特に意識する必要はない。

\subsection{ユーザー一覧}

MySQLで今までに作成したユーザーの一覧を見たい場合、以下のコマンドでできる。
ただし、ルート・ユーザー(root)でないと権限がないと怒られる。

\begin{tcolorbox}
 \verb!MariaDB> SELECT user, host, db FROM mysql.db;!
\end{tcolorbox}

パスワードは暗号化されているので、見ても仕方がないが、以下のコマンドでできる。

\begin{tcolorbox}
 \verb!MariaDB> SELECT user, password FROM mysql.user;!
\end{tcolorbox}

\subsection{パスワードの再設定}

パスワードを忘れたなどで、パスワードを再設定したい場合は、以下のコマンドでできる。
(rootユーザーで実行)

\begin{tcolorbox}
 MariaDB$>$ SET \ PASSWORD \ FOR \ 'ユーザー名'@'localhost' \ = \ password('パスワード');
\end{tcolorbox}





\end{document}

%% 修正時刻: Sat May  2 15:10:04 2020


%% 修正時刻: Thu 2025/10/02 12:34:071
