\documentclass[dvipdfmx]{jsarticle}

\include{begin}

\section{ちょっと複雑なデータベースを考える}

\subsection{データベースの例}

今度は以下のようなデータについて考えてみる。

\vspace{3mm}
 \begin{tabular}{|l|l|} \hline
  氏名 & 菅原文太 \\
  性別 & 男性 \\
  年齢 & 40歳 \\
  誕生年 & 1933年生まれ \\
  部署 & 総務部 \\ 
  趣味 & 釣り、油絵、空手  \\ \hline
 \end{tabular}

\vspace{3mm}
 \begin{tabular}{|l|l|} \hline
  氏名 & 千葉真一 \\
  性別 & 男性 \\
  年齢 & 34歳 \\
  誕生年 &  1939年生まれ \\
  部署 & 営業部 \\
  趣味 & 空手、熱帯魚飼育、サッカー観戦、釣り  \\ \hline
 \end{tabular}

\vspace{3mm}
 \begin{tabular}{|l|l|} \hline
  氏名 & 北大路欣也 \\
  性別 & 男性 \\
  年齢 & 30歳 \\
  誕生年 &  1943年生まれ \\
  部署 & 経理部 \\
  趣味 & 茶道、空手  \\ \hline
 \end{tabular}

\vspace{3mm}
 \begin{tabular}{|l|l|} \hline
  氏名 & 梶芽衣子 \\
  性別 & 女性 \\
  年齢 & 26歳 \\
  誕生年 &  1947年生まれ \\
  部署 & 営業部 \\
  趣味 & 登山、ヨガ、サッカー観戦  \\ \hline
 \end{tabular}
\vspace{3mm}

まず、このような表がイメージされる。

 emp表 \\
  \begin{tabular}[h]{|c|l|c|c|c|c|l|}
   \hline
   ID & 名前       & 性別 & 年齢 & 誕生年 & 部署 & 趣味 \\ \hline\hline
   1  & 菅原文太   & 男性 & 40   & 1933   & 総務 & 釣り, 油絵, 空手  \\ \hline
   2  & 千葉真一   & 男性 & 34   & 1939   & 営業 & 空手, 熱帯魚飼育, サッカー観戦, 釣り \\ \hline
   3  & 北大路欣也 & 男性 & 30   & 1943   & 経理 & 茶道, 空手 \\ \hline
   4  & 梶芽衣子   & 女性 & 26   & 1947   & 営業 & 登山, ヨガ, サッカー観戦  \\ \hline
  \end{tabular}

しかしながら、この表の場合、趣味のフィールドには、複数のデータが含まれている。
これを解消したのが、次の表である。


\subsection{第1正規形}

\vspace{3mm}
emp表 \\
  \begin{tabular}[h]{|c|l|c|c|c|c|l|l|l|l|}
   \hline
   \underline{ID}
      & 名前       & 性別 & 年齢 & 誕生年 & 部署 & 趣味1 & 趣味2     & 趣味3       & 趣味4 \\ \hline\hline
   1  & 菅原文太   & 男性 & 40   & 1933   & 総務 & 釣り & 油絵       & 空手        &      \\ \hline
   2  & 千葉真一   & 男性 & 34   & 1939   & 営業 & 空手 & 熱帯魚飼育 & サッカー観戦 & 釣り \\ \hline
   3  & 北大路欣也 & 男性 & 30   & 1943   & 経理 & 茶道 & 空手       &             & \\ \hline
   4  & 梶芽衣子   & 女性 & 26   & 1947   & 営業 & 登山 & ヨガ       & サッカー観戦 & \\ \hline
  \end{tabular}
\vspace{3mm}


趣味のフィールドが1つのもの、3つのものとバラバラなので、フィールドを1つにする。

\vspace{3mm}
  emp表 \\
  \begin{tabular}[h]{|c|l|c|c|c|c|l|}
   \hline
   \underline{ID}
      & 名前        & 性別 & 年齢 & 誕生年 & 部署 & 趣味 \\ \hline\hline
   1  & 菅原文太    & 男性 & 40   & 1933   & 総務 & 釣り \\ \hline
   1  & 菅原文太    & 男性 & 40   & 1933   & 総務 & 油絵 \\ \hline
   1  & 菅原文太    & 男性 & 40   & 1933   & 総務 & 空手 \\ \hline
   2  & 千葉真一    & 男性 & 34   & 1939   & 営業 & 空手  \\ \hline
   2  & 千葉真一    & 男性 & 34   & 1939   & 営業 & 熱帯魚飼育 \\ \hline
   2  & 千葉真一    & 男性 & 34   & 1939   & 営業 & サッカー観戦 \\ \hline
   2  & 千葉真一    & 男性 & 34   & 1939   & 営業 & 釣り \\ \hline
   3  & 北大路欣也  & 男性 & 30   & 1943   & 経理 & 茶道 \\ \hline
   3  & 北大路欣也  & 男性 & 30   & 1943   & 経理 & 空手 \\ \hline
   4  & 梶芽衣子    & 女性 & 26   & 1947   & 営業 & 登山 \\ \hline
   4  & 梶芽衣子    & 女性 & 26   & 1947   & 営業 & ヨガ \\ \hline
   4  & 梶芽衣子    & 女性 & 26   & 1947   & 営業 & サッカー観戦 \\ \hline
  \end{tabular}
\vspace{3mm}

ここまでが「第1正規化」で、この表を「第1正規形」と呼ぶ。

\vspace{5mm}
\begin{tcolorbox}
{\large \textgt{第1正規形} :\quad 1つのセルには1つの値しか含まない}
\end{tcolorbox}

\newpage
\subsection{第2正規形}


上の表は縦方向にデータが繰り返されている。
これを表を分けることによって、解消する。
名前、性別等の列は、主キーである \textsf{ID} に従属している。
趣味の列は、この表には隠れている別の主キー に従属していると考える。
今回の場合は、emp表の \textsf{ID} と hobby表の \textsf{HID} である。
この2つのキーを主キー(複合キー)として、それに従属していると考える。

\begin{multicols}{2}

 emp表 \\
  \begin{tabular}[h]{|c|l|c|c|c|c|}
   \hline
   \underline{ID}
      & 名前       & 性別 & 年齢 & 誕生年 & 部署  \\ \hline\hline
   1  & 菅原文太   & 男性 & 40   & 1933   & 総務  \\ \hline
   2  & 千葉真一   & 男性 & 34   & 1939   & 営業  \\ \hline
   3  & 北大路欣也 & 男性 & 30   & 1943   & 経理  \\ \hline
   4  & 梶芽衣子   & 女性 & 26   & 1947   & 営業  \\ \hline
  \end{tabular}

 \columnbreak

 hobby表 \\
  \begin{tabular}[h]{|c|l|}
   \hline
   \underline{HID}
         & 趣味 \\ \hline\hline
   H01   & 釣り \\ \hline
   H02   & 油絵 \\ \hline
   H03   & 空手  \\ \hline
   H04   & 熱帯魚飼育 \\ \hline
   H05   & サッカー観戦 \\ \hline
   H06   & 茶道 \\ \hline
   H07   & 登山 \\ \hline
   H08   & ヨガ \\ \hline
  \end{tabular}
\end{multicols}

\noindent
emp\_hobby表\\
\begin{tabular}{|c|c|cc} \cline{1-2}
 \multicolumn{2}{|c|}{主キー} & \multirow{2}{*}{名前} & \multirow{2}{*}{趣味} \\ \cline{1-2}
 \underline{ID} & \underline{HID} &   &   \\ \cline{1-2}\cline{1-2}
 1  & H01 & 菅原文太 &  釣り \\ \cline{1-2}
 1  & H02 & 菅原文太 & 油絵 \\ \cline{1-2}
 1  & H03 & 菅原文太 & 空手 \\ \cline{1-2}
 2  & H03 & 千葉真一 & 空手 \\ \cline{1-2}
 2  & H04 & 千葉真一 & 熱帯魚飼育 \\ \cline{1-2}
 2  & H05 & 千葉真一 & サッカー観戦 \\ \cline{1-2}
 2  & H01 & 千葉真一 & 釣り \\ \cline{1-2}
 3  & H06 & 北大路欣也 & 茶道 \\ \cline{1-2}
 3  & H03 & 北大路欣也 & 空手 \\ \cline{1-2}
 4  & H07 & 梶芽衣子 & 登山 \\ \cline{1-2}
 4  & H08 & 梶芽衣子 & ヨガ \\ \cline{1-2}
 4  & H05 & 梶芽衣子 & サッカー観戦 \\ \cline{1-2}
\end{tabular}

\vspace{3mm}
名前と趣味の列は、わかりやすくするためにつけた。
emp\_hobby表は、ID と HID の2つの列だけの表である。


\vspace{5mm}
\begin{tcolorbox}
{\large \textgt{第2正規形} :\quad 表の中に主キーが1つである。複合列が主キーとなっている場合は、主キーの1つだけに従属する列があってはならない。}
\end{tcolorbox}



\newpage
\subsection{第3正規形}

emp表には、表には現れていないが、id に従属せず、別のキーに従属すると
考えられるフィールドが存在する。それは「性別」と「所属」である。
これを別の表(gender表、dept表)とする。

\begin{multicols}{2}
 emp表 \\
  \begin{tabular}[h]{|c|l|c|c|}
   \hline
   \underline{ID}
      & 名前       & 年齢 & 誕生年 \\ \hline\hline
   1  & 菅原文太   & 40   & 1933   \\ \hline
   2  & 千葉真一   & 34   & 1939   \\ \hline
   3  & 北大路欣也 & 30   & 1943   \\ \hline
   4  & 梶芽衣子   & 26   & 1947   \\ \hline
  \end{tabular}

 \columnbreak

 gender表 \\
 \begin{tabular}{|c|c|} \hline
  \underline{GID}
      & 性別 \\  \hline\hline
  0   & 不明 \\ \hline
  1   & 男性 \\ \hline
  2   & 女性 \\ \hline
  3   & その他 \\ \hline
 \end{tabular}
\end{multicols}

\begin{multicols}{3}
 dept表 \\
 \begin{tabular}{|c|c|} \hline
  \underline{DID}
      & 部署名 \\ \hline\hline
  D01 & 総務部 \\ \hline
  D02 & 営業部 \\ \hline
  D03 & 経理部 \\ \hline
  D04 & 開発部 \\ \hline
 \end{tabular}
  
 \columnbreak

 hobby表 \\
  \begin{tabular}[h]{|c|l|}
   \hline
   \underline{HID}
         & 趣味 \\ \hline\hline
   H01   & 釣り \\ \hline
   H02   & 油絵 \\ \hline
   H03   & 空手  \\ \hline
   H04   & 熱帯魚飼育 \\ \hline
   H05   & サッカー観戦 \\ \hline
   H06   & 茶道 \\ \hline
   H07   & 登山 \\ \hline
   H08   & ヨガ \\ \hline
  \end{tabular}

 \columnbreak

 \noindent
 emp\_hobby表\\
 \begin{tabular}{|c|c|} \hline
  \multicolumn{2}{|c|}{主キー} \\ \hline
  \underline{ID} & \underline{HID}  \\ \hline\hline
  1  & H01  \\ \hline
  1  & H02  \\ \hline
  1  & H03  \\ \hline
  2  & H03  \\ \hline
  2  & H04  \\ \hline
  2  & H05  \\ \hline
  2  & H01  \\ \hline
  3  & H06  \\ \hline
  3  & H03  \\ \hline
  4  & H07  \\ \hline
  4  & H08  \\ \hline
  4  & H05  \\ \hline
 \end{tabular}
\end{multicols}

\vspace{5mm}
\begin{tcolorbox}
\large{\textgt{第3正規形} :\quad すべての列が主キーに従属している。}
\end{tcolorbox}


\newpage
\section{ER図}

第3正規形ができたところで、ER図を作ってみる。

\vspace{3mm}
\includegraphics[width=14cm]{img/ER.png}
\vspace{3mm}

\vspace{3mm}
\begin{center}
\includegraphics[width=6cm]{img/ERD-Notation-ja.png}
\end{center}
\vspace{3mm}


\newpage
\section{実際にデータベースを作ってみる}

\subsection{rootでMySQLサーバーにログインする}

以下のようなデータベースであるとする。

\begin{tcolorbox}
 データベース名 : \quad sample
\end{tcolorbox}

管理者(root) で MySQLサーバーにログインする。

\begin{lstlisting}[numbers=none, language=sql]
 > mysql -u root -p
 Password: ********
\end{lstlisting}

ます、データベースを作成する。
そして、sampleデータベースの使用を宣言する。

\begin{lstlisting}[language=SQL, numbers=none]
 MariaDB[(none)]> CREATE DATABASE sample;
 MariaDB[(none)]> USE sample;
\end{lstlisting}


\subsection{テーブルの定義}

以下のようにテーブルを定義する。

\begin{lstlisting}[caption=gender表, language=sql]
 MariaDB> CREATE TABLE gender (
     ->   id CHAR(1) PRIMARY KEY,
     ->   name VARCHAR(3) NOT NULL
     -> );
\end{lstlisting}

\begin{lstlisting}[caption=dept表, language=sql]
 MariaDB> CREATE TABLE dept (
     ->   id CHAR(3) PRIMARY KEY,
     ->   name VARCHAR(20) NOT NULL
     -> );
\end{lstlisting}

\begin{lstlisting}[caption=emp表, language=sql]
 MariaDB> CREATE TABLE emp (
     ->   id INT AUTO_INCREMENT,
     ->   name VARCHAR(100) NOT NULL,
     ->   gender_id CHAR(1) NOT NULL,
     ->   age INT NOT NULL,
     ->   birthyear INT NOT NULL,
     ->   dept_id CHAR(3),
     ->   PRIMARY KEY (id)
     -> );
\end{lstlisting}

gender\_id に外部キー制約を追加する。参照先は gender表のidである。

\begin{lstlisting}[caption=emp表, language=sql]
 MariaDB> ALTER TABLE emp
     -> ADD
     ->   CONSTRAINT fk_gender_id
     ->     FOREIGN KEY (gender_id)
     ->     REFERENCES gender (id);
\end{lstlisting}

dept\_id に外部キー制約を追加する。参照先は dept表のidである。

\begin{lstlisting}[caption=emp表, language=sql]
 MariaDB> ALTER TABLE emp
     -> ADD
     ->   CONSTRAINT fk_dept_id
     ->     FOREIGN KEY (dept_id)
     ->     REFERENCES dept (id);
\end{lstlisting}

hobby表を定義する。

\begin{lstlisting}[caption=hobby表, language=sql]
 mysql> CREATE TABLE hobby (
     ->   id CHAR(3) PRIMARY KEY,
     ->   name VARCHAR(20) NOT NULL
     -> );
\end{lstlisting}

誰がどういう趣味を持っているかの対応表を作成する。
この表では、カラムemp\_id と カラムhobby\_id の2つセットで主キーとなる。
(複合主キー)

\begin{lstlisting}[caption=emp\_hobby表, language=sql]
 mysql> CREATE TABLE emp_hobby (
     ->   emp_id INT NOT NULL,
     ->   hobby_id CHAR(3) NOT NULL,
     ->   PRIMARY KEY (emp_id, hobby_id)
     -> );
\end{lstlisting}

\subsection{データの登録}

gender表のデータは全部で4件である。gidはchar型1文字である。

\begin{lstlisting}[caption=gender表]
INSERT INTO gender
  (id, name)
VALUES
  ('0', '不明'),
  ('1', '男性'),
  ('2', '女性'),
  ('3', 'その他');
\end{lstlisting}

dept表のデータを入力する。didはchar型3文字である。

\begin{lstlisting}[caption=dept表]
INSERT INTO dept
  (id, name)
VALUES
  ('001', '総務部'),
  ('002', '営業部'),
  ('003', '経理部'),
  ('004', '開発部'); 
\end{lstlisting}

emp表を入力する。idは自動連番である。

\begin{lstlisting}[caption=emp表]
INSERT INTO emp
  (name, gender_id, age, birthyear, dept_id)
VALUES
  ('菅原文太', '1', 40, 1933, '001'),
  ('千葉真一', '1', 34, 1939, '002'),
  ('北大路欣也', '1', 30, 1943, '003'),
  ('梶芽衣子', '2', 26, 1947, '002'); 
\end{lstlisting}

hobby表を入力する。

\begin{lstlisting}[caption=hobby]
INSERT INTO hobby
  (id, name)
VALUES
  ('H01', '釣り'),
  ('H02', '油絵'),
  ('H03', '空手'),
  ('H04', '熱帯魚飼育'),
  ('H05', 'サッカー観戦'),
  ('H06', '茶道'),
  ('H07', '登山'),
  ('H08', 'ヨガ'); 
\end{lstlisting}

社員と趣味との対応表の入力である。

\begin{lstlisting}[caption=emp\_hobby]
INSERT INTO emp_hobby
  (emp_id, hobby_id)
VALUES
  (1, 'H01'),
  (1, 'H02'),
  (1, 'H03'),
  (2, 'H03'),
  (2, 'H04'),
  (2, 'H05'),
  (2, 'H01'),
  (3, 'H06'),
  (3, 'H03'),
  (4, 'H07'),
  (4, 'H08'),
  (4, 'H05'); 
\end{lstlisting}

\subsection{データを表示する}

以下のように4つの表を結合する。

\begin{lstlisting}[language=sql, numbers=none]
 MariaDB> SELECT
   ->   e.name AS 名前,
   ->   g.name AS 性別,
   ->   d.name AS 所属,
   ->   h.name AS 趣味
   -> FROM emp e
   -> INNER JOIN emp_hobby eh
   -> ON eh.emp_id = e.id
   ->   INNER JOIN gender g
   ->   ON g.id = e.gender_id
   ->     INNER JOIN dept d
   ->     ON d.id = e.dept_id
   ->       INNER JOIN hobby h
   ->       ON h.id = eh.hobby_id
   -> ORDER BY e.id;
\end{lstlisting}

結合表を表示するたびに、この長いコマンドを入力するのは大変なので、
(エディタに記述しておいて、必要なときにコピー貼り付けしてもいいのだが)
ビューという形でデータベース内にこのコマンドを登録しておくことができる。

以下のように ビュー を作成してみた。

\begin{lstlisting}[language=sql, numbers=none]
MariaDB> CREATE VIEW hobby_view AS
   -> SELECT
   ->   e.name AS 名前,
   ->   g.name AS 性別,
   ->   d.name AS 所属,
   ->   h.name AS 趣味
   -> FROM emp e
   -> INNER JOIN emp_hobby eh
   -> ON eh.emp_id = e.id
   ->   INNER JOIN gender g
   ->   ON g.id = e.gender_id
   ->     INNER JOIN dept d
   ->     ON d.id = e.dept_id
   ->       INNER JOIN hobby h
   ->       ON h.hid = eh.hobby_id 
   -> ORDER BY e.id;
\end{lstlisting}

こうしておくと、

\begin{lstlisting}[language=sql, numbers=none]
 MariaDB> SELECT * FROM hobby_view;
\end{lstlisting}

とするだけで、先の結合表を表示できる。
また、次のように、検索・表示もできる。

\begin{lstlisting}[language=sql, numbers=none]
 MariaDB> SELECT * FROM hobby_view
    -> WHERE 趣味 LIKE '%空手%';
\end{lstlisting}


\begin{lstlisting}[caption=出力例, numbers=none]
+-----------------+--------+-----------+--------+
| 名前            | 性別    | 所属      | 趣味    |
+-----------------+--------+-----------+--------+
| 菅原文太         | 男性    | 総務部    | 空手    |
| 千葉真一         | 男性    | 営業部    | 空手    |
| 北大路欣也       | 男性    | 経理部    | 空手    |
+-----------------+--------+-----------+--------+
\end{lstlisting}


\include{end}

%% 修正時刻: Thu 2025/10/02 12:34:072
