\documentclass[dvipdfmx]{jsarticle}

\include{begin}

\section{2つのテーブルを結合する}

\subsection{内部結合(JOIN句)}

empテーブルの dept\_id は、deptテーブルの id である。

だから、dept\_idをキーにして、二つのテーブルを結合できる。

結合するには \textsf{JOIN}句を使う。

実行例

\begin{tcolorbox}
 MariaDB [sample]$>$ SELECT \ * \\
 \qquad --$>$ FROM \ emp \\
 \qquad --$>$ \underline{JOIN} \ dept \\
 \qquad --$>$ \underline{ON} \ emp.dept\_id \ = \ dept.id;
\end{tcolorbox}

\begin{tabular}{|l|} \hline
 SELECT * \\
 FROM メインの表 \\
 \verb![INNER]!  JOIN  サブの表 \ \\
 ON  メインの表.キーカラム名  =  サブの表.キーカラム名 \\ \hline
\end{tabular}

\vspace{3mm}
\begin{itemize}
 \item *(アスタリスク) --- メインの表、サブの表のすべての項目を表示する。
 \item emp.dept\_id --- emp表の dept\_id カラム名。ピリオドで区切って指定。
ピリオドの後に空白を入れてはいけない。
 \item dept.id --- dept表の id カラム名。
\end{itemize}


\vspace{3mm}
これで二つのテーブルが結合される。

\begin{spacing}{0.8}        
\begin{verbatim}
+----+------------+-----+----------+---------+-----+--------+
| id | name       | age | birthday | dept_id | id  | name   |
+----+------------+-----+----------+---------+-----+--------+
|  1 | 菅原文太   |  40 |     1933 | 001     | 001 | 総務部 |
|  2 | 千葉真一   |  34 |     1939 | 002     | 002 | 営業部 |
|  4 | 梶芽衣子   |  26 |     1947 | 002     | 002 | 営業部 |
|  3 | 北大路欣也 |  30 |     1943 | 003     | 003 | 経理部 |
+----+------------+-----+----------+---------+-----+--------+
\end{verbatim}
\end{spacing}

\textsf{JOIN} は \textsf{INNER JOIN} と記述できる。
``内部結合''と呼ばれている。

\textsf{ON emp.dept\_id = dept.id} は 
emp の dept\_id と dept の id が等しければ、そのレコードを抜き出す。

\rightline{※ emp.dept\_id は empテーブルの dept\_id という意味になる。}

\subsection{表示項目を絞る}

現在は項目を全て表示しているが、これを変更する。

emp表の id, name, age と dept表の name だけを表示させる。


\begin{tcolorbox}
 MariaDB [sample]$>$ SELECT \\
 \qquad --$>$ \quad emp.id, \\
 \qquad --$>$ \quad emp.name, \\
 \qquad --$>$ \quad age, \\
 \qquad --$>$ \quad dept.name \\ 
 \qquad --$>$ FROM \ emp \\
 \qquad --$>$ JOIN \ dept \\
 \qquad --$>$ \quad ON \ emp.dept\_id \ = \ dept.id;
\end{tcolorbox}

\begin{spacing}{0.8}        
\begin{verbatim}
+----+------------+-----+--------+
| id | name       | age | name   |
+----+------------+-----+--------+
|  1 | 菅原文太   |  40 | 総務部 |
|  2 | 千葉真一   |  34 | 営業部 |
|  4 | 梶芽衣子   |  26 | 営業部 |
|  3 | 北大路欣也 |  30 | 経理部 |
+----+------------+-----+--------+
\end{verbatim}
\end{spacing}        

このように必要な項目のみ表示させることができる。
ただ、name という項目が二つあったり、英語であったりするので、
これを適切な日本語に変えることにする。

それには、\textsf{AS句} というのが使える。

たとえば、``emp.name AS 名前'' とすると、``emp.name'' は ``名前'' と表示される。

\begin{tcolorbox}
 MariaDB [sample]$>$ SELECT \\
 \qquad --$>$ \quad emp.id \ AS \ ID, \\ 
 \qquad --$>$ \quad emp.name \ AS \ 名前, \\
 \qquad --$>$ \quad age \ AS \ 年齢, \\ 
 \qquad --$>$ \quad dept.name \ AS \ 部署名 \\
 \qquad --$>$ FROM \ emp \\
 \qquad --$>$ JOIN \ dept \\
 \qquad --$>$ \quad ON \ emp.dept\_id \ = \ dept.id;
\end{tcolorbox}

\begin{spacing}{0.8}        
\begin{verbatim}
+----+------------+------+--------+
| ID | 名前       | 年齢 | 部署名 |
+----+------------+------+--------+
|  1 | 菅原文太   |   40 | 総務部 |
|  2 | 千葉真一   |   34 | 営業部 |
|  4 | 梶芽衣子   |   26 | 営業部 |
|  3 | 北大路欣也 |   30 | 経理部 |
+----+------------+------+--------+
\end{verbatim}
\end{spacing}

さらによく見てみると、この表は部署名の順に並んでいる。
これを ID順に並びかえる。

そのためには \textsf{ORDER句} というのが使える。

たとえば、今回の場合だと、\fbox{\textsf{ORDER BY emp.id [\! ASC\! ]}} とすることで、ID順になる。

\textsf{ASC} というのは''昇順''という意味で、省略すると ASC と指定したことになる。

また、\textsf{DESC} と指定すると ''降順'' で並びかえできる。


\begin{tcolorbox}
 MariaDB [sample]$>$ SELECT \\
 \qquad --$>$ \quad emp.id \ AS \ ID, \\
 \qquad --$>$ \quad emp.name \ AS \ 名前, \\
 \qquad --$>$ \quad age \ AS \ 年齢,\\
 \qquad --$>$ \quad dept.name \ AS \ 部署名 \\
 \qquad --$>$ FROM \ emp \\
 \qquad --$>$ JOIN \ dept \\
 \qquad --$>$ \quad ON \ emp.dept\_id \ = \ dept.id \\
 \qquad --$>$ ORDER \ BY \ ID;
\end{tcolorbox}

\begin{spacing}{0.8}        
\begin{verbatim}
+----+------------+------+--------+
| ID | 名前       | 年齢 | 部署名 |
+----+------------+------+--------+
|  1 | 菅原文太   |   40 | 総務部 |
|  2 | 千葉真一   |   34 | 営業部 |
|  3 | 北大路欣也 |   30 | 経理部 |
|  4 | 梶芽衣子   |   26 | 営業部 |
+----+------------+------+--------+
\end{verbatim}
\end{spacing}

\textsf{ORDER BY emp.id} とするところを \textsf{ORDER BY ID} としている。

これは、1行目で \textsf{emp.id AS ID} としているので、ID を使うことができるのである。


\subsection{テーブルの指定を簡略化する}

emp とか dept とかのテーブルの指定も別名を使うことで簡略化できる。

\begin{tcolorbox}
 MariaDB [sample]$>$ SELECT \\
 \qquad --$>$ \quad e.id \ AS \ ID, \\
 \qquad --$>$ \quad e.name \ AS \ 名前, \\
 \qquad --$>$ \quad age \ AS \ 年齢,\\
 \qquad --$>$ \quad d.name \ AS \ 部署名 \\
 \qquad --$>$ FROM \ emp \ AS \ e \\
 \qquad --$>$ JOIN \ dept \ AS \ d \\
 \qquad --$>$ \quad ON \ e.dept\_id \ = \ d.id \\
 \qquad --$>$ ORDER \ BY \ ID;
\end{tcolorbox}

さらに、FROM や JOIN の後では、すぐ後ろに 別名(ここでは e や d のこと)がくる場合、{\em AS} は
省略できる。

\begin{tcolorbox}
 MariaDB [sample]$>$ SELECT \\
 \qquad --$>$ \quad e.id \ AS \ ID, \\
 \qquad --$>$ \quad e.name \ AS \ 名前, \\
 \qquad --$>$ \quad age \ AS \ 年齢,\\
 \qquad --$>$ \quad d.name \ AS \ 部署名 \\
 \qquad --$>$ FROM \ emp \ e \\
 \qquad --$>$ JOIN \ dept \ d \\
 \qquad --$>$ \quad ON \ e.dept\_id \ = \ d.id \\
 \qquad --$>$ ORDER \ BY \ ID;
\end{tcolorbox}







\newpage
\subsection{外部結合(LEFT OUTER JOIN / RIGHT OUTER JOIN)}

\subsubsection{左外部結合 LEFT OUTER JOIN}

この emp表に次のデータを追加する。

\begin{tcolorbox}
 \begin{tabular}{lcl}
  ID & : & 5 \\
  名前 & : & 成田三樹夫 \\
  年齢 & : &  38 \\
  誕生年 & : & 1935 \\
  部署ID & : & (なし) \\
 \end{tabular}
\end{tcolorbox}

\begin{tcolorbox}
 MariaDB [sample]$>$ INSERT INTO emp \\
 \qquad --$>$ (name, age, birthday) \\
 \qquad --$>$ VALUES \\
 \qquad --$>$ ('成田三樹夫', 38, 1935);
\end{tcolorbox}

\begin{spacing}{0.8}        
\begin{verbatim}
MariaDB [sample]> SELECT * FROM emp;
+----+------------+-----+----------+---------+
| id | name       | age | birthday | dept_id |
+----+------------+-----+----------+---------+
|  1 | 菅原文太   |  40 |     1933 | 001     |
|  2 | 千葉真一   |  34 |     1939 | 002     |
|  3 | 北大路欣也 |  30 |     1943 | 003     |
|  4 | 梶芽衣子   |  26 |     1947 | 002     |
|  5 | 成田三樹夫 |  38 |     1935 | NULL    |
+----+------------+-----+----------+---------+    
\end{verbatim}
\end{spacing}


このデータには dept\_id、つまり部署ID がない。たとえば社長とかの場合である。

この状態で 内部結合 をすると、どうなるか?

\begin{verbatim}
MariaDB [sample]> SELECT 
    ->   e.id AS ID, 
    ->   e.name AS 名前, 
    ->   e.age AS 年齢, 
    ->   d.name AS 部署名
    -> FROM emp e
    -> JOIN dept d
    ->   ON e.dept_id = d.id
    -> order by ID;
\end{verbatim}

\begin{spacing}{0.8}        
\begin{verbatim}
+----+------------+------+--------+
| ID | 名前       | 年齢 | 部署名 |
+----+------------+------+--------+
|  1 | 菅原文太   |   40 | 総務部 |
|  2 | 千葉真一   |   34 | 営業部 |
|  3 | 北大路欣也 |   30 | 経理部 |
|  4 | 梶芽衣子   |   26 | 営業部 |
+----+------------+------+--------+    
\end{verbatim}
\end{spacing}

成田三樹夫は、結合表には出てこない。

これを図であらわすと、このようになる。

\vspace{3mm}
\includegraphics[width=13cm]{../06-mysql/ketsugo.png}
\vspace{3mm}

成田三樹夫は部署IDがないので結合の対象ではない。

こんなときは \emph{左外部結合(LEFT [OUTER] JOIN)} を使う。

\begin{tcolorbox}
 MariaDB [sample]$>$ SELECT \\
 \qquad --$>$ \quad e.id \ AS \ ID, \\
 \qquad --$>$ \quad e.name \ AS \ 名前, \\
 \qquad --$>$ \quad e.age \ AS \ 年齢, \\ 
 \qquad --$>$ \quad d.name \ as \ 部署名 \\
 \qquad --$>$ FROM \ emp \ e \\
 \qquad --$>$ LEFT JOIN \ dept \ d \\
 \qquad --$>$ \quad ON \ e.dept\_id \ = \ d.id \\
 \qquad --$>$ ORDER \ BY \ ID;
\end{tcolorbox}

\rightline{※ LEFT OUTER JOIN と記述することもできる}

\begin{spacing}{0.8}        
\begin{verbatim}
+----+------------+------+--------+
| ID | 名前       | 年齢 | 部署名 |
+----+------------+------+--------+
|  1 | 菅原文太   |   40 | 総務部 |
|  2 | 千葉真一   |   34 | 営業部 |
|  3 | 北大路欣也 |   30 | 経理部 |
|  4 | 梶芽衣子   |   26 | 営業部 |
|  6 | 成田三樹夫 |   38 | NULL   |
+----+------------+------+--------+
\end{verbatim}
\end{spacing}

\subsubsection{右外部結合 RIGHT [OUTER] JOIN}

また、dept表をみてみると、
id : '004' が  開発部、
id : '005' が  人事部、
id : '006' が  情報システム部
であるが、
emp表には dept\_id が '004'、'005'、'006' である人はいない。

この状態で結合表をつくり、開発部などの項目も表示させるには、次のようにする。

\begin{tcolorbox}
 MariaDB [sample]$>$ SELECT \\
 \qquad --$>$ \quad e.id \ AS \ ID, \\
 \qquad --$>$ \quad e.name \ AS \ 名前, \\
 \qquad --$>$ \quad e.age \ AS \ 年齢, \\ 
 \qquad --$>$ \quad d.name \ AS \ 部署名 \\
 \qquad --$>$ FROM \ emp \ e \\
 \qquad --$>$ RIGHT JOIN \ dept \ d \\
 \qquad --$>$ \quad ON \ e.dept\_id \ = \ d.id \\
 \qquad --$>$ ORDER \ BY \ ID;
\end{tcolorbox}

\begin{spacing}{0.8}        
\begin{verbatim}
+------+------------+------+---------------+
| ID   | 名前       | 年齢 | 部署名         |
+------+------------+------+---------------+
| NULL | NULL       | NULL | 開発部        |
| NULL | NULL       | NULL | 人事部        |
| NULL | NULL       | NULL | 情報システム部 |
|    1 | 菅原文太    |   40 | 総務部        |
|    2 | 千葉真一    |   34 | 営業部        |
|    3 | 北大路欣也  |   30 | 経理部        |
|    4 | 梶芽衣子    |   26 | 営業部        |
+------+------------+------+---------------+
\end{verbatim}
\end{spacing}



\include{end}

%% 修正時刻: Tue 2023/03/28 17:33:091
