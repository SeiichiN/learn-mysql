
\documentclass[dvipdfmx]{jsarticle}


\usepackage{tcolorbox}
\usepackage{color}
\usepackage{listings, plistings}

%% ノート/latexメモ
%% http://pepper.is.sci.toho-u.ac.jp/pepper/index.php?%A5%CE%A1%BC%A5%C8%2Flatex%A5%E1%A5%E2

%% JavaScriptの設定
%% https://e8l.hatenablog.com/entry/2015/11/29/232800
\lstdefinelanguage{javascript}{
  morekeywords = [1]{ %keywords
    await, break, case, catch, class, const, continue, debugger, default, delete, 
    do, else, enum, export, extends, finally, for, function, function*, if, implements, import, in, 
    instanceof, interface, let, new, package, private, protected, public, return, static, super,
    switch, this, throw, try, typeof, var, void, while, with, yield, yield*
  },
  morekeywords = [2]{ %literal
    false, Infinity, NaN, null, true, undefined
  },
  morekeywords = [3] { %Classes
    Array, ArrayBuffer, Boolean, DataView, Date, Error, EvalError, Float32Array, Float64Array,
    Function, Generator, GeneratorFunction, Int16Array, Int32Array, Int8Array, InternalError,
    JSON, Map, Math, Number, Object, Promise, Proxy, RangeError, ReferenceError, Reflect,
    RegExp, Set, String, Symbol, SyntaxError, TypeError, URIError, Uint16Array, Uint32Array,
    Uint8Array, Uint8ClampedArray, WeakMap, WeakSet
  },
  morecomment = [l]{//},
  morecomment = [s]{/*}{*/},
  morestring = [b]{"},
  morestring = [b]{'},
  alsodigit = {-},
  sensitive = true
}

%% 修正時刻: Tue 2022/03/15 10:04:41


% Java
\lstset{% 
  frame=single,
  backgroundcolor={\color[gray]{.9}},
  stringstyle={\ttfamily \color[rgb]{0,0,1}},
  commentstyle={\itshape \color[cmyk]{1,0,1,0}},
  identifierstyle={\ttfamily}, 
  keywordstyle={\ttfamily \color[cmyk]{0,1,0,0}},
  basicstyle={\ttfamily},
  breaklines=true,
  xleftmargin=0zw,
  xrightmargin=0zw,
  framerule=.2pt,
  columns=[l]{fullflexible},
  numbers=left,
  stepnumber=1,
  numberstyle={\scriptsize},
  numbersep=1em,
  language={Java},
  lineskip=-0.5zw,
  morecomment={[s][{\color[cmyk]{1,0,0,0}}]{/**}{*/}},
  keepspaces=true,         % 空白の連続をそのままで
  showstringspaces=false,  % 空白字をOFF
}
%\usepackage[dvipdfmx]{graphicx}
\usepackage{url}
\usepackage[dvipdfmx]{hyperref}
\usepackage{amsmath, amssymb}
\usepackage{itembkbx}
\usepackage{eclbkbox}	% required for `\breakbox' (yatex added)
\usepackage{enumerate}
\usepackage[default]{cantarell}
\usepackage[T1]{fontenc}
\fboxrule=0.5pt
\parindent=1em
\definecolor{mygrey}{rgb}{0.97, 0.97, 0.97}

\makeatletter
\def\verbatim@font{\normalfont
\let\do\do@noligs
\verbatim@nolig@list}
\makeatother

\begin{document}

%\anaumeと入力すると穴埋め解答欄が作れるようにしてる。\anaumesmallで小さめの穴埋めになる。
\newcounter{mycounter} % カウンターを作る
\setcounter{mycounter}{0} % カウンターを初期化
\newcommand{\anaume}[1][]{\refstepcounter{mycounter}{#1}{\boxed{\phantom{aa}\textnormal{\themycounter}\phantom{aa}}}} %穴埋め問題の空欄作ってる。
\newcommand{\anaumesmall}[1][]{\refstepcounter{mycounter}{#1}{\boxed{\tiny{\phantom{a}\themycounter \phantom{a}}}}}%小さい版作ってる。色々改造できる。

%% 修正時刻: Tue 2022/03/15 10:04:411


\section{練習問題}

適当な場所にフォルダを作成し、``ex13-mountain'' と名づけてください。

その中に、mountain.sql と state.sql を置いてください。

そのフォルダでコマンドプロンプトを起動し、以下をおこなってください。

rootユーザーでログインし、
データベースmountain を作成してください。

そして、moutainデータベースを使用宣言してください。

その後、mountain.sql を読み込んでください。
mountain.sqlには、teizanテーブルと stateテーブルがあります。

\begin{lstlisting}[caption=作業例, numbers=none, language=sql]
 > mysql -u root -p
 Password: 
 MariaDB> source mountain.sql     (あるいは)    \. mountain.sql
\end{lstlisting}

これで、teizanテーブルと stateテーブルが読み込まれます。


\subsection{テーブル定義を書く}

以下のテーブルの定義を書いてください。

\subsubsection{gender表}

\vspace{3mm}
gender表 \\
\begin{tabular}{|c|c|} \hline
 gid & gname \\ \hline\hline
 0   & 不明  \\ \hline
 1   & 男性   \\ \hline
 2   & 女性  \\ \hline
 3   & その他 \\ \hline
\end{tabular}
\vspace{3mm}

条件
\vspace{-3mm}
\begin{enumerate}
 \item gender表の主キー(id)はchar型にする。
\end{enumerate}




\subsubsection{person表}

\vspace{3mm}
person表 \\
\begin{tabular}{|c|l|c|l|} \hline
 id & name           & gender\_id & birthday   \\ \hline\hline
 1 & 染谷将太        & 1    & 1992-09-03 \\ \hline
 2 & 二階堂ふみ      & 2    & 1994-09-21 \\ \hline
 3 & 渡辺哲          & 1    & 1950-03-11 \\ \hline
 4 & 窪塚洋介        & 1    & 1979-05-07 \\ \hline
 5 & 吉高由里子      & 2    & 1988-07-22 \\ \hline
\end{tabular}
\vspace{3mm}

条件
\vspace{-3mm}
\begin{enumerate}
 \item person表の主キー(id)は整数型で、自動連番である。
 \item person表の性別の項目は、gender表の主キーを参照している。
 \item person表の birthday カラムは、DATE型にする。
\end{enumerate}



\subsubsection{person\_teizan表}

\vspace{3mm}
person\_teizan表 \\
\begin{tabular}{|c|c|} \hline
 p\_id & t\_id \\ \hline\hline
 1 &   20 \\ \hline
 1 &   30  \\ \hline
 1 &   50  \\ \hline
 2 &   26  \\ \hline
 2 &   30  \\ \hline
 2 &   43  \\ \hline
 2 &   66  \\ \hline
 3 &   44  \\ \hline
 3 &   50  \\ \hline
 3 &   82  \\ \hline
 4 &   53  \\ \hline
 4 &   58  \\ \hline
 4 &   66  \\ \hline
 4 &   67  \\ \hline
 5 &   30  \\ \hline
 5 &   50  \\ \hline
 5 &   53  \\ \hline
 5 &   80  \\ \hline
 5 &   82  \\ \hline
\end{tabular}
\vspace{3mm}

条件
\vspace{-3mm}
\begin{enumerate}
 \item person\_teizan表は、personを表すキーとteizanを表すキーの2つが主キーと
       なっている(複合主キー)。
 \item p\_id は int型であり、主キーである。
 \item t\_id は int型であり、主キーである。
\end{enumerate}


\newpage
\subsection{person表とgender表を結合する}

次のように表示したいと思います。
person表とgender表を結合したselect文を書いてください。

\vspace{3mm}
\noindent
person表 \\
\begin{tabular}{|l|l|l|l|} \hline
id & 名前       & 性別 & 誕生日 \\ \hline\hline 
1 & 染谷将太    & 男性 & 1992-09-03 \\ \hline
2 & 二階堂ふみ  & 女性 & 1994-09-21 \\ \hline
3 & 渡辺哲      & 男性 & 1950-03-11 \\ \hline
4 & 窪塚洋介    & 男性 & 1979-05-07 \\ \hline
5 & 吉高由里子  & 女性 & 1988-07-22  \\ \hline
\end{tabular}
\vspace{3mm}

\subsection{teizan表とstate表を結合する}

次のように表示したいと思います。
teizan表とstate表を結合したselect文を書いてください。

\vspace{3mm}
\noindent
 teizan表 \\
\begin{tabular}{|l|l|l|l|l|} \hline
 id & 山岳名   & かな           &  標高(m) & 所在地   \\ \hline\hline
 20 & 八溝山    & やみぞさん      &   1022  & 福島県   \\ \hline 
 26 & 榛名山    & はるなさん      &   1449  & 群馬県    \\ \hline
 30 & 荒船山    & あらふねやま    &   1423  & 群馬県    \\ \hline
 43 & 権現山    & ごんげんやま    &   1312  & 山梨県    \\ \hline
 44 & 高尾山    & たかおさん      &    599  & 東京都    \\ \hline
 50 & 大山      & おおやま        &   1252  & 神奈川県  \\ \hline
 53 & 駒ガ岳    & こまがたけ      &   1438  & 神奈川県  \\ \hline
 58 & 弥彦山    & やひこやま      &    634  & 新潟県    \\ \hline
 66 & 身延山    & みのぶさん      &   1153  & 山梨県    \\ \hline
 67 & 高社山    & たかやしろやま  &   1351  & 長野県    \\ \hline
 66 & 身延山    & みのぶさん      &   1153  & 山梨県    \\ \hline
 80 & 伊吹山    & いぶきやま      &   1377  & 滋賀県    \\ \hline
 82 & 比叡山    & ひえいざん      &    848  & 京都府    \\ \hline
\end{tabular} \\
この表は1部を表示したものです。
\vspace{3mm}






\vspace{3mm}
\noindent
\begin{tabular}{|l|l|l|} \hline
id & 名前       & 登った山 \\ \hline\hline
1 & 染谷将太    & 八溝山, 荒船山, 大山  \\ \hline
2 & 二階堂ふみ  & 榛名山, 荒船山, 権現山, 身延山 \\ \hline
3 & 渡辺哲      & 高尾山, 大山, 比叡山 \\ \hline
4 & 窪塚洋介    & 駒ガ岳, 弥彦山, 身延山, 高社山 \\ \hline
5 & 吉高由里子  & 荒船山, 大山, 駒ガ岳, 伊吹山, 比叡山 \\ \hline
\end{tabular}
\vspace{3mm}



  \subsection{データを表示する}

以下のように表示するよう、SQL文を組み立ててください。

\vspace{3mm}
\begin{tabular}{|l|l|} \hline
 名前            & 山岳名  \\ \hline\hline
 染谷将太        & 八溝山    \\ \hline
 染谷将太        & 荒船山    \\ \hline
 染谷将太        & 大山      \\ \hline
 二階堂ふみ      & 榛名山    \\ \hline
 二階堂ふみ      & 荒船山    \\ \hline
 二階堂ふみ      & 権現山    \\ \hline
 二階堂ふみ      & 身延山    \\ \hline
 渡辺哲          & 高尾山    \\ \hline
 渡辺哲          & 大山      \\ \hline
 渡辺哲          & 比叡山    \\ \hline
 窪塚洋介        & 駒ガ岳    \\ \hline
 窪塚洋介        & 弥彦山    \\ \hline
 窪塚洋介        & 身延山    \\ \hline
 窪塚洋介        & 高社山    \\ \hline
 吉高由里子      & 荒船山    \\ \hline
 吉高由里子      & 大山      \\ \hline
 吉高由里子      & 駒ガ岳    \\ \hline
 吉高由里子      & 伊吹山    \\ \hline
 吉高由里子      & 比叡山    \\ \hline
\end{tabular}
\vspace{3mm}

\newpage
以下のように表示するよう、SQL文を組み立ててください。

\vspace{3mm}
\begin{tabular}{|c|c|c|c|c|} \hline
 名前            & 性別   & 年齢   & 山岳名    & 所在地   \\ \hline\hline
 染谷将太        & 男性   &     29 & 八溝山    & 福島県   \\ \hline    
 染谷将太        & 男性   &     29 & 荒船山    & 群馬県   \\ \hline    
 染谷将太        & 男性   &     29 & 大山      & 神奈川県 \\ \hline    
 渡辺哲          & 男性   &     71 & 高尾山    & 東京都   \\ \hline    
 渡辺哲          & 男性   &     71 & 大山      & 神奈川県 \\ \hline    
 渡辺哲          & 男性   &     71 & 比叡山    & 京都府   \\ \hline    
 窪塚洋介        & 男性   &     42 & 駒ガ岳    & 神奈川県 \\ \hline    
 窪塚洋介        & 男性   &     42 & 弥彦山    & 新潟県   \\ \hline    
 窪塚洋介        & 男性   &     42 & 身延山    & 山梨県   \\ \hline    
 窪塚洋介        & 男性   &     42 & 高社山    & 長野県   \\ \hline    
 二階堂ふみ      & 女性   &     27 & 榛名山    & 群馬県   \\ \hline    
 二階堂ふみ      & 女性   &     27 & 荒船山    & 群馬県   \\ \hline    
 二階堂ふみ      & 女性   &     27 & 権現山    & 山梨県   \\ \hline    
 二階堂ふみ      & 女性   &     27 & 身延山    & 山梨県   \\ \hline    
 吉高由里子      & 女性   &     33 & 荒船山    & 群馬県   \\ \hline    
 吉高由里子      & 女性   &     33 & 大山      & 神奈川県 \\ \hline    
 吉高由里子      & 女性   &     33 & 駒ガ岳    & 神奈川県 \\ \hline    
 吉高由里子      & 女性   &     33 & 伊吹山    & 滋賀県   \\ \hline    
 吉高由里子      & 女性   &     33 & 比叡山    & 京都府   \\ \hline    
\end{tabular}
\vspace{3mm}

(ヒント)

年齢を表示するためには、以下の関数を使うとよいです。

\vspace{3mm}
\begin{tcolorbox}
\verb!timestampdiff(YEAR, [birthdayのカラム], curdate())!
\end{tcolorbox}
\vspace{3mm}

たとえば、person表の場合、以下のようにすると、年齢が表示されます。

\begin{lstlisting}[numbers=none]
mysql> SELECT timestampdiff(YEAR, birthday, curdate()) AS 年齢 FROM person;

+--------+
| 年齢   |
+--------+
|     29 |
|     27 |
|     71 |
|     42 |
|     33 |
+--------+
\end{lstlisting}




\end{document}

%% 修正時刻: Sat May  2 15:10:04 2020


%% 修正時刻: Sat 2023/09/30 16:53:062
